% Options for packages loaded elsewhere
\PassOptionsToPackage{unicode}{hyperref}
\PassOptionsToPackage{hyphens}{url}
%
\documentclass[
]{article}
\usepackage{amsmath,amssymb}
\usepackage{lmodern}
\usepackage{iftex}
\ifPDFTeX
  \usepackage[T1]{fontenc}
  \usepackage[utf8]{inputenc}
  \usepackage{textcomp} % provide euro and other symbols
\else % if luatex or xetex
  \usepackage{unicode-math}
  \defaultfontfeatures{Scale=MatchLowercase}
  \defaultfontfeatures[\rmfamily]{Ligatures=TeX,Scale=1}
\fi
% Use upquote if available, for straight quotes in verbatim environments
\IfFileExists{upquote.sty}{\usepackage{upquote}}{}
\IfFileExists{microtype.sty}{% use microtype if available
  \usepackage[]{microtype}
  \UseMicrotypeSet[protrusion]{basicmath} % disable protrusion for tt fonts
}{}
\makeatletter
\@ifundefined{KOMAClassName}{% if non-KOMA class
  \IfFileExists{parskip.sty}{%
    \usepackage{parskip}
  }{% else
    \setlength{\parindent}{0pt}
    \setlength{\parskip}{6pt plus 2pt minus 1pt}}
}{% if KOMA class
  \KOMAoptions{parskip=half}}
\makeatother
\usepackage{xcolor}
\usepackage{longtable,booktabs,array}
\usepackage{multirow}
\usepackage{calc} % for calculating minipage widths
% Correct order of tables after \paragraph or \subparagraph
\usepackage{etoolbox}
\makeatletter
\patchcmd\longtable{\par}{\if@noskipsec\mbox{}\fi\par}{}{}
\makeatother
% Allow footnotes in longtable head/foot
\IfFileExists{footnotehyper.sty}{\usepackage{footnotehyper}}{\usepackage{footnote}}
\makesavenoteenv{longtable}
\usepackage{graphicx}
\makeatletter
\def\maxwidth{\ifdim\Gin@nat@width>\linewidth\linewidth\else\Gin@nat@width\fi}
\def\maxheight{\ifdim\Gin@nat@height>\textheight\textheight\else\Gin@nat@height\fi}
\makeatother
% Scale images if necessary, so that they will not overflow the page
% margins by default, and it is still possible to overwrite the defaults
% using explicit options in \includegraphics[width, height, ...]{}
\setkeys{Gin}{width=\maxwidth,height=\maxheight,keepaspectratio}
% Set default figure placement to htbp
\makeatletter
\def\fps@figure{htbp}
\makeatother
\usepackage[normalem]{ulem}
\setlength{\emergencystretch}{3em} % prevent overfull lines
\providecommand{\tightlist}{%
  \setlength{\itemsep}{0pt}\setlength{\parskip}{0pt}}
\setcounter{secnumdepth}{-\maxdimen} % remove section numbering
\ifLuaTeX
  \usepackage{selnolig}  % disable illegal ligatures
\fi
\IfFileExists{bookmark.sty}{\usepackage{bookmark}}{\usepackage{hyperref}}
\IfFileExists{xurl.sty}{\usepackage{xurl}}{} % add URL line breaks if available
\urlstyle{same} % disable monospaced font for URLs
\hypersetup{
  hidelinks,
  pdfcreator={LaTeX via pandoc}}

\author{}
\date{}

\begin{document}

\begin{quote}
\emph{Section A.3. Limites de funções}

Section A.3\\
\textbf{Limites de funções}

No final da Section 9.1 mencionamos o uso de \emph{limites} de funções
sem definir corretamente o que queríamos dizer. Esta seção
reconhecidamente brusca é dedicada a tornar preciso o que queremos dizer
com matemática.

Limites
\end{quote}

\begin{longtable}[]{@{}
  >{\raggedright\arraybackslash}p{(\columnwidth - 4\tabcolsep) * \real{0.3333}}
  >{\raggedright\arraybackslash}p{(\columnwidth - 4\tabcolsep) * \real{0.3333}}
  >{\raggedright\arraybackslash}p{(\columnwidth - 4\tabcolsep) * \real{0.3333}}@{}}
\toprule()
\multirow{2}{*}{\begin{minipage}[b]{\linewidth}\raggedright
\begin{longtable}[]{@{}
  >{\raggedright\arraybackslash}p{(\columnwidth - 0\tabcolsep) * \real{1.0000}}@{}}
\toprule()
\begin{minipage}[b]{\linewidth}\raggedright
\begin{quote}
✦\textbf{Definition A.3.1}\\
Seja \emph{D} ⊆ R. Um ponto limite de \emph{D} é um número real \emph{a}
tal que, para todo \emph{δ \textgreater{}} 0, existe algum \emph{x D}
tal que 0 \emph{\textless{} x a \textless{} δ}.
\end{quote}\strut
\end{minipage} \\
\midrule()
\endhead
\bottomrule()
\end{longtable}\strut
\end{minipage}} &
\multirow{2}{*}{\begin{minipage}[b]{\linewidth}\raggedright
\begin{quote}
✦
\end{quote}
\end{minipage}} & \begin{minipage}[b]{\linewidth}\raggedright
\begin{quote}
∈
\end{quote}
\end{minipage} \\
& & \begin{minipage}[b]{\linewidth}\raggedright
\begin{quote}
\textbar{} − \textbar{}
\end{quote}
\end{minipage} \\
\midrule()
\endhead
\bottomrule()
\end{longtable}

\begin{quote}
✣\textbf{Lemma A.3.2}\\
Seja \emph{D} ⊆ R. Um número real \emph{a} é um ponto limite de \emph{D}
se e somente se existe uma sequência (\emph{xn}) de elementos de
\emph{D}, que não é eventualmente constante, tal que (\emph{xn}) →
\emph{a}

.

\emph{\textbf{Proof}}\\
□

(⇒) Seja \emph{a} ∈ R e assuma que \emph{a} é um ponto limite de
\emph{D}. Para cada \emph{n} ⩾ 1, seja \emph{xn} algum 1
\end{quote}

\begin{longtable}[]{@{}
  >{\raggedright\arraybackslash}p{(\columnwidth - 2\tabcolsep) * \real{0.5000}}
  >{\raggedright\arraybackslash}p{(\columnwidth - 2\tabcolsep) * \real{0.5000}}@{}}
\toprule()
\begin{minipage}[b]{\linewidth}\raggedright
\begin{quote}
elemento de \emph{D} tal que 0 \emph{\textless{}} \textbar{}\emph{xn}
−\emph{a}\textbar{} \emph{\textless{}}
\end{quote}
\end{minipage} & \begin{minipage}[b]{\linewidth}\raggedright
\begin{quote}
.
\end{quote}
\end{minipage} \\
\midrule()
\endhead
\bottomrule()
\end{longtable}

\emph{n}

\begin{longtable}[]{@{}
  >{\raggedright\arraybackslash}p{(\columnwidth - 4\tabcolsep) * \real{0.3333}}
  >{\raggedright\arraybackslash}p{(\columnwidth - 4\tabcolsep) * \real{0.3333}}
  >{\raggedright\arraybackslash}p{(\columnwidth - 4\tabcolsep) * \real{0.3333}}@{}}
\toprule()
\begin{minipage}[b]{\linewidth}\raggedright
\begin{quote}
Evidentemente (\emph{xn})→ \emph{a}: de fato, dado \emph{ε
\textgreater{}} 0, deixando \emph{N} ⩾
\end{quote}
\end{minipage} & \begin{minipage}[b]{\linewidth}\raggedright
\includegraphics[width=1.47222in,height=0.17917in]{vertopal_a41b353a673949349e90465009d25ea6/media/image1.png}
\end{minipage} & \begin{minipage}[b]{\linewidth}\raggedright
\begin{quote}
\emph{ε} para todos \emph{n} ⩾
\end{quote}
\end{minipage} \\
\midrule()
\endhead
\bottomrule()
\end{longtable}

\begin{quote}
\emph{N}.

Além disso, a sequência (\emph{xn}) não é eventualmente constante: se
fosse, existiriam \emph{N} ⩾ 1 e \emph{b} ∈ R tais que \emph{xn} =
\emph{b} para todos \emph{n geN}. Mas então, pelo teorema da compressão
(??), teríamos
\end{quote}

\begin{longtable}[]{@{}
  >{\raggedright\arraybackslash}p{(\columnwidth - 4\tabcolsep) * \real{0.3333}}
  >{\raggedright\arraybackslash}p{(\columnwidth - 4\tabcolsep) * \real{0.3333}}
  >{\raggedright\arraybackslash}p{(\columnwidth - 4\tabcolsep) * \real{0.3333}}@{}}
\toprule()
\begin{minipage}[b]{\linewidth}\raggedright
0
⩽\emph{n}\includegraphics[width=1.975in,height=0.29305in]{vertopal_a41b353a673949349e90465009d25ea6/media/image2.png}
\end{minipage} & \begin{minipage}[b]{\linewidth}\raggedright
\begin{quote}
∞
\end{quote}
\end{minipage} & \begin{minipage}[b]{\linewidth}\raggedright
\begin{quote}
\emph{n} ∞
\end{quote}
\end{minipage} \\
\midrule()
\endhead
\bottomrule()
\end{longtable}

\begin{quote}
e então \emph{b} = \emph{a}. Mas isso contradiz o fato de que
\textbar{}\emph{xn} −\emph{a}\textbar{} \emph{\textgreater{}} 0 para
todo \emph{n} ⩾ 1.

(⇐) Seja \emph{a} ∈ R e assuma que existe uma sequência (\emph{xn}) de
elementos de \emph{D}, que não é eventualmente constante, tal que
(\emph{xn})→ \emph{um}. Então para todo \emph{δ \textgreater{}} 0 existe
algum \emph{N} ∈N tal que \textbar{}\emph{xn} −\emph{a}\textbar{}
\emph{\textless{} ε} para todos \emph{n} ⩾\emph{N}. Como (\emph{xn}) não
é eventualmente constante, existe algum \emph{n} ⩾\emph{N} tal que
\textbar{}\emph{xn} −\emph{a}\textbar{} \emph{\textgreater{}} 0---caso
contrário (\emph{xn}) seria eventualmente constante com valor \emph{a}!
Mas então \emph{xn} ∈ \emph{D} e 0 \emph{\textless{}}
\textbar{}\emph{xn} −\emph{a}\textbar{} \emph{\textless{} δ}, então
\emph{a} é um ponto

limite de \emph{D}.
\end{quote}

\begin{longtable}[]{@{}
  >{\raggedright\arraybackslash}p{(\columnwidth - 2\tabcolsep) * \real{0.5000}}
  >{\raggedright\arraybackslash}p{(\columnwidth - 2\tabcolsep) * \real{0.5000}}@{}}
\toprule()
\begin{minipage}[b]{\linewidth}\raggedright
\begin{longtable}[]{@{}
  >{\raggedright\arraybackslash}p{(\columnwidth - 0\tabcolsep) * \real{1.0000}}@{}}
\toprule()
\begin{minipage}[b]{\linewidth}\raggedright
\begin{quote}
✦\textbf{Definition A.3.3}
\end{quote}

Seja \emph{D} ⊆R. O fechamento de \emph{D} é o conjunto \emph{D} (LATEX
code: \textbackslash overline\{D\}) definido por

\emph{D} = \emph{D}∪\{\emph{a} ∈ R \textbar{} \emph{a} é um ponto limite
de \emph{D}\}

\begin{quote}
Ou seja, \emph{D} é dado por \emph{D} junto com seus pontos limites.
\end{quote}
\end{minipage} \\
\midrule()
\endhead
\bottomrule()
\end{longtable}
\end{minipage} & \begin{minipage}[b]{\linewidth}\raggedright
\begin{quote}
✦

✐\textbf{Example}

\textbf{A.3.4}
\end{quote}
\end{minipage} \\
\midrule()
\endhead
\multicolumn{2}{@{}>{\raggedright\arraybackslash}p{(\columnwidth - 2\tabcolsep) * \real{1.0000} + 2\tabcolsep}@{}}{%
} \\
\bottomrule()
\end{longtable}

\begin{quote}
Temos (0\emph{,}1) = {[}0\emph{,}1{]}. Na verdade, (0\emph{,}1) ⊆
(0\emph{,}1) já que \emph{D} ⊆ \emph{D} para todo \emph{D} ⊆ R. Além
disso, as sequências

\includegraphics[width=0.75in,height=0.18056in]{vertopal_a41b353a673949349e90465009d25ea6/media/image3.png}
não são constantes, assumem valores em (0\sout{\emph{,}1) e
c}onver\sout{gem p}ara 0 e 1 respectivamente, de modo que 0 ∈
(0\emph{,}1) e 1 ∈ (0\emph{,}1). Portanto

{[}0\emph{,}1{]} ⊆ (0\emph{,}1).

Dado \emph{a} ∈ R, se \emph{a \textgreater{}} 1, então deixar \emph{δ} =
1 − \emph{a \textgreater{}} 0 revela que \textbar{}\emph{x} −
\emph{a}\textbar{} ⩾\emph{δ} para todo \emph{x} ∈ \emph{D}; e da mesma

forma, se \emph{a \textless{}} 0, então deixar \emph{δ} = −\emph{a
\textgreater{}} 0 revela que \textbar{}\emph{x}−\emph{a}\textbar{}
⩾\emph{δ}

para todo \emph{x} ∈ \emph{D}. Portanto, nenhum elemento de
R\textbackslash{[}0\emph{,}1{]} é um elemento de \emph{D}, de modo
\end{quote}

\begin{longtable}[]{@{}
  >{\raggedright\arraybackslash}p{(\columnwidth - 2\tabcolsep) * \real{0.5000}}
  >{\raggedright\arraybackslash}p{(\columnwidth - 2\tabcolsep) * \real{0.5000}}@{}}
\toprule()
\begin{minipage}[b]{\linewidth}\raggedright
\begin{quote}
que (0\emph{,}1) = {[}0\emph{,}1{]}.
\end{quote}
\end{minipage} & \begin{minipage}[b]{\linewidth}\raggedright
◁
\end{minipage} \\
\midrule()
\endhead
\bottomrule()
\end{longtable}

\begin{quote}
✎\textbf{Exercise A.3.5}
\end{quote}

\begin{longtable}[]{@{}
  >{\raggedright\arraybackslash}p{(\columnwidth - 2\tabcolsep) * \real{0.5000}}
  >{\raggedright\arraybackslash}p{(\columnwidth - 2\tabcolsep) * \real{0.5000}}@{}}
\toprule()
\begin{minipage}[b]{\linewidth}\raggedright
\begin{quote}
Seja \emph{a,b} ∈ R com \emph{a \textless{} b}. Prove que (\emph{a,b}) =
(\emph{a,b}{]} = {[}\emph{a,b}) = {[}\emph{a,b}{]}.
\end{quote}
\end{minipage} & \begin{minipage}[b]{\linewidth}\raggedright
◁
\end{minipage} \\
\midrule()
\endhead
\bottomrule()
\end{longtable}

\begin{quote}
❖\textbf{Convention A.3.6}\\
Para o restante desta seção, sempre que declararmos \emph{f} : \emph{D}
→ R como uma função, será assumido que

o domínio \emph{D} é um subconjunto de R, e que todo ponto de \emph{D} é
um ponto limite de \emph{D}. Em outras palavras,

\emph{D} não possui \emph{pontos isolados}, que são pontos separados de
todos os outros elementos de \emph{D} por uma

distância positiva. Por exemplo, no conjunto (0\emph{,}1{]}∪\{2\}, o
elemento 2 ∈ R é um ponto isolado. ◁
\end{quote}

\begin{longtable}[]{@{}
  >{\raggedright\arraybackslash}p{(\columnwidth - 2\tabcolsep) * \real{0.5000}}
  >{\raggedright\arraybackslash}p{(\columnwidth - 2\tabcolsep) * \real{0.5000}}@{}}
\toprule()
\begin{minipage}[b]{\linewidth}\raggedright
\begin{longtable}[]{@{}
  >{\raggedright\arraybackslash}p{(\columnwidth - 0\tabcolsep) * \real{1.0000}}@{}}
\toprule()
\begin{minipage}[b]{\linewidth}\raggedright
\begin{quote}
✦\textbf{Definition A.3.7}
\end{quote}

Seja \emph{f} : \emph{D} → R uma função, seja \emph{a} ∈ \emph{D}, e
seja \emph{ℓ} ∈ R. Dizemos que \emph{ℓ} é um limit de \emph{f}(\emph{x})

\begin{quote}
quando \emph{x} se aproxima de \emph{a} se
\end{quote}

∀\emph{ε \textgreater{}} 0\emph{,} ∃\emph{δ \textgreater{}} 0\emph{,}
∀\emph{x} ∈ \emph{D,} 0 \emph{\textless{}}
\textbar{}\emph{x}−\emph{a}\textbar{} \emph{\textless{} δ} ⇒
\textbar{}\emph{f}(\emph{x})−\emph{ℓ}\textbar{} \emph{\textless{} ε}\\
Em outras palavras, para valores de \emph{x} ∈ \emph{D} próximos de
\emph{a} (mas não iguais a \emph{a}), os

\begin{quote}
valores de \emph{f}(\emph{x}) tornam-se arbitrariamente próximos de
\emph{ℓ}.
\end{quote}

Escrevemos `\emph{f}(\emph{x}) → \emph{ℓ} como \emph{x} → \emph{a}'
(LATEX code: \textbackslash to) para denotar a afirmação de que

\begin{quote}
\emph{ℓ} é um limite de \emph{f}(\emph{x}) conforme \emph{x} se aproxima
\emph{a}.
\end{quote}\strut
\end{minipage} \\
\midrule()
\endhead
\bottomrule()
\end{longtable}\strut
\end{minipage} & \begin{minipage}[b]{\linewidth}\raggedright
\begin{quote}
✦

✐\textbf{Example}

\textbf{A.3.8}
\end{quote}

Defina \emph{f} : R → R
\end{minipage} \\
\midrule()
\endhead
\bottomrule()
\end{longtable}

\begin{quote}
por \emph{f}(\emph{x}) = \emph{x} para todo \emph{x} ∈ R. Então
\emph{f}(\emph{x}) → 0 como \emph{x} → 0. Para ver isso, seja \emph{ε
\textgreater{}} 0 e defina \emph{δ} = \emph{ε \textgreater{}} 0.

Então, para todo \emph{x} ∈ R, se 0 \emph{\textless{}}
\textbar{}\emph{x}−\emph{a}\textbar{} \emph{\textless{} δ} = \emph{ε},
então
\end{quote}

\textbar{}\emph{f}(\emph{x})− \emph{f}(\emph{uma})\textbar{} =
\textbar{}\emph{x}−\emph{a}\textbar{} \emph{\textless{} ε}

\begin{quote}
como requerido. ◁\emph{Section A.3. Limites de funções} 49

✎\textbf{Exercise A.3.9}\\
Seja \emph{f} : \emph{D} → R uma função, seja \emph{a} ∈ \emph{D} e seja
\emph{ℓ} ∈ R. Corrija alguma sequência (\emph{xn}) de elementos de
\emph{D}, não eventualmente constante, tal que (\emph{xn}) → \emph{a}.
Prove que se \emph{f}(\emph{x}) → \emph{ℓ} como \emph{x} → \emph{a},
então a sequência (\emph{f}(\emph{xn})) converge para \emph{ℓ}. ◁

O próximo exercício diz-nos que os limites das funções são únicos, desde
que existam. Sua prova se parece muito com o resultado análogo que
provamos para sequências em ??.

✎\textbf{Exercise A.3.10}\\
Seja \emph{f} : \emph{D} → R uma função, seja \emph{a} ∈ \emph{D}, e
\emph{ℓ}1\emph{,ℓ}2 ∈ R. Prove que se \emph{f}(\emph{x}) → \emph{ℓ}1
como \emph{x} → \emph{a}, e \emph{f}(\emph{x}) → \emph{ℓ}2 como \emph{x}
→ \emph{a}, então \emph{ℓ}1 = \emph{ℓ}2. ◁

Quando o domínio \emph{D} de uma função \emph{f} : \emph{D} → R é
ilimitado, também podemos estar interessados em descobrir como os
valores de \emph{f}(\emph{x}) se comportam como \emph{x} ∈ \emph{D} fica
(positiva ou negativamente) cada vez maior.
\end{quote}

\begin{longtable}[]{@{}
  >{\raggedright\arraybackslash}p{(\columnwidth - 2\tabcolsep) * \real{0.5000}}
  >{\raggedright\arraybackslash}p{(\columnwidth - 2\tabcolsep) * \real{0.5000}}@{}}
\toprule()
\begin{minipage}[b]{\linewidth}\raggedright
\begin{longtable}[]{@{}
  >{\raggedright\arraybackslash}p{(\columnwidth - 0\tabcolsep) * \real{1.0000}}@{}}
\toprule()
\begin{minipage}[b]{\linewidth}\raggedright
\begin{quote}
\textbf{Definition A.3.11}\\
Seja \emph{f} : \emph{D} → R uma função e seja \emph{ℓ} ∈ R. Se \emph{D}
é ilimitado acima --- isto é, para todo \emph{p} ∈
\end{quote}

R, existe \emph{x} ∈ \emph{D} com \emph{x \textgreater{} p} --- então
dizemos \emph{ℓ} é um limite de \emph{f}(\emph{x}) à medida que \emph{x}

\begin{quote}
aumenta sem limites se
\end{quote}

∀\emph{ε \textgreater{}} 0\emph{,} ∃\emph{p} ∈ R\emph{,} ∀\emph{x} ∈
\emph{D, x \textgreater{} p} ⇒
\textbar{}\emph{f}(\emph{x})−\emph{ℓ}\textbar{} \emph{\textless{} ε}

Escrevemos `\emph{f}(\emph{x}) → \emph{ℓ} as \emph{x} → ∞' (LATEX code:
\textbackslash infty) para denotar a afirmação de que

\begin{quote}
\emph{ℓ} é um limite de \emph{f}(\emph{x}) como \emph{x} aumenta sem
limite.
\end{quote}

Da mesma forma, se \emph{D} é ilimitado abaixo --- isto é, para todo
\emph{p} ∈ R, existe \emph{x} ∈ \emph{D} com

\begin{quote}
\emph{x \textless{} p}--- então dizemos \emph{ell} é um limite de
\emph{f}(\emph{x}) conforme \emph{x} diminui sem limite se
\end{quote}

∀\emph{ε \textgreater{}} 0\emph{,} ∃\emph{p} ∈ R\emph{,} ∀\emph{x} ∈
\emph{D, x \textless{} p} ⇒
\textbar{}\emph{f}(\emph{x})−\emph{ℓ}\textbar{} \emph{\textless{} ε}

\begin{quote}
Escrevemos `\emph{f}(\emph{x}) → \emph{ℓ} como \emph{x} → −∞' para
denotar a afirmação de que \emph{ℓ} é um limite de \emph{f}(\emph{x}) à
medida que \emph{x} diminui sem limite.
\end{quote}\strut
\end{minipage} \\
\midrule()
\endhead
\bottomrule()
\end{longtable}

→ R a função definida por \emph{f}(\emph{x}) =\strut
\end{minipage} & \begin{minipage}[b]{\linewidth}\raggedright
\begin{quote}
✦

✐\textbf{Example}

\textbf{A.3.12}

Seja \emph{f} : R

\emph{x}\textbar{} \textbar{} para
\end{quote}
\end{minipage} \\
\midrule()
\endhead
\bottomrule()
\end{longtable}

todo \emph{x} ∈ R. Então: 1+ \emph{x}

\begin{longtable}[]{@{}
  >{\raggedright\arraybackslash}p{(\columnwidth - 8\tabcolsep) * \real{0.2000}}
  >{\raggedright\arraybackslash}p{(\columnwidth - 8\tabcolsep) * \real{0.2000}}
  >{\raggedright\arraybackslash}p{(\columnwidth - 8\tabcolsep) * \real{0.2000}}
  >{\raggedright\arraybackslash}p{(\columnwidth - 8\tabcolsep) * \real{0.2000}}
  >{\raggedright\arraybackslash}p{(\columnwidth - 8\tabcolsep) * \real{0.2000}}@{}}
\toprule()
\multicolumn{3}{@{}>{\raggedright\arraybackslash}p{(\columnwidth - 8\tabcolsep) * \real{0.6000} + 4\tabcolsep}}{%
\begin{minipage}[b]{\linewidth}\raggedright
\begin{quote}
\emph{f}(\emph{x}) → 1 como \emph{x} → ∞. Para ver isso, seja \emph{ε
\textgreater{}} 0 e defina \emph{p}
\includegraphics[width=0.71528in,height=0.18056in]{vertopal_a41b353a673949349e90465009d25ea6/media/image4.png}
\end{quote}
\end{minipage}} & \begin{minipage}[b]{\linewidth}\raggedright
. Então, para
\end{minipage} &
\multirow{2}{*}{\begin{minipage}[b]{\linewidth}\raggedright
\begin{quote}
1. Por isso:
\end{quote}
\end{minipage}} \\
\begin{minipage}[b]{\linewidth}\raggedright
\begin{quote}
todo \emph{x \textgreater{} p}, temos \emph{x \textgreater{}} 0, de modo
que \emph{f}
\end{quote}
\end{minipage} & \begin{minipage}[b]{\linewidth}\raggedright
\begin{quote}
\includegraphics[width=0.5375in,height=0.26528in]{vertopal_a41b353a673949349e90465009d25ea6/media/image5.png}
\end{quote}
\end{minipage} & \begin{minipage}[b]{\linewidth}\raggedright
\begin{quote}
\includegraphics[width=0.33611in,height=0.17917in]{vertopal_a41b353a673949349e90465009d25ea6/media/image6.png}
\end{quote}
\end{minipage} & \begin{minipage}[b]{\linewidth}\raggedright
\begin{quote}
, e \emph{x}
\end{quote}
\end{minipage} \\
\midrule()
\endhead
\bottomrule()
\end{longtable}

\emph{x}

\begin{quote}
\includegraphics[width=2.68889in,height=0.34861in]{vertopal_a41b353a673949349e90465009d25ea6/media/image7.png}

como requerido.

\emph{f}(\emph{x}) → −1 como \emph{x} → ∞. Para ver isso, seja \emph{ε
\textgreater{}} 0 e defina \emph{p} = min\{−1\emph{,}
\uline{−}\emph{ε}\uline{1}\}. Então,
\end{quote}

\begin{longtable}[]{@{}
  >{\raggedright\arraybackslash}p{(\columnwidth - 2\tabcolsep) * \real{0.5000}}
  >{\raggedright\arraybackslash}p{(\columnwidth - 2\tabcolsep) * \real{0.5000}}@{}}
\toprule()
\begin{minipage}[b]{\linewidth}\raggedright
\begin{quote}
para todo \emph{x \textless{} p}, temos \emph{x \textless{}} 0, de modo
que
\emph{f}\includegraphics[width=0.46944in,height=0.17639in]{vertopal_a41b353a673949349e90465009d25ea6/media/image8.png}
\end{quote}
\end{minipage} & \begin{minipage}[b]{\linewidth}\raggedright
\begin{quote}
\emph{x}, e \emph{x \textless{}} \uline{−}\emph{ε}\uline{1} +1. Por
isso:
\end{quote}
\end{minipage} \\
\midrule()
\endhead
\bottomrule()
\end{longtable}

\begin{quote}
\includegraphics[width=3.00556in,height=0.35278in]{vertopal_a41b353a673949349e90465009d25ea6/media/image9.png}

como requerido. Então \emph{f}(\emph{x}) → 1 como \emph{x} → ∞ e
\emph{f}(\emph{x}) → −1 como \emph{x} → −∞. ◁

✎\textbf{Exercise A.3.13}\\
Seja \emph{f} : \emph{D} → R uma função e \emph{ℓ}1\emph{,ℓ}2 ∈ R. Prove
que se \emph{D} é ilimitado acima, e se \emph{f}(\emph{x}) → \emph{ℓ}1
como \emph{x} → ∞ e

\emph{f}(\emph{x}) → \emph{ℓ}2 como \emph{x} → ∞, então \emph{ℓ}1 =
\emph{ℓ}2. Prove o resultado análogo para limites como \emph{x} → −∞ no
caso em

que \emph{D} é ilimitado abaixo. ◁

Os resultados de Exercises A.3.10 and A.3.13 justificam a seguinte
definição.
\end{quote}

\begin{longtable}[]{@{}
  >{\raggedright\arraybackslash}p{(\columnwidth - 2\tabcolsep) * \real{0.5000}}
  >{\raggedright\arraybackslash}p{(\columnwidth - 2\tabcolsep) * \real{0.5000}}@{}}
\toprule()
\begin{minipage}[b]{\linewidth}\raggedright
\begin{longtable}[]{@{}
  >{\raggedright\arraybackslash}p{(\columnwidth - 0\tabcolsep) * \real{1.0000}}@{}}
\toprule()
\begin{minipage}[b]{\linewidth}\raggedright
\begin{quote}
\textbf{Definition A.3.14}\\
Seja f : D → R e seja a ∈ {[}−∞,∞{]}. Supondo que os limites sejam bem
alinhados e existam, escrevemos Lim f(x) para denotar o único número
real \emph{ℓ} ∈ R tal que: x→a
\end{quote}

f(\emph{x})→\emph{ℓ} como \emph{x}→\emph{a}.\strut
\end{minipage} \\
\midrule()
\endhead
\bottomrule()
\end{longtable}\strut
\end{minipage} & \begin{minipage}[b]{\linewidth}\raggedright
\begin{quote}
✦
\end{quote}
\end{minipage} \\
\midrule()
\endhead
\bottomrule()
\end{longtable}

\end{document}

