% Options for packages loaded elsewhere
\PassOptionsToPackage{unicode}{hyperref}
\PassOptionsToPackage{hyphens}{url}
%
\documentclass[
]{article}
\usepackage{amsmath,amssymb}
\usepackage{lmodern}
\usepackage{iftex}
\ifPDFTeX
  \usepackage[T1]{fontenc}
  \usepackage[utf8]{inputenc}
  \usepackage{textcomp} % provide euro and other symbols
\else % if luatex or xetex
  \usepackage{unicode-math}
  \defaultfontfeatures{Scale=MatchLowercase}
  \defaultfontfeatures[\rmfamily]{Ligatures=TeX,Scale=1}
\fi
% Use upquote if available, for straight quotes in verbatim environments
\IfFileExists{upquote.sty}{\usepackage{upquote}}{}
\IfFileExists{microtype.sty}{% use microtype if available
  \usepackage[]{microtype}
  \UseMicrotypeSet[protrusion]{basicmath} % disable protrusion for tt fonts
}{}
\makeatletter
\@ifundefined{KOMAClassName}{% if non-KOMA class
  \IfFileExists{parskip.sty}{%
    \usepackage{parskip}
  }{% else
    \setlength{\parindent}{0pt}
    \setlength{\parskip}{6pt plus 2pt minus 1pt}}
}{% if KOMA class
  \KOMAoptions{parskip=half}}
\makeatother
\usepackage{xcolor}
\usepackage{longtable,booktabs,array}
\usepackage{multirow}
\usepackage{calc} % for calculating minipage widths
% Correct order of tables after \paragraph or \subparagraph
\usepackage{etoolbox}
\makeatletter
\patchcmd\longtable{\par}{\if@noskipsec\mbox{}\fi\par}{}{}
\makeatother
% Allow footnotes in longtable head/foot
\IfFileExists{footnotehyper.sty}{\usepackage{footnotehyper}}{\usepackage{footnote}}
\makesavenoteenv{longtable}
\usepackage{graphicx}
\makeatletter
\def\maxwidth{\ifdim\Gin@nat@width>\linewidth\linewidth\else\Gin@nat@width\fi}
\def\maxheight{\ifdim\Gin@nat@height>\textheight\textheight\else\Gin@nat@height\fi}
\makeatother
% Scale images if necessary, so that they will not overflow the page
% margins by default, and it is still possible to overwrite the defaults
% using explicit options in \includegraphics[width, height, ...]{}
\setkeys{Gin}{width=\maxwidth,height=\maxheight,keepaspectratio}
% Set default figure placement to htbp
\makeatletter
\def\fps@figure{htbp}
\makeatother
\usepackage[normalem]{ulem}
\setlength{\emergencystretch}{3em} % prevent overfull lines
\providecommand{\tightlist}{%
  \setlength{\itemsep}{0pt}\setlength{\parskip}{0pt}}
\setcounter{secnumdepth}{-\maxdimen} % remove section numbering
\ifLuaTeX
  \usepackage{selnolig}  % disable illegal ligatures
\fi
\IfFileExists{bookmark.sty}{\usepackage{bookmark}}{\usepackage{hyperref}}
\IfFileExists{xurl.sty}{\usepackage{xurl}}{} % add URL line breaks if available
\urlstyle{same} % disable monospaced font for URLs
\hypersetup{
  hidelinks,
  pdfcreator={LaTeX via pandoc}}

\author{}
\date{}

\begin{document}

\begin{quote}
Uma descida infinita em Matemática Pura

\emph{∞∞∞∞∞∞∞}\\
...
\end{quote}

POR CLIVE NEWSTEAD\\
\emph{Com adaptações de Jonas Galvao Xavier/Tradução por George Galvao
Moreira}

\emph{Última atualização em Thursday 7thMarch 2024}

Trabalho Original © 2023 Clive Newstead, Todos Direitos Reservados.
Adaptatações © 2024 Jonas Galvao Xavier, Todos Direitos Reservados.

Prévia da Primeira Edição(Em breve)

ISBN 978-1-950215-00-3 (Brochura)\\
ISBN 978-1-950215-01-0 (Capa dura)

Uma cópia gratuita de \emph{Uma Descida Infinita Em Matemática Pura}
pode ser obtido através do site do livro:

Este livro, suas figuras e fontes TEX são lançadas sob uma atribuição
cri-ativa internacional "Creative Commons Attribution--ShareAlike 4.0
In-ternational Licence". O texto completo da licença está replicado no
fim do livro, e pode ser achada no site Creative Commons:

\begin{quote}
\emph{Para meus Pais,}\\
\emph{Imogen e Matthew (1952--2021),}\\
\emph{Que me ensinaram tanto.}
\end{quote}

\textbf{Contents}

\textbf{Preface} \textbf{vii}

\textbf{Reconhecimentos} \textbf{xiii}

\textbf{0} \textbf{Começando} \textbf{1}

\begin{quote}
0.E Chapter 0 exercises . . . . . . . . . . . . . . . . . . . . . . . .
. . . . 20
\end{quote}

\textbf{I} \textbf{Conceitos centrais} \textbf{23}

v

vi \emph{Contents}

\textbf{Apêndice} \textbf{27}

\textbf{A Miscelânea matemática} \textbf{27}

\begin{quote}
A.1 Definir fundamentos teóricos . . . . . . . . . . . . . . . . . . . .
. . . 28

A.2 Construções dos conjuntos de números . . . . . . . . . . . . . . . .
. . 29

A.3 Limites de funções . . . . . . . . . . . . . . . . . . . . . . . . .
. . . 47
\end{quote}

\textbf{B} \textbf{Dicas para exercícios selecionados} \textbf{51}

\textbf{Indices} \textbf{55}

\textbf{Índice de tópicos} \textbf{55}

\textbf{Índice de vocabulário} \textbf{55}

\textbf{Índice de notação} \textbf{57}

\textbf{Índices de comandos LATEX} \textbf{57}

\textbf{Licence} \textbf{61}

vi

\textbf{Prefácio}

Olá, e obrigado por tirar tempo pra ler essa rápida introdução à
\emph{Uma Descida Infinita em Pura Matemática}! A versão mais recente do
livro, em inglês, está gratuitamente disponível para download no
seguinte site:

O site também inclui informações sobre mudanças entre diferentes versões
do livro, um arquivo de versões prévias, e alguns recursos para usar
LATEX (veja também \textbf{??}).

\textbf{Sobre o livro}

Um estudante em uma típica classe de cálculo vai aprender sobre a "regra
da cadeia" e posteriormente a como usa-lá para resolver alguns problemas
pré-escritos de "regra de cadeia" assim como calcular a derivada de
sin(1 + \emph{x}2) com respeito a \emph{x}, ou talvez re-solver um
problema de palavras envolvendo taxas de variação relacionadas a
mudança. A expectativa é que o estudante aplique corretamente a regra da
cadeia para derivar a re-sposta correta, e mostrar trabalho suficiente
para ser crível . Nesse sentido, o estudante é um \emph{consumidor} da
matemática. Eles recebem a regra da cadeia como uma ferramenta, a ser
aceita sem questionamento, para então resolver um grupo pequeno de
problemas.

O objetivo desse livro é ajudar o leitor a fazer a transição: de um
\emph{consumidor} da matemática para um \emph{produtor} dela. É isso que
significa `pura' matemática. Enquanto um consumidor da matemática pode
aprender a regra da cadeia e usa-la para calcular uma derivada, um
produtor da matemática pode derivar a regra da cadeia a partir de uma
rigorosa definição de uma derivada, e então provar mais versões
abstratas da regra da cadeia em contextos mais gerais (como análise
multivariável).

Dos consumidores da matemática é esperado que digam como usaram suas
ferramentas para encontrar suas respostas. Produtores da matemática, por
outro lado, tem muito mais a fazer: Eles precisam estar prontos para
acompanhar as definições e hipóteses,

vii

viii \emph{Prefácio}

juntar os dados em novas e interessantes formas, e fazer suas próprias
definições de conceitos matemáticos. Ainda mais, uma vez que eles
fizeram isso, eles devem comu-nicar suas descobertas de uma forma que
outros considerem inteligível, e eles precisam convencer outros que
aquilo que eles fizeram está correto, apropriado e que vale a pena.

É essa a transição do consumo para produção de matemática que guiou os
princípios usados para planejar e escrever esse livro. Em particular:

• \textbf{Comunicação.} Acima de tudo, este livro visa ajudar o leitor a
obter alfabetização matemática e se expressar matematicamente. Isso
ocorre em muitos níveis de ampli-ação. Por exemplo, considere a seguinte
expressão:

\emph{∀x ∈} R\emph{,} {[}\emph{¬}(\emph{x} = 0) \emph{⇒} (\emph{∃y ∈}
R\emph{, y}2\emph{\textless{} x}2){]}

\begin{quote}
Depois de trabalhar com esse livro, você será capaz de dizer o que os
símbolos \emph{∀}, \emph{∈}, R, \emph{¬}, \emph{⇒} e \emph{∃} significam
intuitivamente e como eles são definidos precisamente. Mas você terá
também que saber interpretar o que a expressão significa como um todo,
explicar o que significa em termos simples para que outra pessoa entenda
\emph{sem} usar símbolos confusos, provar que aquilo é verdade, e
comunicar sua prova para outra pessoa de uma forma clara e concisa.

Os tipos de ferramentas necessárias para fazer isso estão desenvolvidas
nos capítulos principais do livro, e mais foco é dado para o lado
escrito das coisas em \textbf{??}.
\end{quote}

• \textbf{Inquérito.} As pessoas aprendem mais quando elas descobrem as
coisas por elas mesmas. Usando esse principio ao extremo o livro estaria
em branco. No entanto, eu acredito que é importante incorporar aspectos
da aprendizagem baseada em invest-igação no texto.

\begin{quote}
Este principio manifesta-se na medida que existem exercícios espalhados
pelo texto, muitos dos quais simplesmente requerem que você prove um
resultado. Muitos leitores vão achar isso frustrante, mas isso é por um
bom motivo: esses exercícios servem como pontos de controle para
garantir que seu entendimento desse material é suficiente para proceder.
Aquele sentimento de frustração é o que se chamaria de aprendizado ---
abrace-o!
\end{quote}

• \textbf{Estrategia.} Uma prova matemática é muito parecida com um
quebra-cabeça. Em qualquer fase dada em uma prova, você vai ter algumas
definições, premissas e res-ultados que estão disponíveis para ser
usados, e você precisa junta-los usando as re-gras lógicas à sua
disposição. Por todo livro, e particularmente nos primeiros capítu-los,
eu fiz um esforço para destacar estratégias úteis de prova em qualquer
lugar que elas surjam.

• \textbf{Conteúdo.} Não há muito sentido em aprender matemática se você
não tem nenhum conceito para prova. Com isso em mente, \textbf{??}
inclui vários capítulos dedicados à introdução de algumas áreas
temáticas em matemática pura, tanto quanto teoria dos números,
combinatórias, análise e teoria da probabilidade.

viii

\emph{Prefácio} ix

• \textbf{LATEX.} O \emph{de facto} padrão para composição matemática é
o LATEX. Eu acho que é importante para matemáticos aprender sobre ele
cedo de uma forma guiada, então eu escrevi um breve tutorial no
\textbf{??} e inclui código LATEX para toda nova notação conforme
definida ao longo do livro.

\textbf{Navegando o livro}

Esse livro não precisa, e enfaticamente \emph{não deve}, ser lido de
frente para trás. A ordem desse material foi escolhida de forma que o
material que aparece depois depende apenas do material que aparece
antes, mas seguir o material na ordem que é apresentado pode ser uma
experiência bastante seca.

A maioria do cursos introdutórios de matemática pura abrange, no mínimo,
lógica simbólica, conjuntos, funções e relações. Este material é o
conteúdo do Part I. Tais cursos frequentemente abrangem tópicos
adicionais da matemática pura, com exata-mente \emph{quais} tópicos
dependendo do que o curso está preparando os estudantes. Com isso em
mente, \textbf{??} serve como uma introdução para uma gama de áreas de
matemática pura, incluindo teoria dos números, combinatórias, teoria dos
conjuntos, análise real, teoria das probabilidade e teoria da ordem.

Não é necessário cobrir toda a Part I antes de seguir os tópicos na
\textbf{??}. Na verdade, intercalar material da \textbf{??} pode ser uma
maneira útil de motivar muitos dos conceitos abstratos que surgem na
Part I.

A seguinte tabela mostra dependências entre seções. Seções prévias
dentro do mesmo capítulo que uma seção deveria ser considerado
`essencial' pré-requisitos, a menos que indicado de outra forma.

ix

\begin{longtable}[]{@{}
  >{\raggedright\arraybackslash}p{(\columnwidth - 2\tabcolsep) * \real{0.5000}}
  >{\raggedright\arraybackslash}p{(\columnwidth - 2\tabcolsep) * \real{0.5000}}@{}}
\toprule()
\begin{minipage}[b]{\linewidth}\raggedright
x
\end{minipage} & \begin{minipage}[b]{\linewidth}\raggedright
\emph{Prefácio}
\end{minipage} \\
\midrule()
\endhead
\bottomrule()
\end{longtable}

\begin{longtable}[]{@{}
  >{\raggedright\arraybackslash}p{(\columnwidth - 4\tabcolsep) * \real{0.3333}}
  >{\raggedright\arraybackslash}p{(\columnwidth - 4\tabcolsep) * \real{0.3333}}
  >{\raggedright\arraybackslash}p{(\columnwidth - 4\tabcolsep) * \real{0.3333}}@{}}
\toprule()
\begin{minipage}[b]{\linewidth}\raggedright
\textbf{Parte}
\end{minipage} & \begin{minipage}[b]{\linewidth}\raggedright
\textbf{Seção}
\end{minipage} & \begin{minipage}[b]{\linewidth}\raggedright
\begin{longtable}[]{@{}
  >{\raggedright\arraybackslash}p{(\columnwidth - 4\tabcolsep) * \real{0.3333}}
  >{\raggedright\arraybackslash}p{(\columnwidth - 4\tabcolsep) * \real{0.3333}}
  >{\raggedright\arraybackslash}p{(\columnwidth - 4\tabcolsep) * \real{0.3333}}@{}}
\toprule()
\begin{minipage}[b]{\linewidth}\raggedright
\textbf{Essencial}
\end{minipage} & \begin{minipage}[b]{\linewidth}\raggedright
\textbf{Recomendado}
\end{minipage} & \begin{minipage}[b]{\linewidth}\raggedright
\textbf{útil}
\end{minipage} \\
\midrule()
\endhead
\bottomrule()
\end{longtable}
\end{minipage} \\
\midrule()
\endhead
\textbf{I} & \begin{minipage}[t]{\linewidth}\raggedright
\begin{quote}
\textbf{??}\\
\textbf{??}\\
\textbf{??}\\
\textbf{??}\\
\textbf{??}\\
\textbf{??}
\end{quote}\strut
\end{minipage} & \begin{minipage}[t]{\linewidth}\raggedright
\begin{longtable}[]{@{}
  >{\raggedright\arraybackslash}p{(\columnwidth - 4\tabcolsep) * \real{0.3333}}
  >{\raggedright\arraybackslash}p{(\columnwidth - 4\tabcolsep) * \real{0.3333}}
  >{\raggedright\arraybackslash}p{(\columnwidth - 4\tabcolsep) * \real{0.3333}}@{}}
\toprule()
\begin{minipage}[b]{\linewidth}\raggedright
0
\end{minipage} &
\multirow{4}{*}{\begin{minipage}[b]{\linewidth}\raggedright
\textbf{??}
\end{minipage}} &
\multirow{4}{*}{\begin{minipage}[b]{\linewidth}\raggedright
\textbf{??}
\end{minipage}} \\
\begin{minipage}[b]{\linewidth}\raggedright
\textbf{??}
\end{minipage} \\
\begin{minipage}[b]{\linewidth}\raggedright
\textbf{??}
\end{minipage} \\
\begin{minipage}[b]{\linewidth}\raggedright
\textbf{??}
\end{minipage} \\
\midrule()
\endhead
\textbf{??} & \textbf{??} & \multirow{2}{*}{\textbf{??}, \textbf{??}} \\
\begin{minipage}[t]{\linewidth}\raggedright
\begin{quote}
\textbf{??}, \textbf{??}
\end{quote}
\end{minipage} & \textbf{??} \\
\bottomrule()
\end{longtable}
\end{minipage} \\
\textbf{??} & \begin{minipage}[t]{\linewidth}\raggedright
\begin{quote}
\textbf{??}\\
\textbf{??}\\
\textbf{??}\\
\textbf{??}\\
\textbf{??}\\
\textbf{??}\\
\textbf{??}\\
\textbf{??}\\
\textbf{??}\\
\textbf{??}\\
\textbf{??}
\end{quote}\strut
\end{minipage} & \begin{minipage}[t]{\linewidth}\raggedright
\begin{longtable}[]{@{}
  >{\raggedright\arraybackslash}p{(\columnwidth - 4\tabcolsep) * \real{0.3333}}
  >{\raggedright\arraybackslash}p{(\columnwidth - 4\tabcolsep) * \real{0.3333}}
  >{\raggedright\arraybackslash}p{(\columnwidth - 4\tabcolsep) * \real{0.3333}}@{}}
\toprule()
\multirow{2}{*}{\begin{minipage}[b]{\linewidth}\raggedright
\textbf{??}
\end{minipage}} & \begin{minipage}[b]{\linewidth}\raggedright
\textbf{??}, \textbf{??}
\end{minipage} &
\multirow{3}{*}{\begin{minipage}[b]{\linewidth}\raggedright
\textbf{??}
\end{minipage}} \\
& \multirow{4}{*}{\begin{minipage}[b]{\linewidth}\raggedright
\textbf{??}
\end{minipage}} \\
\multirow{2}{*}{\begin{minipage}[b]{\linewidth}\raggedright
\textbf{??}
\end{minipage}} \\
& & \multirow{3}{*}{\begin{minipage}[b]{\linewidth}\raggedright
\textbf{??}
\end{minipage}} \\
\begin{minipage}[b]{\linewidth}\raggedright
\begin{quote}
\textbf{??}, \textbf{??}
\end{quote}
\end{minipage} \\
\multirow{2}{*}{\begin{minipage}[b]{\linewidth}\raggedright
\textbf{??}
\end{minipage}} &
\multirow{2}{*}{\begin{minipage}[b]{\linewidth}\raggedright
\textbf{??}
\end{minipage}} \\
& & \multirow{2}{*}{\begin{minipage}[b]{\linewidth}\raggedright
\textbf{??}, \textbf{??}
\end{minipage}} \\
\begin{minipage}[b]{\linewidth}\raggedright
\textbf{??}
\end{minipage} & \begin{minipage}[b]{\linewidth}\raggedright
\textbf{??}
\end{minipage} \\
\midrule()
\endhead
\textbf{??} & \multirow{4}{*}{\textbf{??}, \textbf{??}} &
\multirow{6}{*}{\textbf{??}} \\
\textbf{??} \\
\textbf{??} \\
\multirow{2}{*}{\textbf{??}} \\
& \multirow{2}{*}{\textbf{??}} \\
\begin{minipage}[t]{\linewidth}\raggedright
\begin{quote}
\textbf{??}, \textbf{??}
\end{quote}
\end{minipage} \\
\bottomrule()
\end{longtable}
\end{minipage} \\
\bottomrule()
\end{longtable}

Pré-requisitos são cumulativos. Por exemplo, a fim de cobrir
\textbf{??}, você deveria primeiro cobrir 0--\textbf{??} e
\textbf{????????}.

\textbf{O que os números, cores e símbolos significam}\\
Falando amplamente, o material nesse livro está fragmentado em itens
enumerados que em em uma de cinco categorias: definições, resultados,
observações, exemplos e exercí-cios. Na \textbf{??}, nós também temos
extratos de provas. Para melhorar navegabilidade, essas categorias são
distintas por nome, cor e símbolo, como indicado na seguinte tabela.

\begin{longtable}[]{@{}
  >{\raggedright\arraybackslash}p{(\columnwidth - 6\tabcolsep) * \real{0.2500}}
  >{\raggedright\arraybackslash}p{(\columnwidth - 6\tabcolsep) * \real{0.2500}}
  >{\raggedright\arraybackslash}p{(\columnwidth - 6\tabcolsep) * \real{0.2500}}
  >{\raggedright\arraybackslash}p{(\columnwidth - 6\tabcolsep) * \real{0.2500}}@{}}
\toprule()
\begin{minipage}[b]{\linewidth}\raggedright
\begin{quote}
\textbf{Categoria}
\end{quote}
\end{minipage} &
\multicolumn{2}{>{\raggedright\arraybackslash}p{(\columnwidth - 6\tabcolsep) * \real{0.5000} + 2\tabcolsep}}{%
\begin{minipage}[b]{\linewidth}\raggedright
\textbf{Símbolo}
\end{minipage}} & \begin{minipage}[b]{\linewidth}\raggedright
\textbf{Cor}
\end{minipage} \\
\midrule()
\endhead
\begin{minipage}[t]{\linewidth}\raggedright
\begin{quote}
Definições
\end{quote}
\end{minipage} &
\multicolumn{2}{>{\raggedright\arraybackslash}p{(\columnwidth - 6\tabcolsep) * \real{0.5000} + 2\tabcolsep}}{%
✦} & \textbf{Vermelho} \\
Resultados &
\multicolumn{2}{>{\raggedright\arraybackslash}p{(\columnwidth - 6\tabcolsep) * \real{0.5000} + 2\tabcolsep}}{%
✣} & \textbf{Azul} \\
Observações &
\multicolumn{2}{>{\raggedright\arraybackslash}p{(\columnwidth - 6\tabcolsep) * \real{0.5000} + 2\tabcolsep}}{%
❖} & \textbf{Roxo} \\
\begin{minipage}[t]{\linewidth}\raggedright
\begin{quote}
\textbf{Categoria}
\end{quote}
\end{minipage} &
\multicolumn{2}{>{\raggedright\arraybackslash}p{(\columnwidth - 6\tabcolsep) * \real{0.5000} + 2\tabcolsep}}{%
\textbf{Símbolo}} & \begin{minipage}[t]{\linewidth}\raggedright
\begin{quote}
\textbf{Cor}
\end{quote}
\end{minipage} \\
\begin{minipage}[t]{\linewidth}\raggedright
\begin{quote}
Exemplos
\end{quote}
\end{minipage} &
\multicolumn{2}{>{\raggedright\arraybackslash}p{(\columnwidth - 6\tabcolsep) * \real{0.5000} + 2\tabcolsep}}{%
✐} & \begin{minipage}[t]{\linewidth}\raggedright
\begin{quote}
\textbf{Verde}
\end{quote}
\end{minipage} \\
\begin{minipage}[t]{\linewidth}\raggedright
\begin{quote}
Exercícios
\end{quote}
\end{minipage} &
\multicolumn{2}{>{\raggedright\arraybackslash}p{(\columnwidth - 6\tabcolsep) * \real{0.5000} + 2\tabcolsep}}{%
✎} & \begin{minipage}[t]{\linewidth}\raggedright
\begin{quote}
\textbf{Dourado}
\end{quote}
\end{minipage} \\
\multicolumn{2}{@{}>{\raggedright\arraybackslash}p{(\columnwidth - 6\tabcolsep) * \real{0.5000} + 2\tabcolsep}}{%
\begin{minipage}[t]{\linewidth}\raggedright
\begin{quote}
Extratos de provas
\end{quote}
\end{minipage}} & ❝ & \begin{minipage}[t]{\linewidth}\raggedright
\begin{quote}
\textbf{Verde}
\end{quote}
\end{minipage} \\
\bottomrule()
\end{longtable}

Estes itens estão enumerados de acordo com próprias seções ---por
exemplo, \textbf{??} é em \textbf{??}. Definições e teoremas (Resultados
importantes) aparecem em uma caixa .

x

\begin{quote}
\emph{Prefácio} xi

Você também vai encontrar os símbolos □ e ◁ cujos significados são os
seguintes:
\end{quote}

□ \textbf{Fim de uma prova.} É um padrão em documentos matemáticos para
identifi- car quando uma prova acabou por desenhar um pequeno quadrado
ou por escre- ver `\emph{Q.E.D.}' (Este último significa \emph{quod erat
demonstrandum}, que é latin para \emph{que} \emph{deveria ser
mostrado}.)

◁ \textbf{Fim do item.} Isso \emph{não é} um uso padrão, e está incluído
apenas para ajudá-lo a identificar quando um item foi concluído e o
conteúdo principal do livro continua.

\begin{quote}
Algumas subjeções são rotuladas com o símbolo \emph{⋆}. Isso indica que
o material dessa subseção pode ser ignorada sem consequências drásticas.

\textbf{Licença}

Este livro está licenciado sob Creative Commons Attribution-ShareAlike
4.0 (CC BY-SA 4.0) licence. Isso significa que você está convidado a
compartilhar o conteúdo desse livro,desde que você dê crédito ao autor e
que qualquer cópia ou derivados desde livro sejam liberados sob a mesma
licença.

A licenca pode ser lida em sua totalidade no final do livro ou seguindo
o URL:

\textbf{Comentários e correções}

Qualquer feedback, seja de alunos, professores assistentes, instrutores
ou quaisquer outros leitores, será muito apreciado. Particularmente
úteis são correções de erros tipo-gráficos, sugestões de formas
alternativas de descrever conceitos ou provar teoremas e solicitações de
novos conteúdos. (e.g. se você conhecer um bom exemplo que possa
ilustrar um conceito,ou se tiver um conceito relevante que você
desejaria ver nesse livro).

Tal feedback pode ser mandado para o autor e adaptador por email (clive@
infinitedescent.xyz e jonas.agx@gmail.com, respectivamente).
\end{quote}

xi

xii \emph{Prefácio}

xii

\textbf{Reconhecimentos}

Quando reflito sobre o tempo que passei escrevendo este livro, fico
impressionado com o número de pessoas que tiveram algum tipo de
influência em seu conteúdo.

Este livro nunca teria existido se não fosse pelo curso 38-801 de Chad
Hershock\emph{Ensinos Baseados em Evidências ciêntificas},No qual fiz no
outono de 2014 como estudante de pós-graduação na Carnegie Mellon
University. Seu curso influenciou fortemente minha abordagem de ensino
e, em primeiro lugar, motivou-me a escrever este livro. Muitas das
decisões pedagógicas que tomei ao escrever este livro foram informadas
por pesquisas às quais fui exposto quando era aluno da turma de Chad.

O lendário professor da Carnegie Mellon, John Mackey, tem usado este
livro (em vários formatos) como notas de curso para 21-128
\emph{Conceitos Matemáticos e Provas} e 15-151 \emph{Fudamentos
Matemáticos da Ciência da Computação} Desde o outono de 2016. Sua
influência pode ser sentida ao longo de todo o livro: graças às
discussões com John, muitas provas foram reformuladas, seções
reestruturadas e explicações melhoradas. Como resultado, há alguma
sobreposição entre os exercícios deste livro e as questões nas folhas de
problemas. Estou extremamente grato por seu apoio contínuo.

Steve Awodey, que foi meu orientador de tese de doutorado, tem sido uma
fonte de inspiração para mim há muito tempo. Muitas das escolhas que fiz
ao escolher como apresentar o material deste livro baseiam-se no meu
desejo de fazer matemática. \emph{O caminho certo}---foi esse desejo que
me levou a estudar a teoria das categorias e, por fim, a me tornar aluno
de doutorado de Steve. Aprendi muito com ele e apreciei muito sua
paciência e flexibilidade em ajudar a direcionar minha pesquisa, apesar
de minha agenda lotada de ensino e de meus interesses extracurriculares
(como escrever este livro).

Talvez sem o conhecimento deles, muitas conversas esclarecedoras com as
seguintes pessoas ajudaram a moldar o material deste livro de uma forma
ou de outra: Jeremy Avigad, Deb Brandon, Santiago Cañez, Heather Dwyer,
Thomas Forster, Will Gunther, Kate Hamilton, Jessica Harrell, Bob
Harper, Brian Kell, Marsha Lovett, Ben Millwood,

xiii

xiv \emph{Reconhecimentos}

David Offner, Ruth Poproski, Emily Riehl, Hilary Schuldt, Gareth Taylor,
Katie Walsh, Emily Weiss e Andy Zucker.

Uma rede \emph{Stack Exchange} influenciou o desenvolvimento deste livro
de duas maneiras importantes. Primeiro, sou um membro ativo do
\emph{Mathematics Stack Exchange} () desde o início de 2012 e aprendi
muito sobe maneira eficaz; ocasionalmente, uma pergunta sobre
Mathematics Stack Exchange me inspira a adicionar um novo exemplo ou
exercício ao livro. Em segundo lugar, tenho feito uso frequente do
\emph{LATEX Stack Exchange} () para implementar alguns dos aspectos mais
técAEX .

O Departamento de Ciências Matemáticas da Carnegie Mellon University
apoiou-me academicamente, profissionalmente e financeiramente ao longo
do meu doutoramento e me apresentou mais oportunidades do que eu poderia
esperar para me desenvolver como professor. Este apoio é agora
continuado pelo Departamento de Matemática da Northwestern University,
onde trabalho atualmente como professor.

Gostaria também de agradecer a todos nos centros de ensino da Carnegie
Mellon e da Northwestern, no Eberly Center e no Searle Center,
respectivamente. Através de vários workshops, programas e bolsas em
ambos os centros de ensino, aprendi muito sobre como as pessoas aprendem
e transformei-me como professor. Sua abordagem da ciência do ensino e da
aprendizagem, centrada no aluno e baseada em evidências, está subjacente
a tudo o que faço como professor, inclusive ao escrever este livro ---
sua influência não pode ser subestimada.

Finalmente, e mais importante, sou grato aos mais de 1.000 alunos que já
usaram este livro para aprender matemática. Cada vez que um aluno entra
em contato comigo para tirar uma dúvida ou apontar um erro, o livro fica
melhor; isso se reflete nas dezenas de erros tipográficos que foram
corrigidos como consequência.

\begin{quote}
Clive Newstead\\
Janeiro de 2020\\
Evanston, Illinois
\end{quote}

xiv

\begin{quote}
Chapter 0

\textbf{Começando}

Antes de podermos começar a provar coisas, precisamos eliminar certos
tipos de afirm-ações que poderíamos tentar provar. Considere a seguinte
afirmação:
\end{quote}

\emph{Essa sentença é falsa.}

\begin{quote}
Isso e verdadeiro ou falso? Se você pensar sobre isso por alguns
segundos, você ficará em apuros.

Agora considere a seguinte sentença :
\end{quote}

\emph{O burro mais feliz do mundo.}

\begin{quote}
Isso e verdadeiro ou falso? Bem, não é nem uma frase; não faz sentido
nem \emph{perguntar} se é verdadeiro ou falso!

É claro que estaremos desperdiçando nosso tempo tentando escrever provas
de afirm-ações como as duas listadas acima -- precisamos restringir
nosso escopo a afirmações que possamos realmente ter uma chance de
provar (ou talvez refutar)! Isso motiva a seguinte definição (informal).
\end{quote}

\begin{longtable}[]{@{}
  >{\raggedright\arraybackslash}p{(\columnwidth - 0\tabcolsep) * \real{1.0000}}@{}}
\toprule()
\begin{minipage}[b]{\linewidth}\raggedright
✦ \textbf{Definition 0.0.1}\\
Uma \textbf{proposição} é a declaração na qual é possível atribuir a
\textbf{Valor verdadeiro} (`ver-dadeiro' ou `falso'). Se uma proposição
é verdadeira, uma \textbf{prova} da preposição é um argumento
logicamente válido que demonstra que é verdadeiro, apresentado a um
nível tal que um membro do público-alvo possa verificar a sua
veracidade.\strut
\end{minipage} \\
\midrule()
\endhead
\bottomrule()
\end{longtable}

1

\begin{quote}
2 \emph{Capítulo 0. Começando}

Assim, as afirmações dadas acima são proposições porque não há forma
possível de lhes atribuir um valor de verdade. Observe que, em
Definition 0.0.1, tudo o que importa é que \emph{faz sentido} dizer que
é verdadeiro ou falso, independentemente de realmente \emph{ser}
verdadeiro ou falso --- o valor de verdade de muitas proposições é
desconhecida, mesmo as muito simples.
\end{quote}

✎ \textbf{Exercise 0.0.2}\\
Pense em um exemplo de proposição verdadeira, de proposição falsa, de
proposição cujo valor de verdade você não conhece e de uma afirmação que
não é uma proposição.

◁

\begin{quote}
Os resultados em artigos e livros didáticos de matemática podem ser
chamados de \emph{pro-posições}, mas também podem ser chamados de
\emph{teoremas}, \emph{lemas} ou \emph{corolários} depend-endo do uso
pretendido.

• Uma \textbf{proposição} é um termo abrangente que pode ser usado para
qualquer resultado.

• Um \textbf{teorema} é um resultado chave que é particularmente
importante.

• Um \textbf{lema} é um resultado que é provado com o propósito de ser
usado na prova de um teorema.

• Um \textbf{corolário} é um resultado que segue de um teorema sem muito
esforço adicional.

Estas não são definições precisas e não pretendem ser --- você poderia
chamar cada resultado de \emph{proposição} se quisesse --- mas usar
essas palavras apropriadamente ajuda os leitores a descobrir como ler um
artigo. Por exemplo, se você quiser apenas folhear um artigo e encontrar
seus principais resultados, procure resultados rotulados como
\emph{teoremas}.

Não adianta muito tentar provar resultados se não temos nada para provar
os resulta-dos. Com isso em mente, apresentaremos agora os
\emph{conjuntos de números} e provaremos alguns resultados sobre eles no
contexto de quatro tópicos, a saber: divisão de inteiros, bases
numéricas, números racionais e irracionais e polinômios. Estes tópicos
forne-cerão contexto para o material em Part I, e servirão como uma
introdução aos tópicos abordados em \textbf{??}.

Não nos aprofundaremos muito neste capítulo. Em vez disso, pense nisso
como um exercício de aquecimento -- uma introdução rápida e leve, com
mais provas a serem fornecidas no restante do livro.
\end{quote}

2

\begin{quote}
\emph{Capítulo 0. Começando} 3

\textbf{Conjuntos}

Fundamental para a matemática é a noção de \emph{conjunto}. Estudaremos
conjuntos detalha-damente em \textbf{??}, mas você os encontrará em
todos os capítulos do livro, então levaremos algum tempo para pensar
sobre eles agora. Não trataremos conjuntos formalmente neste estágio --
por enquanto, a definição a seguir será o suficiente.
\end{quote}

\begin{longtable}[]{@{}
  >{\raggedright\arraybackslash}p{(\columnwidth - 0\tabcolsep) * \real{1.0000}}@{}}
\toprule()
\begin{minipage}[b]{\linewidth}\raggedright
✦ \textbf{Definition 0.0.3} (a ser revisado em \textbf{??})\\
Um \textbf{conjunto} é uma coleção de objetos. Os objetos nesse conjunto
são chamados \textbf{elementos} do conjunto. Se \emph{X} é um conjunto e
\emph{x} é um objeto, então escrevemos \emph{x ∈ X} (LATEX code: x
\textbackslash{} in X) para denotar a afirmação que \emph{x} é um
elemento de \emph{X}.\strut
\end{minipage} \\
\midrule()
\endhead
\bottomrule()
\end{longtable}

\begin{quote}
Os conjuntos que nos interessam em primeiro lugar são os \emph{conjuntos
de números}---isto é, conjuntos cujos elementos são tipos particulares
de \emph{número}. Neste nível introdutório, muitos detalhes serão
temporariamente varridos para debaixo do tapete; trabalharemos com um
nível de precisão apropriado ao nosso estágio atual, mas que ainda nos
permita desenvolver uma quantidade razoável de intuição.

Então aqui vamos nós. Aqui está uma linha infinita:

\includegraphics[width=4.33333in,height=\textheight]{vertopal_90446958eb2044f0ba6a30adb46b64c5/media/image1.png}

As setas indicam que se supõe que se estenda em ambas as direções sem
fim. Os pontos na linha representarão números (especificamente,
\emph{números reais}, um termo enganoso que será definido em Definition
0.0.25).

Agora vamos fixar um ponto nesta linha e rotulá-lo `0':
\end{quote}

0

\begin{quote}
\includegraphics[width=4.33333in,height=\textheight]{vertopal_90446958eb2044f0ba6a30adb46b64c5/media/image2.png}

Este ponto pode ser considerado como uma representação do número zero; é
o ponto contra o qual todos os outros números serão medidos. Os números
à esquerda de 0 na reta numérica são considerados \emph{negativos}, e os
à direita são \emph{positivos}; 0 em si não é positivo nem negativo.

Finalmente, vamos fixar uma unidade de comprimento:

Esta unidade de comprimento será utilizada, entre outras coisas, para
comparar até que ponto os outros números diferem de zero.
\end{quote}

3

\includegraphics[width=4.875in,height=1.88889in]{vertopal_90446958eb2044f0ba6a30adb46b64c5/media/image3.png}

\begin{longtable}[]{@{}
  >{\raggedright\arraybackslash}p{(\columnwidth - 2\tabcolsep) * \real{0.5000}}
  >{\raggedright\arraybackslash}p{(\columnwidth - 2\tabcolsep) * \real{0.5000}}@{}}
\toprule()
\begin{minipage}[b]{\linewidth}\raggedright
\begin{quote}
4
\end{quote}
\end{minipage} & \begin{minipage}[b]{\linewidth}\raggedright
\emph{Capítulo 0. Começando}
\end{minipage} \\
\midrule()
\endhead
\bottomrule()
\end{longtable}

\begin{longtable}[]{@{}
  >{\raggedright\arraybackslash}p{(\columnwidth - 0\tabcolsep) * \real{1.0000}}@{}}
\toprule()
\begin{minipage}[b]{\linewidth}\raggedright
✦ \textbf{Definition 0.0.4}\\
Uma linha infinita acima, junto com seu ponto zero fixo e comprimento
unitário fixo, constituem a (\textbf{real})\textbf{linha
numérica}.\strut
\end{minipage} \\
\midrule()
\endhead
\bottomrule()
\end{longtable}

\begin{quote}
Usaremos a reta numérica para construir cinco conjuntos de números de
nosso interesse: o conjunto N de \emph{números naturais} (Definition
0.0.5), o conjunto Z de \emph{inteiros} (Defini-tion 0.0.11), o conjunto
Q de \emph{números racionais} (Definition 0.0.24), o R de \emph{números
reais} (Definition 0.0.25), e

Cada um desses conjuntos tem um caráter diferente e é usado para
propósitos diferentes, como veremos mais adiante neste capítulo e ao
longo deste livro.

\textbf{Números Naturais (}N\textbf{)}

Os \emph{números naturais} são os números usados para contar---são as
respostas a questões da forma `quantos'---por exemplo, tenho \emph{três}
tios, \emph{três} guinéus porcos e \emph{zero} gatos.

Contar é uma habilidade que os humanos possuem há muito tempo; sabemos
disso porque há evidências de pessoas que usaram marcadores há dezenas
de milhares de anos. As marcas de contagem fornecem um método de contar
números pequenos: começando do zero, prossiga pelos objetos que deseja
contar um por um e faça uma marca para cada objeto. Quando terminar,
haverá tantas marcas quantos objetos. Somos ensinados desde pequenos a
contar com os dedos; este é outro exemplo de fazer marcas de registro,
onde agora, em vez de fazer uma marca, levantamos um dedo.

Fazer uma marca representa um \emph{incremento} na quantidade --- isto
é, adicionar um. Na nossa reta numérica, podemos representar um
incremento na quantidade movendo para a direita pela unidade de
comprimento. Então, a distância do zero que nos move-mos, que é igual ao
número de vezes que nos movemos para a direita pela unidade de
comprimento, é portanto, igual ao número de objetos que estão sendo
contados.
\end{quote}

✦ \textbf{Definition 0.0.5}\\
Os \textbf{números naturais} são representados pelos pontos na reta
numérica que podem ser obtidos começando em 0 e movendo-se para a
direita pela unidade de comprimento qualquer número de vezes:

\begin{longtable}[]{@{}
  >{\raggedright\arraybackslash}p{(\columnwidth - 10\tabcolsep) * \real{0.1667}}
  >{\raggedright\arraybackslash}p{(\columnwidth - 10\tabcolsep) * \real{0.1667}}
  >{\raggedright\arraybackslash}p{(\columnwidth - 10\tabcolsep) * \real{0.1667}}
  >{\raggedright\arraybackslash}p{(\columnwidth - 10\tabcolsep) * \real{0.1667}}
  >{\raggedright\arraybackslash}p{(\columnwidth - 10\tabcolsep) * \real{0.1667}}
  >{\raggedright\arraybackslash}p{(\columnwidth - 10\tabcolsep) * \real{0.1667}}@{}}
\toprule()
\begin{minipage}[b]{\linewidth}\raggedright
0
\end{minipage} & \begin{minipage}[b]{\linewidth}\raggedright
1
\end{minipage} & \begin{minipage}[b]{\linewidth}\raggedright
2
\end{minipage} & \begin{minipage}[b]{\linewidth}\raggedright
3
\end{minipage} & \begin{minipage}[b]{\linewidth}\raggedright
4
\end{minipage} & \begin{minipage}[b]{\linewidth}\raggedright
5
\end{minipage} \\
\midrule()
\endhead
\bottomrule()
\end{longtable}

\begin{quote}
Em termos mais familiares, eles são \emph{os números inteiros não
negativos}. Nós escrevemos N (LATEX code: \textbackslash mathbb\{N\})
para o conjunto de todos os números naturais; assim, a notação `\emph{n
∈} N' significa que \emph{n} é um número natural.
\end{quote}

4

\emph{Capítulo 0. Começando} 5

Os números naturais possuem uma estrutura matemática muito importante e
interess-ante, e são centrais no material em \textbf{??}. Uma
caracterização mais precisa dos números naturais será fornecida em
\textbf{??}, e uma construção matemática do conjunto de números naturais
pode ser encontrada em Section A.1 (ver Construction A.2.5). No centro
des-tas caracterizações mais precisas estarão as noções de ``zero'' e de
``adicionar um'' -- tal como fazer marcas de contagem.

Alguns autores definem os números naturais como sendo os números
inteiros \emph{positivos}, excluindo assim o zero. Consideramos 0 um
número natural, pois nosso principal uso dos números naturais será para
contar conjuntos finitos, e um conjunto sem nada é cer-tamente finito!
Dito isto, como acontece com qualquer definição matemática, a escolha
sobre se 0 \emph{∈} N ou 0 \emph{̸∈} N é uma questão de gosto ou
conveniência, e é apenas uma convenção --- não é algo que possa ser
provado ou refutado.

\textbf{Bases Numéricas}

Escrever números é algo que pode parecer fácil para você agora, mas
provavelmente levou vários anos quando criança para realmente entender o
que estava acontecendo. Historicamente, existiram muitos sistemas
diferentes para representar números simbol-icamente, chamados
\emph{sistemas numéricos}. Primeiro veio o mais primitivo de todos, os
marcadores de contagem, aparecendo na Idade da Pedra e ainda sendo
usados para alguns propósitos hoje. Milhares de anos e centenas de
sistemas de numeração de-pois, existe um sistema de numeração dominante,
compreendido em todo o mundo: o \textbf{sistema de numeração
hindu-árabe}. Este sistema de numeração consiste em dez símbolos,
chamados \emph{dígitos}. É um sistema numérico \emph{posicional}, o que
significa que a posição de um símbolo em uma cadeia determina seu valor
numérico.

Em inglês, os \emph{algarismos arábicos} são usados como os dez dígitos:

\begin{longtable}[]{@{}
  >{\raggedright\arraybackslash}p{(\columnwidth - 18\tabcolsep) * \real{0.1000}}
  >{\raggedright\arraybackslash}p{(\columnwidth - 18\tabcolsep) * \real{0.1000}}
  >{\raggedright\arraybackslash}p{(\columnwidth - 18\tabcolsep) * \real{0.1000}}
  >{\raggedright\arraybackslash}p{(\columnwidth - 18\tabcolsep) * \real{0.1000}}
  >{\raggedright\arraybackslash}p{(\columnwidth - 18\tabcolsep) * \real{0.1000}}
  >{\raggedright\arraybackslash}p{(\columnwidth - 18\tabcolsep) * \real{0.1000}}
  >{\raggedright\arraybackslash}p{(\columnwidth - 18\tabcolsep) * \real{0.1000}}
  >{\raggedright\arraybackslash}p{(\columnwidth - 18\tabcolsep) * \real{0.1000}}
  >{\raggedright\arraybackslash}p{(\columnwidth - 18\tabcolsep) * \real{0.1000}}
  >{\raggedright\arraybackslash}p{(\columnwidth - 18\tabcolsep) * \real{0.1000}}@{}}
\toprule()
\begin{minipage}[b]{\linewidth}\raggedright
0
\end{minipage} & \begin{minipage}[b]{\linewidth}\raggedright
1
\end{minipage} & \begin{minipage}[b]{\linewidth}\raggedright
2
\end{minipage} & \begin{minipage}[b]{\linewidth}\raggedright
3
\end{minipage} & \begin{minipage}[b]{\linewidth}\raggedright
4
\end{minipage} & \begin{minipage}[b]{\linewidth}\raggedright
5
\end{minipage} & \begin{minipage}[b]{\linewidth}\raggedright
6
\end{minipage} & \begin{minipage}[b]{\linewidth}\raggedright
7
\end{minipage} & \begin{minipage}[b]{\linewidth}\raggedright
8
\end{minipage} & \begin{minipage}[b]{\linewidth}\raggedright
\begin{quote}
9
\end{quote}
\end{minipage} \\
\midrule()
\endhead
\bottomrule()
\end{longtable}

O dígito mais à direita em uma cadeia está na casa das unidades e o
valor de cada dígito aumenta por um fator de dez movendo-se para a
esquerda. Por exemplo, quando escrevemos `2812', o `2' mais à esquerda
representa o número dois mil, enquanto o último `2' representa o número
dois.

O fato de existirem dez dígitos e de o sistema numérico ser baseado em
potências de dez é um acidente biológico que corresponde ao fato de a
maioria dos humanos ter dez dedos. Para muitos propósitos, isso é
inconveniente. Por exemplo, dez não tem muitos divisores positivos
(apenas quatro: 1, 2, 5 e 10) --- isto tem implicações para a facilidade
de realizar aritmética; um sistema baseado no número doze, que possui
seis divisores positivos (1, 2, 3, 4, 6 e 12), pode ser mais
conveniente. Outro exemplo é na computação e na eletrônica digital, onde
é mais conveniente trabalhar em um

5

\begin{quote}
6 \emph{Capítulo 0. Começando}

sistema \emph{binário}, com apenas dois dígitos ---0 e 1---que
representam `off' e `on' ( ou«baixa tensão» e «alta tensão»),
respetivamente; a aritmética pode então ser realizada diretamente usando
sequências de \emph{portas lógicas} em um circuito elétrico.

Portanto, vale a pena ter alguma compreensão dos sistemas numéricos
posicionais baseados em números diferentes de dez. A abstração
matemática que fazemos leva à definição de \emph{expansão base-b}.
\end{quote}

\begin{longtable}[]{@{}
  >{\raggedright\arraybackslash}p{(\columnwidth - 0\tabcolsep) * \real{1.0000}}@{}}
\toprule()
\begin{minipage}[b]{\linewidth}\raggedright
✦ \textbf{Definition 0.0.6}\\
Seja \emph{b} um número natural maior que 1. A \textbf{base-}\emph{b}
\textbf{expansão} de um número natural \emph{n} é a\emph{a}cadeia
\emph{drdr−}1 \emph{...d}0 tal que:

\begin{quote}
(i) \emph{n} = \emph{dr ·br}+\emph{dr−}1
\emph{·br−}1+\emph{···}+\emph{d}0 \emph{·b}0;\\
(ii) 0 ⩽ \emph{di \textless{} b} para cada \emph{i}; e

(iii) If \emph{n \textgreater{}} 0 então \emph{dr ̸}= 0---a base de
expansão-\emph{b} de zero é 0 em todas as bases \emph{b}.

Certas bases numéricas têm nomes; por exemplo, as expansões base 2, 3,
8, 10 e 16 são chamadas respectivamente de \emph{binário},
\emph{ternário}, \emph{octal}, \emph{decimal} e \emph{hexadecimal}.

\emph{a}Essa frase é problemática, pois assume que todo número natural
tem apenas uma base-\emph{b} expansão. Na verdade, esse fato requer
prova --- veja \textbf{??}.
\end{quote}\strut
\end{minipage} \\
\midrule()
\endhead
\bottomrule()
\end{longtable}

\begin{quote}
Antes de olharmos um exemplo de Definition 0.0.6 em ação, vamos examinar
a defin-ição, que é um pouco concisa à primeira vista.

• Condição (i) nos diz que os dígitos na cadeia nos dizem quantos de
cada potência de \emph{b} são somados para obter \emph{n}. Por exemplo,
quando \emph{b} = 10, os dígitos da direita para a esquerda indicam as
unidades, dezenas, centenas, milhares e assim por diante.

• Condição (ii) nos diz que os dígitos em uma expansão de base \emph{b}
devem ser menores que \emph{b} --- por exemplo, os dígitos de base 4 são
0, 1, 2 e 3. Se permitíssemos mais dígitos, então coisas bobas
aconteceriam --- por exemplo, se `X' fosse um novo dígito de base 10
representando o número dez, então `X2' e `102' seriam cadeias
diferentes, ambas representando o número cento e dois.

• Condição (iii) garante que a cadeia que representa um número positivo
não tenha nenhum `0' inicial --- caso contrário, por exemplo, `01423' e
`1423' seriam cadeias diferentes representando o mesmo número natural.
\end{quote}

✐ \textbf{Example 0.0.7}\\
Considere o número 1023. Sua expansão decimal (base-10) é 1023, já que

1023 = 1\emph{·}103+0\emph{·}102+2\emph{·}101+3\emph{·}100

6

\begin{quote}
\emph{Capítulo 0. Começando} 7

Sua expansão binária (base-2) é 1111111111, já que

1023 =
1\emph{·}29+1\emph{·}28+1\emph{·}27+1\emph{·}26+1\emph{·}25+1\emph{·}24+1\emph{·}23+1\emph{·}22+1\emph{·}21+1\emph{·}20

Podemos expressar números na base 36 usando os dez dígitos usuais de 0 a
9 e as vinte e seis letras de A a Z; por exemplo, A representa 10, M
representa 22 e Z representa 35.

A expansão de base 36 de 1023 é SF, já que:
\end{quote}

1023 = 28\emph{·}361+15\emph{·}360= S\emph{·}361+F\emph{·}360

◁

✎ \textbf{Exercise 0.0.8}\\
Encontre as expansões binária, ternária, octal, decimal, hexadecimal e
de base 36 do número 21127, usando as letras A--F como dígitos
adicionais para a expansão hexa- decimal ( representando os números
10--15, respectivamente) e as letras A--Z como dígitos adicionais para a
expansão da base 36. ◁

\begin{quote}
Às vezes desejamos especificar um número natural em termos de sua
expansão de base \emph{b}; temos alguma notação para isso.

✦ \textbf{Notation 0.0.9}\\
Seja \emph{b \textgreater{}} 1. Se os números
\emph{d}0\emph{,d}1\emph{,...,dr} são dígitos base-\emph{b} (no sentido
de Defini-tion 0.0.6), então nós escrevemos:

\emph{drdr−}1 \emph{...d}0(\emph{b}) = \emph{dr ·br}+\emph{dr−}1
\emph{·br−}1+\emph{···}+\emph{d}0 \emph{·b}0

para o número natural cuja expansão de base \emph{b} é \emph{drdr−}1
\emph{...d}0. Se não houver subscrito (\emph{b}) e uma base não for
especificada explicitamente, a expansão será assumida como base-10.
\end{quote}

✐ \textbf{Example 0.0.10}\\
Usando nossa nova notação,nós temos:

\begin{quote}
1023 = 1111111111(2) = 1101220(3) = 1777(8) = 1023(10) = 3FF(16) =
SF(36)
\end{quote}

◁

\begin{quote}
\textbf{Inteiros (}Z\textbf{)}

\emph{inteiros} podem ser usados para medir a diferença entre dois
números naturais. Por exemplo, suponha que eu tenha cinco maçãs e cinco
bananas. Outra pessoa, também segurando maçãs e bananas, deseja
negociar. Após a nossa troca, tenho sete maçãs e apenas uma banana.
Assim, tenho mais duas maçãs e quatro bananas a menos.
\end{quote}

7

\includegraphics[width=4.875in,height=1.79167in]{vertopal_90446958eb2044f0ba6a30adb46b64c5/media/image4.png}

\begin{quote}
8 \emph{Capítulo 0. Começando}

Como um incremento na quantidade pode ser representado movendo-se para a
direita na reta numérica pela unidade de comprimento, um
\emph{decremento} na quantidade pode, portanto, ser representado
movendo-se para a \emph{esquerda} pela unidade de comprimento. Fazer
isso dá origem aos inteiros.
\end{quote}

✦ \textbf{Definition 0.0.11}\\
Os \textbf{inteiros} são representados pelos pontos na reta numérica que
podem ser obtidos começando em 0 e movendo-se em qualquer direção pela
unidade de comprimento qualquer número de vezes:

\begin{longtable}[]{@{}
  >{\raggedright\arraybackslash}p{(\columnwidth - 20\tabcolsep) * \real{0.0909}}
  >{\raggedright\arraybackslash}p{(\columnwidth - 20\tabcolsep) * \real{0.0909}}
  >{\raggedright\arraybackslash}p{(\columnwidth - 20\tabcolsep) * \real{0.0909}}
  >{\raggedright\arraybackslash}p{(\columnwidth - 20\tabcolsep) * \real{0.0909}}
  >{\raggedright\arraybackslash}p{(\columnwidth - 20\tabcolsep) * \real{0.0909}}
  >{\raggedright\arraybackslash}p{(\columnwidth - 20\tabcolsep) * \real{0.0909}}
  >{\raggedright\arraybackslash}p{(\columnwidth - 20\tabcolsep) * \real{0.0909}}
  >{\raggedright\arraybackslash}p{(\columnwidth - 20\tabcolsep) * \real{0.0909}}
  >{\raggedright\arraybackslash}p{(\columnwidth - 20\tabcolsep) * \real{0.0909}}
  >{\raggedright\arraybackslash}p{(\columnwidth - 20\tabcolsep) * \real{0.0909}}
  >{\raggedright\arraybackslash}p{(\columnwidth - 20\tabcolsep) * \real{0.0909}}@{}}
\toprule()
\begin{minipage}[b]{\linewidth}\raggedright
\emph{−}5
\end{minipage} & \begin{minipage}[b]{\linewidth}\raggedright
\emph{−}4
\end{minipage} & \begin{minipage}[b]{\linewidth}\raggedright
\emph{−}3
\end{minipage} & \begin{minipage}[b]{\linewidth}\raggedright
\emph{−}2
\end{minipage} & \begin{minipage}[b]{\linewidth}\raggedright
\emph{−}1
\end{minipage} & \begin{minipage}[b]{\linewidth}\raggedright
0
\end{minipage} & \begin{minipage}[b]{\linewidth}\raggedright
1
\end{minipage} & \begin{minipage}[b]{\linewidth}\raggedright
2
\end{minipage} & \begin{minipage}[b]{\linewidth}\raggedright
3
\end{minipage} & \begin{minipage}[b]{\linewidth}\raggedright
4
\end{minipage} & \begin{minipage}[b]{\linewidth}\raggedright
5
\end{minipage} \\
\midrule()
\endhead
\bottomrule()
\end{longtable}

\begin{quote}
Nós escrevemos Z (LATEX code: \textbackslash mathbb\{Z\}) para o
conjunto de todos os inteiros; portanto,a notação `\emph{n ∈} Z'
significa que \emph{n} é um inteiro.

Os inteiros têm uma estrutura tão fascinante que um capítulo inteiro
deste livro é ded-icado a eles --- veja \textbf{??}. Isso tem a ver com
o fato de que, embora seja possível somar, subtrair e multiplicar dois
números inteiros e obter outro número inteiro, o mesmo não acontece com
a divisão. Este ``mau comportamento'' da divisão é o que torna os
números inteiros interessantes. Veremos agora alguns resultados básicos
sobre divisão.

\textbf{Divisão de Inteiros}

A motivação que daremos em breve para a definição dos números racionais
(Defini-tion 0.0.24) é que o resultado da divisão de um inteiro por
outro inteiro não é neces-sariamente outro inteiro. Contudo, o resultado
é \emph{às vezes} outro número inteiro; por exemplo, posso dividir seis
maçãs entre três pessoas e cada pessoa receberá um número inteiro de
maçãs. Isto torna a divisão interessante: como podemos medir o fracasso
da divisibilidade de um número inteiro por outro? Como podemos deduzir
quando um número inteiro é divisível por outro? Qual é a estrutura do
conjunto de inteiros quando visto pelas lentes da divisão? Isso motiva
Definition 0.0.12.
\end{quote}

\begin{longtable}[]{@{}
  >{\raggedright\arraybackslash}p{(\columnwidth - 0\tabcolsep) * \real{1.0000}}@{}}
\toprule()
\begin{minipage}[b]{\linewidth}\raggedright
✦ \textbf{Definition 0.0.12} (Para ser repetido in \textbf{??})\\
Seja \emph{a,b ∈} Z. Dizemos \emph{b} \textbf{divide} \emph{a} se
\emph{a} = \emph{qb} para algum inteiro \emph{q}. Existem muitas outras
maneiras de dizer que \emph{b} divide \emph{a}, como: \emph{a} é
\emph{divisível por b}, \emph{b} é um \emph{divisor} de \emph{a},
\emph{b} é um \emph{fator} de \emph{a}, ou \emph{a} é um \emph{múltiplo}
de \emph{b}.\strut
\end{minipage} \\
\midrule()
\endhead
\bottomrule()
\end{longtable}

\begin{quote}
Observe que, talvez de forma contraintuitiva, a definição de
divisibilidade não envolve a operação aritmética de divisão: ela é
definida em termos de multiplicação.
\end{quote}

8

\begin{quote}
\emph{Capítulo 0. Começando} 9
\end{quote}

✐ \textbf{Example 0.0.13}\\
O inteiro 12 é divisível por 1, 2, 3, 4, 6 e 12, desde que

\begin{quote}
12 = 12\emph{·}1 = 6\emph{·}2 = 4\emph{·}3 = 3\emph{·}4 = 2\emph{·}6 =
1\emph{·}12\\
Também é divisível pelos negativos de todos esses números; por exemplo,
12 é divisível por \emph{−}3 desde que 12 =
(\emph{−}4)\emph{·}(\emph{−}3). ◁
\end{quote}

✎ \textbf{Exercise 0.0.14}\\
Prove que 1 divide todo número inteiro e que todo número inteiro divide
0. ◁

\begin{quote}
Uma consequência de Exercise 0.0.14 é que 0 é divisível por 0. Isto é
surpreendente: durante toda a nossa vida ouvimos que não podemos dividir
por zero, mas agora descobrimos que podemos dividir zero por zero como
pode ser isso? Isso destaca por que era tão importante que a definição
de divisibilidade (Definition 0.0.12) fosse dada em termos de
multiplicação, sem usar a operação de divisão: dizer que 0 divide 0
sig-nifica simplesmente que 0 pode ser multiplicado por um inteiro para
obter 0 (o que é verdade)---mas isso não implica que a expressão
`\uline{0 0}' possa (ou deva) ser definida de forma significativa.

usando Definition 0.0.12, podemos provar alguns fatos básicos gerais
sobre a divisibil-idade.

✣ \textbf{Proposition 0.0.15}\\
Seja \emph{a,b,c ∈} Z. Se \emph{c} divide \emph{b} e \emph{b} divide
\emph{a}, então \emph{c} divide \emph{a}.

\emph{\textbf{Proof}}\\
Suponha que \emph{c} divide \emph{b} e \emph{b} divide \emph{a}. Por
Definition 0.0.12, segue que
\end{quote}

\begin{longtable}[]{@{}
  >{\raggedright\arraybackslash}p{(\columnwidth - 4\tabcolsep) * \real{0.3333}}
  >{\raggedright\arraybackslash}p{(\columnwidth - 4\tabcolsep) * \real{0.3333}}
  >{\raggedright\arraybackslash}p{(\columnwidth - 4\tabcolsep) * \real{0.3333}}@{}}
\toprule()
\begin{minipage}[b]{\linewidth}\raggedright
\emph{b} = \emph{qc}
\end{minipage} & \begin{minipage}[b]{\linewidth}\raggedright
e
\end{minipage} & \begin{minipage}[b]{\linewidth}\raggedright
\begin{quote}
\emph{a} = \emph{rb}
\end{quote}
\end{minipage} \\
\midrule()
\endhead
\bottomrule()
\end{longtable}

\begin{quote}
para alguns inteiros \emph{q} e \emph{r}. Usando a primeira equação,
podemos substituir \emph{qc} por \emph{b} na segunda equação, para
obter\\
\emph{a} = \emph{r}(\emph{qc})

Mas \emph{r}(\emph{qc}) = (\emph{rq})\emph{c}, e \emph{rq} é um número
inteiro, então segue de Definition 0.0.12 que \emph{c} divide \emph{a}.
□
\end{quote}

✎ \textbf{Exercise 0.0.16}\\
Sejam \emph{a,b,d ∈} Z. Suponha que \emph{d} divide \emph{a} e \emph{d}
divide \emph{b}. Dados inteiros \emph{u} e \emph{v}, prove que \emph{d}
divide \emph{au}+\emph{bv}. ◁

\begin{quote}
Alguns conceitos familiares, como par e impar, podem ser caracterizados
em termos de divisibilidade.
\end{quote}

\begin{longtable}[]{@{}
  >{\raggedright\arraybackslash}p{(\columnwidth - 0\tabcolsep) * \real{1.0000}}@{}}
\toprule()
\begin{minipage}[b]{\linewidth}\raggedright
✦ \textbf{Definition 0.0.17}\\
Um inteiro \emph{n} é \textbf{par} se é divisível por 2; de outra forma,
\emph{n} é \textbf{impar}.\strut
\end{minipage} \\
\midrule()
\endhead
\bottomrule()
\end{longtable}

9

\begin{longtable}[]{@{}
  >{\raggedright\arraybackslash}p{(\columnwidth - 2\tabcolsep) * \real{0.5000}}
  >{\raggedright\arraybackslash}p{(\columnwidth - 2\tabcolsep) * \real{0.5000}}@{}}
\toprule()
\begin{minipage}[b]{\linewidth}\raggedright
\begin{quote}
10
\end{quote}
\end{minipage} & \begin{minipage}[b]{\linewidth}\raggedright
\emph{Capítulo 0. Começando}
\end{minipage} \\
\midrule()
\endhead
\multicolumn{2}{@{}>{\raggedright\arraybackslash}p{(\columnwidth - 2\tabcolsep) * \real{1.0000} + 2\tabcolsep}@{}}{%
\begin{minipage}[t]{\linewidth}\raggedright
\begin{quote}
Não é apenas interessante saber quando um inteiro \emph{divide} outro;
entretanto, provar que um inteiro \emph{não} divide outro é muito mais
difícil. Na verdade, para provar que um inteiro \emph{b} não divide um
inteiro \emph{a}, devemos provar que \emph{a ̸}= \emph{qb} para
\emph{qualquer} inteiro \emph{q}. Veremos métodos para fazer isso em
\textbf{??}; esses métodos usam o seguinte resultado extremamente
importante, que será a base de toda a \textbf{??}.
\end{quote}

\begin{longtable}[]{@{}
  >{\raggedright\arraybackslash}p{(\columnwidth - 0\tabcolsep) * \real{1.0000}}@{}}
\toprule()
\begin{minipage}[b]{\linewidth}\raggedright
✣ \textbf{Theorem 0.0.18} (Teorema da divisão, para ser repetido in
\textbf{??}) Seja \emph{a,b ∈} Z com \emph{b ̸}= 0. Existe exatamente
uma maneira de escrever

\emph{a} = \emph{qb}+\emph{r}
\end{minipage} \\
\midrule()
\endhead
\bottomrule()
\end{longtable}

\begin{quote}
de tal modo que \emph{q} e \emph{r} são inteiros, e 0 ⩽ \emph{r
\textless{} b} (se \emph{b \textgreater{}} 0) ou 0 ⩽ \emph{r \textless{}
−b} (se \emph{b \textless{}} 0).
\end{quote}
\end{minipage}} \\
\bottomrule()
\end{longtable}

\begin{quote}
O número \emph{q} em Theorem 0.0.18 é chamado o \textbf{quociente} de
\emph{a} quando dividido por \emph{b}, e o número \emph{r} é chamado o
\textbf{restante}.
\end{quote}

\begin{longtable}[]{@{}
  >{\raggedright\arraybackslash}p{(\columnwidth - 2\tabcolsep) * \real{0.5000}}
  >{\raggedright\arraybackslash}p{(\columnwidth - 2\tabcolsep) * \real{0.5000}}@{}}
\toprule()
\begin{minipage}[b]{\linewidth}\raggedright
✐ \textbf{Example 0.0.19}\\
O número 12 deixa um restante de 2 quando dividido por 5, desde que 12 =
2\emph{·}5+2.\strut
\end{minipage} & \begin{minipage}[b]{\linewidth}\raggedright
◁
\end{minipage} \\
\midrule()
\endhead
\bottomrule()
\end{longtable}

\begin{quote}
Aqui está um exemplo um pouco mais complicado.

✣ \textbf{Proposition 0.0.20}\\
Suponha um número inteiro \emph{a} deixa um resto de \emph{r} quando
dividido por um número inteiro \emph{b}, e que \emph{r \textgreater{}}
0. Então \emph{−a} deixa um resto de \emph{b−r} quando dividido por
\emph{b}.
\end{quote}

\emph{\textbf{Proof}}\\
Suponha que \emph{a} deixa um resto de \emph{r} quando dividido por
\emph{b}. Então\\
\emph{a} = \emph{qb}+\emph{r}\\
Para algum número inteiro \emph{q}. Um pouco de rendimento de álgebra\\
\emph{−a} = \emph{−qb−r} = \emph{−qb−r} +(\emph{b−b}) =
\emph{−}(\emph{q}+1)\emph{b}+(\emph{b−r})\\
Since 0 \emph{\textless{} r \textless{} b}, Nós temos 0
\emph{\textless{} b − r \textless{} b}. Por isso \emph{−}(\emph{q} + 1)
é o quociente de \emph{−a} quando dividido por \emph{b}, e \emph{b−r} é
o restante. □✎ \textbf{Exercise 0.0.21}\\
Prove que se um inteiro \emph{a} deixa um resto de \emph{r} quando
dividido por um inteiro \emph{b}, então \emph{a} deixa um resto de
\emph{r} quando dividido por \emph{−b}. ◁

\begin{quote}
Terminaremos esta parte sobre divisão de inteiros conectando-a com o
material em bases numéricas --- podemos usar o teorema da divisão
(Theorem 0.0.18) para encon-trar a expansão de base \emph{b} de um
determinado número natural. É baseado na seguinte observação: o número
natural \emph{n} cuja expansão de base \emph{b} é \emph{drdr−}1
\emph{···d}1\emph{d}0 é igual a \emph{d}0 +\emph{b}(\emph{d}1
+\emph{b}(\emph{d}2 +\emph{···}+\emph{b}(\emph{dr−}1
+\emph{bdr})\emph{···}))\\
10

\emph{Capítulo 0. Começando} 11

Além disso, 0 ⩽ \emph{di \textless{} b} para todo \emph{i}. Em
particular, \emph{n} deixa um resto de \emph{d}0 quando dividido por
\emph{b}. Por isso

\emph{n−d}0 \emph{b} = \emph{d}1
+\emph{d}2\emph{b}+\emph{···}+\emph{drbr−}1

A base -b \emph{b} expansão de\emph{\uline{n}\sout{−}\uline{d}}\uline{0
\emph{b }} é portanto

\emph{drdr−}1 \emph{···d}1\\
Em outras palavras, o resto de \emph{n} quando dividido por \emph{b} é o
último dígito de base \emph{b} de \emph{n}, e então subtrair esse número
de \emph{n} e dividir o resultado por \emph{b} trunca o final dígito.
Repetir esse processo nos dá \emph{d}1, e depois \emph{d}2, e assim por
diante, até chegarmos a 0. Isso sugere o seguinte algoritmo para
calcular a expansão de base \emph{b} de um número \emph{n}:

• \textbf{Passo 1.} Seja \emph{d}0 seja o resto quando \emph{n} é
dividido por \emph{b}, e Seja \emph{n}0
=\emph{\uline{n}\sout{−}\uline{d}}\uline{0 \emph{b }} seja o quociente.
Conserte \emph{i} = 0.

• \textbf{Step 2.} Suponha \emph{ni} e \emph{di} foram definidos. Se
\emph{ni} = 0, em seguida, prossiga para o Passo 3. Por outro lado
define \emph{di}+1 o seja o resto quando \emph{ni} é dividido por
\emph{b} e defina \emph{ni}+1
=\emph{n\uline{i}\sout{−}d\uline{i}}\uline{+1 \emph{b }} . Aumente
\emph{i} e repita a Etapa 2.

• \textbf{Etapa 3.} A expansão base-\emph{b} de \emph{n} é
\end{quote}

\begin{longtable}[]{@{}
  >{\raggedright\arraybackslash}p{(\columnwidth - 2\tabcolsep) * \real{0.5000}}
  >{\raggedright\arraybackslash}p{(\columnwidth - 2\tabcolsep) * \real{0.5000}}@{}}
\toprule()
\begin{minipage}[b]{\linewidth}\raggedright
✐ \textbf{Example 0.0.22}
\end{minipage} & \begin{minipage}[b]{\linewidth}\raggedright
\emph{didi−}1 \emph{···d}0
\end{minipage} \\
\midrule()
\endhead
\bottomrule()
\end{longtable}

\begin{quote}
Calculamos a expansão de base 17 de 15213, usando as letras A--G para
representar os números 10 até 16.

• 15213 = 894\emph{·}17+15, então \emph{d}0 = 15 = F e \emph{n}0 = 894.•
894 = 52\emph{·}17+10, então \emph{d}1 = 10 = A e \emph{n}1 = 52.• 52 =
3\emph{·}17+1, então \emph{d}2 = 1 and \emph{n}2 = 3.

• 3 = 0\emph{·}17+3, então \emph{d}3 = 3 e \emph{n}3 = 0.

• A base -17 expansão de 15213 é portanto 31AF.

Uma rápida verificação

31AF(17) = 3\emph{·}173+1\emph{·}172+10\emph{·}17+15 = 15213

como desejado. ◁
\end{quote}

✎ \textbf{Exercise 0.0.23}\\
Encontre a expansão de base 17 de 408735787 e a expansão de base 36 de
1442151747. ◁

11

\includegraphics[width=4.875in,height=1.95833in]{vertopal_90446958eb2044f0ba6a30adb46b64c5/media/image5.png}

\begin{quote}
12 \emph{Capítulo 0. Começando}

\textbf{Números Racionais (}Q\textbf{)}

Cansado de comer maçã e banana, compro uma pizza dividida em oito
fatias. Um amigo e eu decidimos dividir a pizza. Não tenho muito
apetite, então como três fatias e meu amigo come cinco. Infelizmente,
não podemos representar a proporção da pizza que cada um de nós comeu
usando números naturais ou inteiros. Porém, não estamos muito longe:
podemos contar o número de partes iguais em que a pizza foi dividida e,
dessas partes, podemos contar quantas tivemos. Na reta numérica, isso
poderia ser representado dividindo o segmento de reta unitária de 0 a 1
em oito partes iguais e procedendo a partir daí. Este tipo de
procedimento dá origem aos \emph{números racionais}.
\end{quote}

✦ \textbf{Definition 0.0.24}\\
Os \textbf{números racionais} são representados pelos pontos na reta
numérica que podem ser obtidos dividindo qualquer um dos segmentos da
reta unitária entre inteiros em um número igual de partes.

\begin{longtable}[]{@{}
  >{\raggedright\arraybackslash}p{(\columnwidth - 20\tabcolsep) * \real{0.0909}}
  >{\raggedright\arraybackslash}p{(\columnwidth - 20\tabcolsep) * \real{0.0909}}
  >{\raggedright\arraybackslash}p{(\columnwidth - 20\tabcolsep) * \real{0.0909}}
  >{\raggedright\arraybackslash}p{(\columnwidth - 20\tabcolsep) * \real{0.0909}}
  >{\raggedright\arraybackslash}p{(\columnwidth - 20\tabcolsep) * \real{0.0909}}
  >{\raggedright\arraybackslash}p{(\columnwidth - 20\tabcolsep) * \real{0.0909}}
  >{\raggedright\arraybackslash}p{(\columnwidth - 20\tabcolsep) * \real{0.0909}}
  >{\raggedright\arraybackslash}p{(\columnwidth - 20\tabcolsep) * \real{0.0909}}
  >{\raggedright\arraybackslash}p{(\columnwidth - 20\tabcolsep) * \real{0.0909}}
  >{\raggedright\arraybackslash}p{(\columnwidth - 20\tabcolsep) * \real{0.0909}}
  >{\raggedright\arraybackslash}p{(\columnwidth - 20\tabcolsep) * \real{0.0909}}@{}}
\toprule()
\begin{minipage}[b]{\linewidth}\raggedright
\emph{−}5
\end{minipage} & \begin{minipage}[b]{\linewidth}\raggedright
\emph{−}4
\end{minipage} & \begin{minipage}[b]{\linewidth}\raggedright
\emph{−}3
\end{minipage} & \begin{minipage}[b]{\linewidth}\raggedright
\emph{−}2
\end{minipage} & \begin{minipage}[b]{\linewidth}\raggedright
\emph{−}1
\end{minipage} & \begin{minipage}[b]{\linewidth}\raggedright
0
\end{minipage} & \begin{minipage}[b]{\linewidth}\raggedright
1
\end{minipage} & \begin{minipage}[b]{\linewidth}\raggedright
2
\end{minipage} & \begin{minipage}[b]{\linewidth}\raggedright
3
\end{minipage} & \begin{minipage}[b]{\linewidth}\raggedright
4
\end{minipage} & \begin{minipage}[b]{\linewidth}\raggedright
5
\end{minipage} \\
\midrule()
\endhead
\bottomrule()
\end{longtable}

\begin{quote}
Os números racionais são aqueles da forma\emph{a b}, onde \emph{a,b ∈} Z
e \emph{b ̸}= 0. Escrevemos Q (LATEX code: \textbackslash mathbb\{Q\})
para o conjunto de todos os números racionais; portanto, a notação
`\emph{q ∈} Q' significa que \emph{q} é um número racional.

Os números racionais são um exemplo muito importante de um tipo de
estrutura al-gébrica conhecida como \emph{campo} --- eles são
particularmente centrais para a teoria al-gébrica dos números e para a
geometria algébrica.

\textbf{Números reais (}R\textbf{)}

Quantidade e mudança podem ser medidas abstratamente usando
\emph{números reais}.
\end{quote}

\begin{longtable}[]{@{}
  >{\raggedright\arraybackslash}p{(\columnwidth - 0\tabcolsep) * \real{1.0000}}@{}}
\toprule()
\begin{minipage}[b]{\linewidth}\raggedright
✦ \textbf{Definition 0.0.25}\\
Os \textbf{números reais} são os pontos na reta numérica. Escrevemos R
(LATEX code: \textbackslash mathbb\{R\}) para o conjunto de todos os
números reais; portanto, a notação `\emph{a ∈} R'significa que \emph{a}
é um número real.\strut
\end{minipage} \\
\midrule()
\endhead
\bottomrule()
\end{longtable}

\begin{quote}
Os números reais são centrais para a análise real, um ramo da matemática
introduzido em \textbf{??}. Eles transformam os racionais em um
\emph{continuum} ao `preencher as lacunas'---especificamente, eles têm a
propriedade de \emph{completude}, o que significa que se uma
\end{quote}

12

\emph{Capítulo 0. Começando} 13

quantidade pode ser aproximada com precisão arbitrária por números
reais, então essa

quantidade é em si um número real.

Podemos definir as operações aritméticas básicas (adição, subtração,
multiplicação e

divisão) sobre os números reais, e uma noção de ordenação dos números
reais, em

termos da reta numérica infinita.

• \textbf{Ordenando.} Um número real \emph{a} é menor que um número real
\emph{b}, escrito \emph{a \textless{} b}, se

\begin{quote}
\emph{a} estiver à esquerda de \emph{b} na reta numérica. As convenções
usuais para os símbolos

⩽ (LATEX code: \textbackslash le), \emph{\textgreater{}} e ⩾ (LATEX
code: \textbackslash ge) se aplicam, por exemplo `\emph{a} ⩽ \emph{b}'

significa que \emph{a \textless{} b} ou \emph{a} = \emph{b}.
\end{quote}

• \textbf{Adição.} Suponha que queremos adicionar um número real
\emph{a} a um número real \emph{b}. Para

\begin{quote}
fazer isso, \emph{traduzimos a} em \emph{b} unidades para a direita ---
se \emph{b \textless{}} 0 então isso equivale

a traduzir \emph{a} em um número equivalente de unidades para a
esquerda. Concretamente,

faça duas cópias da reta numérica, uma acima da outra, com a mesma
escolha de

comprimento unitário; mova o 0 da reta numérica inferior abaixo do ponto
\emph{a} da reta

numérica superior. Então \emph{a}+\emph{b} é o ponto na reta numérica
superior situado acima do

ponto \emph{b} da reta numérica inferior. (\emph{−}3)+5 = 2:

\emph{−}8 \emph{−}7 \emph{−}6 \emph{−}5 \emph{−}4 \emph{−}3 \emph{−}2
\emph{−}1 0 1 2 3 4 5

\includegraphics[width=4.31944in,height=0.38889in]{vertopal_90446958eb2044f0ba6a30adb46b64c5/media/image6.png}

\emph{−}5 \emph{−}4 \emph{−}3 \emph{−}2 \emph{−}1 0 1 2 3 4 5 6 7 8
\end{quote}

• \textbf{Multiplicação.} Essa é divertida. Suponha-se que queiramos
multiplicar um número

\begin{quote}
real a por um número real b. Para fazer isso, escala-se a reta numérica
e talvez

ambas as partes se refletirão. Concretamente, façamos duas cópias da
reta numérica,

uma acima da outra; alinhe os pontos 0 em ambas as retas numéricas e
estique a

reta numérica inferior uniformemente até que o ponto 1 na reta numérica
inferior

esteja abaixo do ponto a na reta numérica superior --- observe que se a
\textless{} 0 então a

reta numérica deve ser refletida para que isso aconteça. Então a·b é o
ponto na reta

numérica superior situado acima de b na reta numérica inferior. 5·4 =
20. 5\emph{·}4 = 20.
\end{quote}

\begin{longtable}[]{@{}
  >{\raggedright\arraybackslash}p{(\columnwidth - 24\tabcolsep) * \real{0.0769}}
  >{\raggedright\arraybackslash}p{(\columnwidth - 24\tabcolsep) * \real{0.0769}}
  >{\raggedright\arraybackslash}p{(\columnwidth - 24\tabcolsep) * \real{0.0769}}
  >{\raggedright\arraybackslash}p{(\columnwidth - 24\tabcolsep) * \real{0.0769}}
  >{\raggedright\arraybackslash}p{(\columnwidth - 24\tabcolsep) * \real{0.0769}}
  >{\raggedright\arraybackslash}p{(\columnwidth - 24\tabcolsep) * \real{0.0769}}
  >{\raggedright\arraybackslash}p{(\columnwidth - 24\tabcolsep) * \real{0.0769}}
  >{\raggedright\arraybackslash}p{(\columnwidth - 24\tabcolsep) * \real{0.0769}}
  >{\raggedright\arraybackslash}p{(\columnwidth - 24\tabcolsep) * \real{0.0769}}
  >{\raggedright\arraybackslash}p{(\columnwidth - 24\tabcolsep) * \real{0.0769}}
  >{\raggedright\arraybackslash}p{(\columnwidth - 24\tabcolsep) * \real{0.0769}}
  >{\raggedright\arraybackslash}p{(\columnwidth - 24\tabcolsep) * \real{0.0769}}
  >{\raggedright\arraybackslash}p{(\columnwidth - 24\tabcolsep) * \real{0.0769}}@{}}
\toprule()
\begin{minipage}[b]{\linewidth}\raggedright
-2
\end{minipage} & \begin{minipage}[b]{\linewidth}\raggedright
-1
\end{minipage} & \begin{minipage}[b]{\linewidth}\raggedright
0
\end{minipage} & \begin{minipage}[b]{\linewidth}\raggedright
1
\end{minipage} & \begin{minipage}[b]{\linewidth}\raggedright
2
\end{minipage} & \begin{minipage}[b]{\linewidth}\raggedright
3
\end{minipage} & \begin{minipage}[b]{\linewidth}\raggedright
4
\end{minipage} & \begin{minipage}[b]{\linewidth}\raggedright
5
\end{minipage} & \begin{minipage}[b]{\linewidth}\raggedright
6
\end{minipage} & \begin{minipage}[b]{\linewidth}\raggedright
7
\end{minipage} & \begin{minipage}[b]{\linewidth}\raggedright
8
\end{minipage} & \begin{minipage}[b]{\linewidth}\raggedright
9
\end{minipage} & \begin{minipage}[b]{\linewidth}\raggedright
\begin{quote}
10 11 12 13 14 15 16 17 18 19 20 21 22 23 24
\end{quote}
\end{minipage} \\
\midrule()
\endhead
\bottomrule()
\end{longtable}

\begin{quote}
\includegraphics[width=4.31944in,height=0.38889in]{vertopal_90446958eb2044f0ba6a30adb46b64c5/media/image7.png}
\end{quote}

\begin{longtable}[]{@{}
  >{\raggedright\arraybackslash}p{(\columnwidth - 8\tabcolsep) * \real{0.2000}}
  >{\raggedright\arraybackslash}p{(\columnwidth - 8\tabcolsep) * \real{0.2000}}
  >{\raggedright\arraybackslash}p{(\columnwidth - 8\tabcolsep) * \real{0.2000}}
  >{\raggedright\arraybackslash}p{(\columnwidth - 8\tabcolsep) * \real{0.2000}}
  >{\raggedright\arraybackslash}p{(\columnwidth - 8\tabcolsep) * \real{0.2000}}@{}}
\toprule()
\begin{minipage}[b]{\linewidth}\raggedright
0
\end{minipage} & \begin{minipage}[b]{\linewidth}\raggedright
1
\end{minipage} & \begin{minipage}[b]{\linewidth}\raggedright
2
\end{minipage} & \begin{minipage}[b]{\linewidth}\raggedright
3
\end{minipage} & \begin{minipage}[b]{\linewidth}\raggedright
4
\end{minipage} \\
\midrule()
\endhead
\bottomrule()
\end{longtable}

\begin{quote}
e aqui está uma ilustração do fato de que (\emph{−}5)\emph{·}4 =
\emph{−}20:

-22 -21 -20 -19 -18 -17 -16 -15 -14 -13 -12 -11 -10 -9 -8 -7 -6 -5 -4 -3
-2 -1 0 1 2 3 4

\includegraphics[width=4.31944in,height=0.38889in]{vertopal_90446958eb2044f0ba6a30adb46b64c5/media/image8.png}

4 3 2 1 0
\end{quote}

13

\begin{quote}
14 \emph{Capítulo 0. Começando}
\end{quote}

✎ \textbf{Exercise 0.0.26}\\
Interprete as operações de subtração e divisão como transformações
geométricas da reta numérica real. ◁

\begin{quote}
Tomaremos como certas as propriedades aritméticas dos números reais
neste capítulo, esperando até \textbf{??} para afundar nossos dentes nos
detalhes. Por exemplo, tomaremos como certas as propriedades básicas dos
números racionais, por exemplo
\end{quote}

\begin{longtable}[]{@{}
  >{\raggedright\arraybackslash}p{(\columnwidth - 4\tabcolsep) * \real{0.3333}}
  >{\raggedright\arraybackslash}p{(\columnwidth - 4\tabcolsep) * \real{0.3333}}
  >{\raggedright\arraybackslash}p{(\columnwidth - 4\tabcolsep) * \real{0.3333}}@{}}
\toprule()
\begin{minipage}[b]{\linewidth}\raggedright
\emph{b}+ \emph{c d}= \emph{ad} +\emph{bc}
\end{minipage} & \begin{minipage}[b]{\linewidth}\raggedright
and
\end{minipage} & \begin{minipage}[b]{\linewidth}\raggedright
\begin{quote}
\emph{b· c d}= \emph{ac}
\end{quote}
\end{minipage} \\
\midrule()
\endhead
\bottomrule()
\end{longtable}

\begin{quote}
\textbf{Números Racionais e Irracionais}

Antes de podermos falar sobre números irracionais, devemos dizer o que
são.
\end{quote}

\begin{longtable}[]{@{}
  >{\raggedright\arraybackslash}p{(\columnwidth - 0\tabcolsep) * \real{1.0000}}@{}}
\toprule()
\begin{minipage}[b]{\linewidth}\raggedright
✦ \textbf{Definition 0.0.27}\\
An \textbf{Número irracional}é um número que não é racional.\strut
\end{minipage} \\
\midrule()
\endhead
\bottomrule()
\end{longtable}

\begin{quote}
Ao contrário de N\emph{,}Z\emph{,}Q\emph{,}R\emph{,}C, não existe uma
única letra padrão que expresse os números irracionais. No entanto, até
ao final de \textbf{??}, seremos capazes de escrever o conjunto de
números irracionais como `R\emph{\textbackslash{}}Q'.

Provar que um número real é \emph{irracional} não é particularmente
fácil, em geral. Colo-caremos o pé na porta permitindo-nos assumir o
seguinte resultado, que é reafirmado e provado em \textbf{??}.
\end{quote}

\begin{longtable}[]{@{}
  >{\raggedright\arraybackslash}p{(\columnwidth - 4\tabcolsep) * \real{0.3333}}
  >{\raggedright\arraybackslash}p{(\columnwidth - 4\tabcolsep) * \real{0.3333}}
  >{\raggedright\arraybackslash}p{(\columnwidth - 4\tabcolsep) * \real{0.3333}}@{}}
\toprule()
\multicolumn{2}{@{}>{\raggedright\arraybackslash}p{(\columnwidth - 4\tabcolsep) * \real{0.6667} + 2\tabcolsep}}{%
\begin{minipage}[b]{\linewidth}\raggedright
\begin{quote}
✣ \textbf{Proposition 0.0.28}\\
O número real\emph{√}2 é irracional.
\end{quote}\strut
\end{minipage}} & \begin{minipage}[b]{\linewidth}\raggedright
\begin{longtable}[]{@{}
  >{\raggedright\arraybackslash}p{(\columnwidth - 0\tabcolsep) * \real{1.0000}}@{}}
\toprule()
\begin{minipage}[b]{\linewidth}\raggedright
\end{minipage} \\
\midrule()
\endhead
\bottomrule()
\end{longtable}
\end{minipage} \\
\midrule()
\endhead
\begin{minipage}[t]{\linewidth}\raggedright
\begin{quote}
Podemos usar o fato de que
\end{quote}
\end{minipage} &
\multicolumn{2}{>{\raggedright\arraybackslash}p{(\columnwidth - 4\tabcolsep) * \real{0.6667} + 2\tabcolsep}@{}}{%
\emph{√}2 é irracional para provar alguns fatos sobre a relação} \\
\multicolumn{3}{@{}>{\raggedright\arraybackslash}p{(\columnwidth - 4\tabcolsep) * \real{1.0000} + 4\tabcolsep}@{}}{%
\begin{minipage}[t]{\linewidth}\raggedright
\begin{quote}
entre números racionais e números irracionais.
\end{quote}
\end{minipage}} \\
\bottomrule()
\end{longtable}

\begin{quote}
✣ \textbf{Proposition 0.0.29}\\
Sejam \emph{a} e \emph{b} números irracionais. É possível que \emph{ab}
seja racional.
\end{quote}

\begin{longtable}[]{@{}
  >{\raggedright\arraybackslash}p{(\columnwidth - 4\tabcolsep) * \real{0.3333}}
  >{\raggedright\arraybackslash}p{(\columnwidth - 4\tabcolsep) * \real{0.3333}}
  >{\raggedright\arraybackslash}p{(\columnwidth - 4\tabcolsep) * \real{0.3333}}@{}}
\toprule()
\begin{minipage}[b]{\linewidth}\raggedright
\begin{quote}
\emph{\textbf{Proof}}
\end{quote}
\end{minipage} &
\multirow{2}{*}{\begin{minipage}[b]{\linewidth}\raggedright
\begin{quote}
\emph{√}2. Então \emph{a} e \emph{b} são irracionais,e \emph{ab} = 2
=\uline{2 1}, o que é racional.
\end{quote}
\end{minipage}} & \begin{minipage}[b]{\linewidth}\raggedright
\end{minipage} \\
\begin{minipage}[b]{\linewidth}\raggedright
\begin{quote}
Seja \emph{a} = \emph{b} =
\end{quote}
\end{minipage} & & \begin{minipage}[b]{\linewidth}\raggedright
□
\end{minipage} \\
\midrule()
\endhead
\bottomrule()
\end{longtable}

✎ \textbf{Exercise 0.0.30}\\
Seja \emph{r} um número racional e \emph{a} um número irracional. Prove
que é possível que \emph{ra} seja racional, e é possível que \emph{ra}
seja irracional. ◁

14

\begin{quote}
\emph{Capítulo 0. Começando} 15

\textbf{Números Complexos (}C\textbf{)}

Vimos que a multiplicação por números reais corresponde ao escalar e
reflexão da reta numérica -- escalonamento sozinho quando o
multiplicando é positivo, e escalando com reflexão quando é negativo.
Poderíamos alternativamente interpretar esta reflexão como uma
\emph{rotação} de meia volta, uma vez que o efeito na reta numérica é o
mesmo. Você pode então se perguntar o que acontece se girarmos em
ângulos arbitrários, em vez de apenas meias voltas.

O que obtemos é um \emph{plano} de números, não apenas uma linha ---
veja Figure 1. Além disso, acontece que as regras que esperamos que as
operações aritméticas satisfaçam ainda são válidas -- a adição
corresponde à translação e a multiplicação corresponde à escala e à
rotação. Este conjunto de números resultante é o dos \emph{números
complexos}.
\end{quote}

\begin{longtable}[]{@{}
  >{\raggedright\arraybackslash}p{(\columnwidth - 4\tabcolsep) * \real{0.3333}}
  >{\raggedright\arraybackslash}p{(\columnwidth - 4\tabcolsep) * \real{0.3333}}
  >{\raggedright\arraybackslash}p{(\columnwidth - 4\tabcolsep) * \real{0.3333}}@{}}
\toprule()
\multicolumn{3}{@{}>{\raggedright\arraybackslash}p{(\columnwidth - 4\tabcolsep) * \real{1.0000} + 4\tabcolsep}@{}}{%
\begin{minipage}[b]{\linewidth}\raggedright
\begin{longtable}[]{@{}
  >{\raggedright\arraybackslash}p{(\columnwidth - 0\tabcolsep) * \real{1.0000}}@{}}
\toprule()
\begin{minipage}[b]{\linewidth}\raggedright
✦ \textbf{Definition 0.0.31}\\
Os \textbf{números complexos} são aqueles obtidos pelos números reais
não negativos após rotação de qualquer ângulo em torno do ponto 0.
Escrevemos C (LATEX code: \textbackslash mathbb\{C\}) para o conjunto de
todos os números complexos; portanto, a notação\strut
\end{minipage} \\
\midrule()
\endhead
\bottomrule()
\end{longtable}

\begin{quote}
`\emph{z ∈} C' significa que \emph{z} é um número complexo.

Existe um número complexo particularmente importante, \emph{i}, que é o
ponto no plano complexo exatamente uma unidade acima de 0 --- isso é
ilustrado em Figure 1. A multiplicação por \emph{i} tem o efeito de
girar o plano um quarto de volta no sentido anti-horário. Em particular,
temos \emph{i}2= \emph{i·i} = \emph{−}1; os números complexos têm a
surpreendente propriedade de que existem raízes quadradas de
\emph{todos} os números complexos (incluindo todos os números reais).

Na verdade, todo número complexo pode ser escrito na forma \emph{a} +
\emph{bi}, onde \emph{a,b ∈} R; este número corresponde ao ponto no
plano complexo obtido movendo \emph{a} unidades para a direita e
\emph{b} unidades para cima, invertendo as direções como de costume se
\emph{a} ou \emph{b} for negativo. A aritmética nos números complexos
funciona da mesma forma que nos números reais; em particular, usando o
fato de que \emph{i}2= \emph{−}1, obtemos
\end{quote}\strut
\end{minipage}} \\
\midrule()
\endhead
(\emph{a}+\emph{bi})+(\emph{c}+\emph{di}) =
(\emph{a}+\emph{c})+(\emph{b}+\emph{d})\emph{i} & e &
\begin{minipage}[t]{\linewidth}\raggedright
\begin{quote}
(\emph{a}+\emph{bi})\emph{·}(\emph{c}+\emph{di}) =
(\emph{ac−bd})+(\emph{anncio}+\emph{bc})\emph{i}
\end{quote}
\end{minipage} \\
\bottomrule()
\end{longtable}

\begin{quote}
Discutiremos números complexos mais detalhadamente na parte deste
capítulo sobre polinômios abaixo.
\end{quote}

15

\includegraphics[width=4.73611in,height=4.38889in]{vertopal_90446958eb2044f0ba6a30adb46b64c5/media/image11.png}

\begin{longtable}[]{@{}
  >{\raggedright\arraybackslash}p{(\columnwidth - 6\tabcolsep) * \real{0.2500}}
  >{\raggedright\arraybackslash}p{(\columnwidth - 6\tabcolsep) * \real{0.2500}}
  >{\raggedright\arraybackslash}p{(\columnwidth - 6\tabcolsep) * \real{0.2500}}
  >{\raggedright\arraybackslash}p{(\columnwidth - 6\tabcolsep) * \real{0.2500}}@{}}
\toprule()
\begin{minipage}[b]{\linewidth}\raggedright
16
\end{minipage} & \begin{minipage}[b]{\linewidth}\raggedright
\includegraphics[width=\textwidth,height=0.11111in]{vertopal_90446958eb2044f0ba6a30adb46b64c5/media/image9.png}
\end{minipage} & \begin{minipage}[b]{\linewidth}\raggedright
\begin{quote}
5\emph{i}
\end{quote}
\end{minipage} & \begin{minipage}[b]{\linewidth}\raggedright
\begin{quote}
\emph{Capítulo 0. Começando}
\end{quote}
\end{minipage} \\
\midrule()
\endhead
\bottomrule()
\end{longtable}

\begin{quote}
4\emph{i}\\
3\emph{i}\\
2\emph{i}\\
\emph{i}
\end{quote}

\begin{longtable}[]{@{}
  >{\raggedright\arraybackslash}p{(\columnwidth - 20\tabcolsep) * \real{0.0909}}
  >{\raggedright\arraybackslash}p{(\columnwidth - 20\tabcolsep) * \real{0.0909}}
  >{\raggedright\arraybackslash}p{(\columnwidth - 20\tabcolsep) * \real{0.0909}}
  >{\raggedright\arraybackslash}p{(\columnwidth - 20\tabcolsep) * \real{0.0909}}
  >{\raggedright\arraybackslash}p{(\columnwidth - 20\tabcolsep) * \real{0.0909}}
  >{\raggedright\arraybackslash}p{(\columnwidth - 20\tabcolsep) * \real{0.0909}}
  >{\raggedright\arraybackslash}p{(\columnwidth - 20\tabcolsep) * \real{0.0909}}
  >{\raggedright\arraybackslash}p{(\columnwidth - 20\tabcolsep) * \real{0.0909}}
  >{\raggedright\arraybackslash}p{(\columnwidth - 20\tabcolsep) * \real{0.0909}}
  >{\raggedright\arraybackslash}p{(\columnwidth - 20\tabcolsep) * \real{0.0909}}
  >{\raggedright\arraybackslash}p{(\columnwidth - 20\tabcolsep) * \real{0.0909}}@{}}
\toprule()
\begin{minipage}[b]{\linewidth}\raggedright
-5
\end{minipage} & \begin{minipage}[b]{\linewidth}\raggedright
-4
\end{minipage} & \begin{minipage}[b]{\linewidth}\raggedright
-3
\end{minipage} & \begin{minipage}[b]{\linewidth}\raggedright
-2
\end{minipage} & \begin{minipage}[b]{\linewidth}\raggedright
-1
\end{minipage} & \begin{minipage}[b]{\linewidth}\raggedright
0
\end{minipage} & \begin{minipage}[b]{\linewidth}\raggedright
1
\end{minipage} & \begin{minipage}[b]{\linewidth}\raggedright
2
\end{minipage} & \begin{minipage}[b]{\linewidth}\raggedright
3
\end{minipage} & \begin{minipage}[b]{\linewidth}\raggedright
4
\end{minipage} & \begin{minipage}[b]{\linewidth}\raggedright
5
\end{minipage} \\
\midrule()
\endhead
\bottomrule()
\end{longtable}

-\emph{i}

-2\emph{i}

-3\emph{i}

-4\emph{i}

\includegraphics[width=\textwidth,height=0.18056in]{vertopal_90446958eb2044f0ba6a30adb46b64c5/media/image10.png}-5\emph{i}

Figure 1: Ilustração do plano complexo, com alguns pontos rotulados.

16

\begin{quote}
\emph{Capítulo 0. Começando} 17

\textbf{Polinômios}

Os números naturais, inteiros, números racionais, números reais e
números complexos são todos exemplos de \emph{semirings}, o que
significa que eles vêm equipados com noções de adição e multiplicação
bem comportadas.
\end{quote}

\begin{longtable}[]{@{}
  >{\raggedright\arraybackslash}p{(\columnwidth - 0\tabcolsep) * \real{1.0000}}@{}}
\toprule()
\begin{minipage}[b]{\linewidth}\raggedright
\begin{longtable}[]{@{}
  >{\raggedright\arraybackslash}p{(\columnwidth - 0\tabcolsep) * \real{1.0000}}@{}}
\toprule()
\begin{minipage}[b]{\linewidth}\raggedright
✦ \textbf{Definition 0.0.32}\\
Seja S = N, Z, Q, R ou C. Um (\textbf{univariado}) \textbf{polinômio
sobre} S no \textbf{indeterminado} \emph{x} é uma expressão da forma\\
\emph{a}0 +\emph{a}1\emph{x}+\emph{···}+\emph{anxn}\\
onde \emph{n ∈} N e cada \emph{ak ∈} S. Os números \emph{ak} são
chamados de \textbf{coeficientes} do polinômio. Se nem todos os
coeficientes forem zero, o maior valor de \emph{k} para o qual \emph{ak
̸}= 0 é\strut
\end{minipage} \\
\midrule()
\endhead
\bottomrule()
\end{longtable}

\begin{quote}
chamado de \textbf{grau} do polinômio. Por convenção, o grau do
polinômio 0 é \emph{−}∞.
\end{quote}\strut
\end{minipage} \\
\midrule()
\endhead
\bottomrule()
\end{longtable}

\begin{quote}
Polinômios de grau 1, 2, 3, 4 e 5 são respectivamente chamados
\emph{lineares}, \emph{quadrático}, \emph{cúbico}, \emph{quártico} e
\emph{quíntico} polinomiais.
\end{quote}

✐ \textbf{Example 0.0.33}\\
As seguintes expressões são todas polinômios:

\begin{quote}
3 2\emph{x−}1 (3+\emph{i})\emph{x}2\emph{−x}

Seus diplomas são 0, 1 e 2, respectivamente. Os dois primeiros são
polinômios sobre Z, e o terceiro é um polinômio sobre C. ◁
\end{quote}

✎ \textbf{Exercise 0.0.34}\\
Escreva um polinômio de grau 4 sobre R que não seja um polinômio sobre
Q. ◁

\begin{quote}
✦ \textbf{Notation 0.0.35}\\
Em vez de escrever os coeficientes de um polinômio todas as vezes,
podemos escrever algo como \emph{p}(\emph{x}) ou \emph{q}(\emph{x}). O
`(\emph{x})' indica que \emph{x} é o indeterminado do polinômio. Se
\emph{α} for um número{[}a{]}e \emph{p}(\emph{x}) é um polinômio em
\emph{x} indeterminado, escrevemos \emph{p}(\emph{α}) para o resultado
de \textbf{substituir} \emph{α} por \emph{x} na expressão
\emph{p}(\emph{x}).
\end{quote}

\begin{longtable}[]{@{}
  >{\raggedright\arraybackslash}p{(\columnwidth - 0\tabcolsep) * \real{1.0000}}@{}}
\toprule()
\begin{minipage}[b]{\linewidth}\raggedright
\begin{quote}
Note que, se \emph{A} é qualquer um dos conjuntos N, Z, Q, R ou C, e
\emph{p}(\emph{x}) é um polinômio sobre \emph{A}, então
\emph{p}(\emph{α}) \emph{∈ A} para toda \emph{α ∈ A}.
\end{quote}

✐ \textbf{Example 0.0.36}

\begin{quote}
Seja \emph{p}(\emph{x}) = \emph{x}3\emph{−}3\emph{x}2+3\emph{x−}1. então
\emph{p}(\emph{x}) é um polinômio sobre Z com indeterminado \emph{x}.
Para qualquer número inteiro \emph{α}, o valor \emph{p}(\emph{α}) também
será um número inteiro. Por
\end{quote}
\end{minipage} \\
\midrule()
\endhead
\bottomrule()
\end{longtable}

\begin{quote}
{[}a{]}Ao lidar com polinômios, normalmente reservaremos a letra
\emph{x} para a variável indeterminada e usaremos as letras gregas
\emph{α,β,γ} (LATEX code: \textbackslash alpha, \textbackslash beta,
\textbackslash gamma) para números a serem substituídos em um polinômio.
\end{quote}

17

\begin{quote}
18 \emph{Capítulo 0. Começando}

exemplo
\end{quote}

\begin{longtable}[]{@{}
  >{\raggedright\arraybackslash}p{(\columnwidth - 4\tabcolsep) * \real{0.3333}}
  >{\raggedright\arraybackslash}p{(\columnwidth - 4\tabcolsep) * \real{0.3333}}
  >{\raggedright\arraybackslash}p{(\columnwidth - 4\tabcolsep) * \real{0.3333}}@{}}
\toprule()
\begin{minipage}[b]{\linewidth}\raggedright
\begin{quote}
\emph{p}(0) = 03\emph{−}3\emph{·}02+3\emph{·}0\emph{−}1 = \emph{−}1
\end{quote}
\end{minipage} & \begin{minipage}[b]{\linewidth}\raggedright
e
\end{minipage} & \begin{minipage}[b]{\linewidth}\raggedright
\begin{quote}
\emph{p}(3) = 33\emph{−}3\emph{·}32+3\emph{·}3\emph{−}1 = 8
\end{quote}
\end{minipage} \\
\midrule()
\endhead
\bottomrule()
\end{longtable}

◁

\begin{longtable}[]{@{}
  >{\raggedright\arraybackslash}p{(\columnwidth - 0\tabcolsep) * \real{1.0000}}@{}}
\toprule()
\begin{minipage}[b]{\linewidth}\raggedright
✦ \textbf{Definition 0.0.37}\\
Seja \emph{p}(\emph{x}) m polinômio. Uma \textbf{raiz} de
\emph{p}(\emph{x}) é um número complexo \emph{α} de tal modo que
\emph{p}(\emph{α}) = 0.\strut
\end{minipage} \\
\midrule()
\endhead
\bottomrule()
\end{longtable}

\begin{longtable}[]{@{}
  >{\raggedright\arraybackslash}p{(\columnwidth - 8\tabcolsep) * \real{0.2000}}
  >{\raggedright\arraybackslash}p{(\columnwidth - 8\tabcolsep) * \real{0.2000}}
  >{\raggedright\arraybackslash}p{(\columnwidth - 8\tabcolsep) * \real{0.2000}}
  >{\raggedright\arraybackslash}p{(\columnwidth - 8\tabcolsep) * \real{0.2000}}
  >{\raggedright\arraybackslash}p{(\columnwidth - 8\tabcolsep) * \real{0.2000}}@{}}
\toprule()
\multicolumn{5}{@{}>{\raggedright\arraybackslash}p{(\columnwidth - 8\tabcolsep) * \real{1.0000} + 8\tabcolsep}@{}}{%
\begin{minipage}[b]{\linewidth}\raggedright
\begin{quote}
A \emph{fórmula quadrática} (\textbf{??}) nos diz que as raízes do
polinômio \emph{x}2+\emph{ax}+\emph{b}, onde \emph{a,b ∈} C, são
precisamente o complexo números
\end{quote}
\end{minipage}} \\
\midrule()
\endhead
\emph{−a}+ & \begin{minipage}[t]{\linewidth}\raggedright
\begin{quote}
\emph{√a}2\emph{−}4\emph{b}

2
\end{quote}
\end{minipage} & e & \begin{minipage}[t]{\linewidth}\raggedright
\begin{quote}
\emph{−a−}
\end{quote}
\end{minipage} & \begin{minipage}[t]{\linewidth}\raggedright
\begin{quote}
\emph{√a}2\emph{−}4\emph{b}

2
\end{quote}
\end{minipage} \\
\multicolumn{5}{@{}>{\raggedright\arraybackslash}p{(\columnwidth - 8\tabcolsep) * \real{1.0000} + 8\tabcolsep}@{}}{%
\begin{minipage}[t]{\linewidth}\raggedright
\begin{quote}
Observe que evitamos o símbolo `\emph{±}', que é comumente encontrado em
discussões sobre polinômios quadráticos. O símbolo `\emph{±}' é perigoso
porque pode suprimir a palavra `e'ou a palavra `ou', dependendo do
contexto --- este tipo de ambigüidade não é algo com

o qual desejaremos lidar ao discutir a lógica estrutura de uma
proposição em \textbf{??}!
\end{quote}
\end{minipage}} \\
\bottomrule()
\end{longtable}

✐ \textbf{Example 0.0.38}

\begin{quote}
Seja \emph{p}(\emph{x}) = \emph{x}2\emph{−}2\emph{x}+5. A fórmula
quadrática nos diz que as raízes de \emph{p} são
\end{quote}

\begin{longtable}[]{@{}
  >{\raggedright\arraybackslash}p{(\columnwidth - 12\tabcolsep) * \real{0.1429}}
  >{\raggedright\arraybackslash}p{(\columnwidth - 12\tabcolsep) * \real{0.1429}}
  >{\raggedright\arraybackslash}p{(\columnwidth - 12\tabcolsep) * \real{0.1429}}
  >{\raggedright\arraybackslash}p{(\columnwidth - 12\tabcolsep) * \real{0.1429}}
  >{\raggedright\arraybackslash}p{(\columnwidth - 12\tabcolsep) * \real{0.1429}}
  >{\raggedright\arraybackslash}p{(\columnwidth - 12\tabcolsep) * \real{0.1429}}
  >{\raggedright\arraybackslash}p{(\columnwidth - 12\tabcolsep) * \real{0.1429}}@{}}
\toprule()
\begin{minipage}[b]{\linewidth}\raggedright
2+
\end{minipage} & \begin{minipage}[b]{\linewidth}\raggedright
\emph{√}4\emph{−}4\emph{·}5

2
\end{minipage} & \begin{minipage}[b]{\linewidth}\raggedright
\begin{quote}
= 1+\emph{√−}4 = 1+2\emph{i}
\end{quote}
\end{minipage} & \begin{minipage}[b]{\linewidth}\raggedright
and
\end{minipage} & \begin{minipage}[b]{\linewidth}\raggedright
2\emph{−}
\end{minipage} & \begin{minipage}[b]{\linewidth}\raggedright
\emph{√}4\emph{−}4\emph{·}5

\begin{quote}
2
\end{quote}
\end{minipage} & \begin{minipage}[b]{\linewidth}\raggedright
\begin{quote}
= 1\emph{−√−}4 = 1\emph{−}2\emph{i}
\end{quote}
\end{minipage} \\
\midrule()
\endhead
\multicolumn{7}{@{}>{\raggedright\arraybackslash}p{(\columnwidth - 12\tabcolsep) * \real{1.0000} + 12\tabcolsep}@{}}{%
\begin{minipage}[t]{\linewidth}\raggedright
\begin{quote}
Os números 1 + 2\emph{i} e 1 \emph{−} 2\emph{i} estão relacionados
porque suas partes reais são iguais e suas partes imaginárias diferem
apenas por um sinal. Exercise 0.0.39 generaliza esta
\end{quote}
\end{minipage}} \\
\multicolumn{2}{@{}>{\raggedright\arraybackslash}p{(\columnwidth - 12\tabcolsep) * \real{0.2857} + 2\tabcolsep}}{%
\begin{minipage}[t]{\linewidth}\raggedright
\begin{quote}
observação.
\end{quote}
\end{minipage}} &
\multicolumn{5}{>{\raggedright\arraybackslash}p{(\columnwidth - 12\tabcolsep) * \real{0.7143} + 8\tabcolsep}@{}}{%
◁} \\
\multicolumn{7}{@{}>{\raggedright\arraybackslash}p{(\columnwidth - 12\tabcolsep) * \real{1.0000} + 12\tabcolsep}@{}}{%
\begin{minipage}[t]{\linewidth}\raggedright
✎ \textbf{Exercise 0.0.39}\\
Seja \emph{α} = \emph{a} + \emph{bi} um número complexo, onde \emph{a,b
∈} R. Prove que \emph{α} é a raiz de um polinômio quadrático sobre R e
encontre a outra raiz desse polinômio. ◁\strut
\end{minipage}} \\
\bottomrule()
\end{longtable}

\begin{quote}
O exercício a seguir prova o conhecido resultado que classifica o número
de raízes reais

de um polinômio sobre R em termos de seus coeficientes.
\end{quote}

✎ \textbf{Exercise 0.0.40}

\begin{quote}
Seja \emph{a,b ∈} C e seja \emph{p}(\emph{x}) =
\emph{x}2+\emph{ax}+\emph{b}. O valor ∆ = \emph{a}2\emph{−}4\emph{b} é
chamado de \textbf{discriminante}

de \emph{p}. Prove que \emph{p} tem duas raízes se ∆ \emph{̸}= 0 e uma
raiz se ∆ = 0. Além disso, se \emph{a,b ∈} R, prove que \emph{p} não tem
raízes reais se ∆ \emph{\textless{}} 0, uma raiz real se ∆ = 0, e duas
raízes reais se

∆ \emph{\textgreater{}} 0. ◁
\end{quote}

18

\begin{quote}
\emph{Capítulo 0. Começando} 19
\end{quote}

✐ \textbf{Example 0.0.41}

\begin{quote}
Considere o polinômio \emph{x}2\emph{−} 2\emph{x} + 5. Seu discriminante
é igual a (\emph{−}2)2\emph{−} 4 \emph{·} 5 = \emph{−}16, que é
negativo. Exercise 0.0.40 nos diz que tem duas raízes, nenhuma das quais
é real.

Isto foi verificado por Example 0.0.38, onde descobrimos que as raízes
de \emph{x}2\emph{−} 2\emph{x} + 5 are 1+2\emph{i} e 1\emph{−}2\emph{i}.

Agora considere o polinômio \emph{x}2\emph{−}2\emph{x−}3. Seu
discriminante é igual a (\emph{−}2)2\emph{−}4\emph{·}(\emph{−}3) = 16, o
que é positivo. Exercise 0.0.40 nos diz que tem duas raízes, ambas
reais; e real-

mente
\end{quote}

\emph{x}2\emph{−}2\emph{x−}3 = (\emph{x}+1)(\emph{x−}3)

\begin{quote}
então as raízes de \emph{x}2\emph{−}2\emph{x−}3 são \emph{−}1 e 3. ◁
\end{quote}

19

\begin{quote}
20 \emph{Capítulo 0. Começando}

Section 0.E

\textbf{Chapter 0 exercises}

\textbf{0.1.} O site de compartilhamento de vídeos \emph{YouTube}
atribui a cada vídeo um identific-

ador único, que é uma cadeia de 11 caracteres do conjunto

\emph{\{}A\emph{,}B\emph{,...,}Z\emph{,}a\emph{,}b\emph{,...,}z\emph{,}0\emph{,}1\emph{,}2\emph{,}3\emph{,}4\emph{,}5\emph{,}6\emph{,}7\emph{,}8\emph{,}9\emph{,}-\emph{,}\_\emph{\}}
\end{quote}

\begin{longtable}[]{@{}
  >{\raggedright\arraybackslash}p{(\columnwidth - 0\tabcolsep) * \real{1.0000}}@{}}
\toprule()
\begin{minipage}[b]{\linewidth}\raggedright
\begin{quote}
Esta cadeia é na verdade um número natural expresso em base 64, onde os
caracteres no conjunto acima representam os números 0 a 63, na mesma
ordem --- portanto C representa 2, \emph{mathttc} representa 28, 3
representa 55 e \_ representa 63. De acordo com este esquema, encontre o
número natural cuja expansão na base 64 é dQw4w9WgXcQ, e encontre a
expansão na base 64 do número natural 7159047702620056984.

\textbf{0.2.} Seja \emph{a,b,c,d ∈} Z. Sob quais condições
(\emph{a}+\emph{b√}2)(\emph{c}+\emph{d√}2) é um número inteiro?

\textbf{0.3.} Suponha que um inteiro \emph{m} deixe um resto de \emph{i}
quando dividido por 3, e um inteiro \emph{n} deixe um resto de \emph{j}
quando dividido por 3, onde 0 ⩽ \emph{i, j \textless{}} 3. Prove que
\emph{m}+\emph{n} e \emph{i}+ \emph{j} deixam o mesmo resto quando
divididos por 3.
\end{quote}
\end{minipage} \\
\midrule()
\endhead
\bottomrule()
\end{longtable}

\begin{quote}
\textbf{0.4.} Quais são os possíveis restos de \emph{n}2quando dividido
por 3, onde \emph{n ∈} Z?
\end{quote}

\begin{longtable}[]{@{}
  >{\raggedright\arraybackslash}p{(\columnwidth - 0\tabcolsep) * \real{1.0000}}@{}}
\toprule()
\begin{minipage}[b]{\linewidth}\raggedright
✦ \textbf{Definition 0.E.1}\\
set \emph{X} é \textbf{closed} sob uma operação \emph{⊙} se, sempre que
\emph{a} e \emph{b} são elementos de \emph{X}, \emph{a ⊙ b} também é um
elemento de \emph{X}.\strut
\end{minipage} \\
\midrule()
\endhead
\bottomrule()
\end{longtable}

\begin{longtable}[]{@{}
  >{\raggedright\arraybackslash}p{(\columnwidth - 6\tabcolsep) * \real{0.2500}}
  >{\raggedright\arraybackslash}p{(\columnwidth - 6\tabcolsep) * \real{0.2500}}
  >{\raggedright\arraybackslash}p{(\columnwidth - 6\tabcolsep) * \real{0.2500}}
  >{\raggedright\arraybackslash}p{(\columnwidth - 6\tabcolsep) * \real{0.2500}}@{}}
\toprule()
\multicolumn{4}{@{}>{\raggedright\arraybackslash}p{(\columnwidth - 6\tabcolsep) * \real{1.0000} + 6\tabcolsep}@{}}{%
\begin{minipage}[b]{\linewidth}\raggedright
\begin{quote}
Em Questions 0.5 to 0.11, determine quais dos conjuntos de números N, Z,
Q e R são fechados em a operação \emph{⊙} definida na pergunta.
\end{quote}
\end{minipage}} \\
\midrule()
\endhead
\multirow{2}{*}{\begin{minipage}[t]{\linewidth}\raggedright
\begin{quote}
\textbf{0.5.} \emph{a⊙b} = \emph{a}+\emph{b}\\
\textbf{0.6.} \emph{a⊙b} = \emph{a−b}\\
\textbf{0.7.} \emph{a⊙b} = (\emph{a−b})(\emph{a}+\emph{b})\\
\textbf{0.8.} \emph{a⊙b} = (\emph{a−}1)(\emph{b−}1)+2(\emph{a}+\emph{b})
\end{quote}\strut
\end{minipage}} & \begin{minipage}[t]{\linewidth}\raggedright
\begin{quote}
\textbf{0.9.} \emph{a⊙b} =
\end{quote}
\end{minipage} &
\multicolumn{2}{>{\raggedright\arraybackslash}p{(\columnwidth - 6\tabcolsep) * \real{0.5000} + 2\tabcolsep}@{}}{%
\begin{minipage}[t]{\linewidth}\raggedright
\emph{a}\\
\emph{b}2+1\strut
\end{minipage}} \\
&
\multicolumn{2}{>{\raggedright\arraybackslash}p{(\columnwidth - 6\tabcolsep) * \real{0.5000} + 2\tabcolsep}}{%
\begin{minipage}[t]{\linewidth}\raggedright
\begin{quote}
\textbf{0.10.} \emph{a⊙b} = \textbf{0.11.} \emph{a⊙b} =
\end{quote}
\end{minipage}} & \begin{minipage}[t]{\linewidth}\raggedright
\emph{a}\\
\emph{√b}2+1

\begin{quote}
�\emph{ab}

0\\
if \emph{b \textgreater{}} 0
\end{quote}

if \emph{b ̸∈} Q\strut
\end{minipage} \\
\bottomrule()
\end{longtable}

\begin{longtable}[]{@{}
  >{\raggedright\arraybackslash}p{(\columnwidth - 0\tabcolsep) * \real{1.0000}}@{}}
\toprule()
\begin{minipage}[b]{\linewidth}\raggedright
✦ \textbf{Definition 0.E.2}\\
Um número complexo \emph{α} é \textbf{algébrico} se \emph{p}(\emph{α}) =
0 para algum polinômio diferente de zero \emph{p}(\emph{x}) sobre
Q.\strut
\end{minipage} \\
\midrule()
\endhead
\bottomrule()
\end{longtable}

\begin{quote}
\textbf{0.12.} SeJA \emph{x} seja um número racional. Prove que \emph{x}
é um número algébrico.
\end{quote}

20

\begin{longtable}[]{@{}
  >{\raggedright\arraybackslash}p{(\columnwidth - 6\tabcolsep) * \real{0.2500}}
  >{\raggedright\arraybackslash}p{(\columnwidth - 6\tabcolsep) * \real{0.2500}}
  >{\raggedright\arraybackslash}p{(\columnwidth - 6\tabcolsep) * \real{0.2500}}
  >{\raggedright\arraybackslash}p{(\columnwidth - 6\tabcolsep) * \real{0.2500}}@{}}
\toprule()
\multicolumn{3}{@{}>{\raggedright\arraybackslash}p{(\columnwidth - 6\tabcolsep) * \real{0.7500} + 4\tabcolsep}}{%
\begin{minipage}[b]{\linewidth}\raggedright
\emph{Section 0.E. Chapter 0 exercises}
\end{minipage}} &
\multirow{3}{*}{\begin{minipage}[b]{\linewidth}\raggedright
21
\end{minipage}} \\
\begin{minipage}[b]{\linewidth}\raggedright
\textbf{0.13.} Prove que
\end{minipage} &
\multicolumn{2}{>{\raggedright\arraybackslash}p{(\columnwidth - 6\tabcolsep) * \real{0.5000} + 2\tabcolsep}}{%
\begin{minipage}[b]{\linewidth}\raggedright
\begin{quote}
\emph{√}2 é um número algébrico.
\end{quote}
\end{minipage}} \\
\begin{minipage}[b]{\linewidth}\raggedright
\textbf{0.14.} Prove que
\end{minipage} & \begin{minipage}[b]{\linewidth}\raggedright
\emph{√}2+
\end{minipage} & \begin{minipage}[b]{\linewidth}\raggedright
\begin{quote}
\emph{√}3 é um número algébrico.
\end{quote}
\end{minipage} \\
\midrule()
\endhead
\bottomrule()
\end{longtable}

\textbf{0.15.} Prove que \emph{x} + \emph{yi} é um número algébrico,
onde \emph{x} e \emph{y} são dois números racionais quaisquer.

\textbf{Perguntas verdadeiras-falsas}

Nas questões 1.16 a 1.23, determine (com prova) se a afirmação é
verdadeira ou falsa \textbf{0.16.} Todo número inteiro é um número
natural.

\textbf{0.17.} Todo número inteiro é um número racional.

\textbf{0.18.} Todo número inteiro divide zero.

\textbf{0.19.} Todo número inteiro divide seu quadrado.

\textbf{0.20.} O quadrado de todo número racional é um número racional.

\textbf{0.21.} O quadrado de todo número racional é um número racional.

\textbf{0.22.} Quando um inteiro \emph{a} é dividido por um inteiro
positivo \emph{b}, o resto é sempre menor que \emph{a}.

\textbf{0.23.} Cada polinômio quadrático tem duas raízes complexas
distintas.

\textbf{Sempre - Às vezes - Nunca questiona}

In Questions 0.24 to 0.32, determine (with proof) whether the conclusion
is always, sometimes or never true under the given hypotheses.

\textbf{0.24.} Seja \emph{n,b}1\emph{,b}2 \emph{∈} N com 1
\emph{\textless{} n \textless{} b}1 \emph{\textless{} b}2. Então a
expansão de base \emph{b}1 de \emph{n} é igual à expansão de base
\emph{b}2 de \emph{n}.

\textbf{0.25.} Seja \emph{n,b}1\emph{,b}2 \emph{∈} N com 1
\emph{\textless{} b}1 \emph{\textless{} b}2 \emph{\textless{} n}. Então
a expansão de base \emph{b}1 de \emph{n} é igual à expansão de base
\emph{b}2 de \emph{n}.

\textbf{0.26.} Seja \emph{a,b,c ∈} Z e suponha que \emph{a} divida
\emph{c} e \emph{b} divida \emph{c}. Então \emph{ab} divide \emph{c}.
\textbf{0.27.} Seja \emph{a,b,c ∈} Z e suponha que \emph{a} divida
\emph{c} e \emph{b} divida \emph{c}. Então \emph{ab} divide \emph{c}2.

\textbf{0.28.} Seja \emph{x,y ∈} Q e seja \emph{a,b,c,d ∈} Z com
\emph{cy}+\emph{d ̸}= 0. Então\emph{ax}+\emph{b cy}+\emph{d∈} Q.
\textbf{0.29.} Seja\emph{a b}um número racional. Então \emph{a ∈} Z e
\emph{b ∈} Z.

21

22 \emph{Capítulo 0. Começando}

\textbf{0.30.} Seja \emph{x ∈} R e assuma que \emph{x}2\emph{∈} Q. Então
\emph{x ∈} Q.

\textbf{0.31.} Seja \emph{x ∈} R e assuma que \emph{x}2+1 \emph{∈} Q e
\emph{x}5+1 \emph{∈} Q. Então \emph{x ∈} Q.

\textbf{0.32.} Seja \emph{p}(\emph{x}) = \emph{ax}2+ \emph{bx} +
\emph{c} um polinômio com \emph{a,b,c ∈} R e \emph{a ̸}= 0, e suponha
que \emph{u}+\emph{vi} seja uma raiz complexa de \emph{p}(\emph{x}) com
\emph{v ̸}= 0. Então \emph{u−vi} é uma raiz de \emph{p}(\emph{x}).

22

\begin{quote}
\textbf{Part I}\\
\textbf{Conceitos centrais}
\end{quote}

23

\textbf{Apêndices}

25

Appendix A

\textbf{Miscelânea matemática}

Houve diversas ocasiões no livro em que evitamos nos aprofundar muito
nos aspec-tos mais técnicos ou obscuros de uma definição ou prova.
Geralmente isso acontecia porque a exploração desses aspectos não era
central para o tópico em questão, ou porque os detalhes envolvidos eram
suficientemente confusos para que fornecer todos os detal-hes ofuscasse
as ideias principais em discussão.

Este apêndice fornece um espaço para os comentários que não fizemos, os
teoremas que não provamos, os detalhes que não fornecemos e as
obscuridades que não exploramos.

Começamos com uma rápida olhada nos \emph{fundamentos da matemática} em
Section A.1. Forneceremos os axiomas para a teoria dos conjuntos de
Zermelo-Fraenkel (ZF), que codifica todos os objetos matemáticos como
conjuntos e nos permite derivar todos os objetos matemáticos de uma
coleção de axiomas. Também demonstraremos como co-dificar números
naturais, inteiros, números racionais e números complexos dentro desta
estrutura.

27

28 \emph{Appendix A. Miscelânea matemática}

Section A.1

\textbf{Definir fundamentos teóricos}

\textbf{Teoria ingênua dos conjuntos e o Paradoxo de Russel}

28

\begin{quote}
\emph{Section A.2. Construções dos conjuntos de números} 29

Section A.2

\textbf{Construções dos conjuntos de números}

Na base da matemática mais comumente usada, a teoria dos conjuntos de
Zermelo-Fraenkel com o axioma da escolha (discutido detalhadamente em
Section A.1), todo objeto matemático é um conjunto. Isto levanta a
questão: e os outros objetos matemáti-cos? Por exemplo, se os números
são conjuntos, que conjuntos são eles? E quanto às funções e relações?

Para funções e relações, existe uma solução fácil: simplesmente as
identificamos com seus gráficos. Por exemplo, podemos fingir que a
função \emph{f} : R \emph{→} R dada por \emph{f}(\emph{x}) = \emph{x}2
para todo \emph{x ∈} R ` é' o conjunto \emph{\{}(\emph{x,y}) \emph{∈}
R\emph{×}R \emph{\textbar{} y} = \emph{x}2\emph{\}}.

Para números, entretanto, a resposta não é tão simples. Por exemplo, que
conjunto deve ser o número 3? E faz diferença se considerarmos 3 um
número natural ou um número real?

Esta seção apresenta uma das muitas maneiras possíveis de codificar
números -- isto é, números naturais, inteiros, números racionais,
números reais e números complexos --como conjuntos.

\textbf{Os Números Naturais}

Podemos usar a estrutura fornecida pela teoria dos conjuntos de
Zermelo-Fraenkel (Sec-tion A.1) para fornecer construções teóricas dos
conjuntos de números N, Z, Q, R e C. Na verdade, se quisermos raciocinar
sobre matemática dentro dos limites de ZF, deve-mos codificar tudo
(incluindo números) como conjuntos!

Começaremos com uma construção teórica dos conjuntos dos números
naturais --- isto é, construiremos uma noção de números naturais no
sentido de \textbf{??}. Codificaremos os números naturais como
conjuntos, chamados \emph{números naturais de von Neumann}.
Iden-tificaremos o número natural 0 com o conjunto vazio ∅, e
identificaremos a operação sucessora s com uma operação envolvendo
conjuntos.
\end{quote}

\begin{longtable}[]{@{}
  >{\raggedright\arraybackslash}p{(\columnwidth - 0\tabcolsep) * \real{1.0000}}@{}}
\toprule()
\begin{minipage}[b]{\linewidth}\raggedright
✦ \textbf{Definition A.2.1}\\
Um \textbf{número natural de Von Neuman} ié qualquer conjunto que pode
ser obtido de ∅tomando repetidamente conjuntos sucessores (ver
\textbf{??}). Escrito 0vN = ∅ e (\emph{n}+1)vN = (\emph{n}vN)+; que é

\begin{longtable}[]{@{}
  >{\raggedright\arraybackslash}p{(\columnwidth - 10\tabcolsep) * \real{0.1667}}
  >{\raggedright\arraybackslash}p{(\columnwidth - 10\tabcolsep) * \real{0.1667}}
  >{\raggedright\arraybackslash}p{(\columnwidth - 10\tabcolsep) * \real{0.1667}}
  >{\raggedright\arraybackslash}p{(\columnwidth - 10\tabcolsep) * \real{0.1667}}
  >{\raggedright\arraybackslash}p{(\columnwidth - 10\tabcolsep) * \real{0.1667}}
  >{\raggedright\arraybackslash}p{(\columnwidth - 10\tabcolsep) * \real{0.1667}}@{}}
\toprule()
\begin{minipage}[b]{\linewidth}\raggedright
\begin{quote}
0vN = ∅\emph{,}
\end{quote}
\end{minipage} & \begin{minipage}[b]{\linewidth}\raggedright
1vN = ∅+\emph{,}
\end{minipage} & \begin{minipage}[b]{\linewidth}\raggedright
2vN = ∅++\emph{,}
\end{minipage} & \begin{minipage}[b]{\linewidth}\raggedright
3vN = ∅+++\emph{,}
\end{minipage} & \begin{minipage}[b]{\linewidth}\raggedright
4vN = ∅++++\emph{,}
\end{minipage} & \begin{minipage}[b]{\linewidth}\raggedright
\emph{...}
\end{minipage} \\
\midrule()
\endhead
\bottomrule()
\end{longtable}\strut
\end{minipage} \\
\midrule()
\endhead
\bottomrule()
\end{longtable}

29

\begin{quote}
30 \emph{Appendix A. Miscelânea matemática}
\end{quote}

✐ \textbf{Example A.2.2}\\
Os três primeiros números naturais de von Neumann são:

\begin{quote}
• 0vN = ∅;

• 1vN = ∅+= ∅\emph{∪\{}∅\emph{\}} = \emph{\{}∅\emph{\}};\\
• 2vN = ∅++= \emph{\{}∅\emph{\}}+= \emph{\{}∅\emph{\}∪\{\{}∅\emph{\}\}}
= \emph{\{}∅\emph{,\{}∅\emph{\}\}}.
\end{quote}

◁

\begin{longtable}[]{@{}
  >{\raggedright\arraybackslash}p{(\columnwidth - 2\tabcolsep) * \real{0.5000}}
  >{\raggedright\arraybackslash}p{(\columnwidth - 2\tabcolsep) * \real{0.5000}}@{}}
\toprule()
\begin{minipage}[b]{\linewidth}\raggedright
✎ \textbf{Exercise A.2.3}\\
Escreva os elementos de 3vN (= ∅+++) e de 4vN.\strut
\end{minipage} & \begin{minipage}[b]{\linewidth}\raggedright
◁
\end{minipage} \\
\midrule()
\endhead
\bottomrule()
\end{longtable}

✎ \textbf{Exercise A.2.4}

\begin{longtable}[]{@{}
  >{\raggedright\arraybackslash}p{(\columnwidth - 2\tabcolsep) * \real{0.5000}}
  >{\raggedright\arraybackslash}p{(\columnwidth - 2\tabcolsep) * \real{0.5000}}@{}}
\toprule()
\multicolumn{2}{@{}>{\raggedright\arraybackslash}p{(\columnwidth - 2\tabcolsep) * \real{1.0000} + 2\tabcolsep}@{}}{%
\begin{minipage}[b]{\linewidth}\raggedright
\begin{quote}
Lembre-se da definição dos números naturais de von Neumann em Definition
A.2.1.
\end{quote}
\end{minipage}} \\
\midrule()
\endhead
\begin{minipage}[t]{\linewidth}\raggedright
\begin{quote}
Prove que \emph{\textbar n}vN\emph{\textbar{}} = \emph{n} para todos
\emph{n ∈} N.
\end{quote}
\end{minipage} & ◁ \\
\multicolumn{2}{@{}>{\raggedright\arraybackslash}p{(\columnwidth - 2\tabcolsep) * \real{1.0000} + 2\tabcolsep}@{}}{%
\begin{minipage}[t]{\linewidth}\raggedright
\begin{longtable}[]{@{}
  >{\raggedright\arraybackslash}p{(\columnwidth - 0\tabcolsep) * \real{1.0000}}@{}}
\toprule()
\begin{minipage}[b]{\linewidth}\raggedright
✦ \textbf{Construction A.2.5}\\
Construímos o conjunto NvN de todos os números naturais de von Neumann
como segue. Seja \emph{X} um conjunto arbitrário que satisfaça o axioma
do infinito (\textbf{??}) e então defina NvN como a interseção de todos
os subconjuntos de \emph{X} que também satisfazem o axioma do infinito
--- isto é:\strut
\end{minipage} \\
\midrule()
\endhead
\bottomrule()
\end{longtable}

\begin{quote}
NvN = \emph{\{x ∈ X \textbar{} ∀U ∈ P}(\emph{X})\emph{,} {[}\emph{U}
satisfaz o axioma do infinito \emph{⇒ x ∈ U}{]}\emph{\}}
\end{quote}\strut
\end{minipage}} \\
\bottomrule()
\end{longtable}

\begin{quote}
A existência de NvN segue dos axiomas do conjunto de potências
(\textbf{??}) e separação (\textbf{??}).
\end{quote}

\begin{longtable}[]{@{}
  >{\raggedright\arraybackslash}p{(\columnwidth - 0\tabcolsep) * \real{1.0000}}@{}}
\toprule()
\begin{minipage}[b]{\linewidth}\raggedright
\begin{longtable}[]{@{}
  >{\raggedright\arraybackslash}p{(\columnwidth - 0\tabcolsep) * \real{1.0000}}@{}}
\toprule()
\begin{minipage}[b]{\linewidth}\raggedright
✣ \textbf{Theorem A.2.6}\\
O conjunto NvN, elemento zero 0vN e função sucessora \emph{s} : NvN
\emph{para}NvN definido por s(\emph{n}vN) = \emph{n}+\strut
\end{minipage} \\
\midrule()
\endhead
\bottomrule()
\end{longtable}

\begin{quote}
vNpara todos \emph{n}vN \emph{∈} NvN, defina uma noção de números
naturais.
\end{quote}\strut
\end{minipage} \\
\midrule()
\endhead
\bottomrule()
\end{longtable}

\begin{quote}
\emph{\textbf{Proof}}\\
Devemos verificar os axiomas de Peano, que são condições (i)--(iii) de
\textbf{??}.

Para provar (i), observe que para todos os conjuntos \emph{X} temos
\emph{X}+= \emph{X ∪\{X\}}, de modo que \emph{X ∈ X}+. Em particular,
temos \emph{n}vN \emph{∈ n}+∅ = 0vN. vNpara todos \emph{n}vN \emph{∈}
NvN e, portanto, \emph{n}+ vN\emph{̸}=

Para (ii), seja \emph{m}vN\emph{,n}vN \emph{∈} NvN e assuma que
\emph{m}+ vN= \emph{n}+ vN. Então \emph{m}vN = \emph{n}vN por
\textbf{??}.

Para (iii), seja \emph{X} um conjunto e suponha que 0vN \emph{∈ X} e,
para todo \emph{n}vN \emph{∈} NvN, se \emph{n}vN\emph{X}, então
\emph{n}+ struction A.2.5 temos NvN \emph{⊆ X}. vN\emph{∈ X}. Então
\emph{X} satisfaz o axioma do infinito (\textbf{??}), e então por Con-□
\end{quote}

30

\begin{quote}
\emph{Section A.2. Construções dos conjuntos de números} 31

À luz de Theorem A.2.6, podemos declarar que ``os números naturais'' são
os números naturais de von Neumann, e pronto. Como tal, você pode - se
quiser - pensar em todos os números naturais nessas notas como
\emph{sendo} seu número natural de von Neumann cor-respondente. Com isto
em mente, omitimos agora o subscrito `vN', deixando implícito o fato de
que estamos nos referindo aos números naturais de von Neumann.

No entanto, existem muitas outras noções possíveis de números naturais.
Em The-orem A.2.8, provamos que quaisquer duas noções de números
naturais são essencial-mente as mesmas e, portanto, as especificidades
de como realmente definimos N, o elemento zero e a operação sucessora,
são irrelevantes para a maioria dos propósitos. .

Primeiro provaremos o seguinte lema útil, que fornece um meio
conveniente de provar quando uma função é a função identidade
(\textbf{??}).
\end{quote}

✣ \textbf{Lemma A.2.7}\\
Seja (N\emph{,z,s}) seja uma noção de números naturais, e seja \emph{j}
: N \emph{→} N uma função tal que \emph{j}(\emph{z}) = 0 e
\emph{j}(\emph{s}(\emph{n})) = \emph{s}( \emph{j}(\emph{n})) para todo
\emph{n ∈} N. Então \emph{j} = idN.

\begin{quote}
\emph{Proof.} Por \textbf{??}, existe uma função única \emph{i} : N
\emph{→} N tal que \emph{i}(\emph{z}) = 0 e \emph{i}(\emph{s}(\emph{n}))
= \emph{s}(\emph{i}(\emph{n})) para todo \emph{n ∈} N. Mas então:

\emph{j} = \emph{i} pela unicidade de \emph{i}, já que \emph{j} satisfaz
as mesmas condições que \emph{i}; e

idN = \emph{i} pela exclusividade de \emph{i}, já que idN(\emph{z}) =
\emph{z} e idN(\emph{s}(\emph{n})) = \emph{s}(\emph{n}) =
\emph{s}(idN(\emph{n})) para todos \emph{n ∈} N. Portanto, \emph{j} =
idN, conforme necessário.
\end{quote}

\begin{longtable}[]{@{}
  >{\raggedright\arraybackslash}p{(\columnwidth - 0\tabcolsep) * \real{1.0000}}@{}}
\toprule()
\begin{minipage}[b]{\linewidth}\raggedright
\end{minipage} \\
\midrule()
\endhead
\bottomrule()
\end{longtable}

\begin{longtable}[]{@{}
  >{\raggedright\arraybackslash}p{(\columnwidth - 0\tabcolsep) * \real{1.0000}}@{}}
\toprule()
\begin{minipage}[b]{\linewidth}\raggedright
\begin{longtable}[]{@{}
  >{\raggedright\arraybackslash}p{(\columnwidth - 0\tabcolsep) * \real{1.0000}}@{}}
\toprule()
\begin{minipage}[b]{\linewidth}\raggedright
✣ \textbf{Theorem A.2.8}\\
Quaisquer duas noções de números naturais são essencialmente iguais, num
sentido muito forte. Mais precisamente, se (N1\emph{,z}1\emph{,s}1) e
(N2\emph{,z}2\emph{,s}2) são noções de números naturais, então existe
uma bijeção única \emph{f} : N1 \emph{→} N2 tal que \emph{f}(\emph{z}1)
= \emph{z}2 e \emph{f}(\emph{s}1(\emph{n})) =\strut
\end{minipage} \\
\midrule()
\endhead
\bottomrule()
\end{longtable}

\begin{quote}
\emph{s}2( \emph{f}(\emph{n})) para todo \emph{n ∈} N1.

\emph{Proof.} Aplicando o teorema da recursão (\textbf{??}) para
(N1\emph{,z}1\emph{,s}1), com \emph{X} = N2, \emph{a} = \emph{z}2 e

\emph{h} : \emph{mathbbN}1 \emph{×} N2 \emph{→} N2 definido por
\emph{h}(\emph{m,n}) = \emph{s}2(\emph{n}) para todos \emph{m ∈} N1 e
\emph{n ∈} N2, obtemos uma função \emph{f} : N1 \emph{→} N2 tal que
\emph{f}(\emph{z}1) = \emph{z}2 e \emph{f}(\emph{s}1(\emph{n})) =
\emph{s}2( \emph{f}(\emph{n})) para todo \emph{n ∈} N1. Isso também nos
dá a unicidade de \emph{f}, então resta apenas provar que \emph{f} é uma
bijeção.

Da mesma forma, aplicando \textbf{??} a (N2\emph{,z}2\emph{,s}2), com
\emph{X} = N1, \emph{a} = \emph{z}1 e \emph{h} : N2 \emph{×}N1 \emph{→}
N1 definido por \emph{h}(\emph{m,n}) = \emph{s}1(\emph{n}) para todos
\emph{m ∈} N2 e \emph{n ∈} N1, obtemos uma função (única!)
\end{quote}

\begin{longtable}[]{@{}
  >{\raggedright\arraybackslash}p{(\columnwidth - 0\tabcolsep) * \real{1.0000}}@{}}
\toprule()
\begin{minipage}[b]{\linewidth}\raggedright
\end{minipage} \\
\midrule()
\endhead
\bottomrule()
\end{longtable}

\begin{quote}
\emph{g} : N2 \emph{→} N1 tal que \emph{g}(\emph{z}2) = \emph{z}1 e
\emph{g}(\emph{s}2(\emph{n})) = \emph{s}1(\emph{g}(\emph{n})) para todos
\emph{n ∈} N2.
\end{quote}\strut
\end{minipage} \\
\midrule()
\endhead
\bottomrule()
\end{longtable}

31

\begin{quote}
32 \emph{Appendix A. Miscelânea matemática}

But then g( f (z1)) = g(z2) = z1 and, for all n N1, we have

g( f (s1(n))) = g(s2( f (n))) = s2(g( f (n)))

e assim g ◦ f = idN1 pelo Lema A.2.7. Com isso resolvido, vamos agora
simplesmente

trabalhar com uma noção fixa de números naturais (N\emph{,z,s}), e não
nos preocupar muito

com como ela é construída --- se quiser, pegue N = NvN, \emph{z} = 0vN =
∅ and \emph{s}(\emph{n}) = \emph{n}+=

\emph{n∪\{n\}} para todo \emph{n ∈} N.

As operações aritméticas que conhecemos e amamos --- adição,
multiplicação e ex-

ponenciação --- são definidas usando recursão em \textbf{??}; assim como
os fatoriais e os

coeficientes binomiais.

Podemos definir noções mais familiares envolvendo números naturais. Por
exemplo,

podemos definir a relação ⩽ em N definindo \emph{m} ⩽ \emph{n} como
significando \emph{∃k ∈} N\emph{, m}+\emph{k} = \emph{n} . Uma abordagem
alternativa seria definir `\emph{m} ⩽ \emph{n}' para \emph{m,n ∈} N por
uma recursão iterada em \emph{m} e \emph{n} --- especificamente:

• 0 ⩽ 0 é verdade;

• For all \emph{m ∈} N, \emph{m}+1 ⩽ 0 é falso;

• For all \emph{n ∈} N, 0 ⩽ \emph{n}+1 é verdade; e

• For all \emph{m,n ∈} N, \emph{m}+1 ⩽ \emph{n}+1 é verdade se e somente
se \emph{m} ⩽ \emph{n} é verdade.
\end{quote}

✎ \textbf{Exercise A.2.9}

\begin{quote}
Prove que a definição recursiva de ⩽ é equivalente à definição de
`\emph{m} ⩽ \emph{n}' como `\emph{∃k ∈}N\emph{, m}+\emph{k} = \emph{n}'.
◁

\textbf{Inteiros}

Agora que construímos os números naturais, é hora de construir os
inteiros. A intuição

por trás da definição que estamos prestes a dar é que um número inteiro
nos diz a

diferença entre dois números naturais. Assim, `3 como um número inteiro'
refere-se ao

aumento de 3 de um número natural para outro (digamos 0 a 3, ou 7 a 10);
e `\emph{−}4 como um número inteiro' refere-se a uma diminuição de 4 de
um número natural para outro

(digamos 4 para 0, ou 10 para 6).

Uma primeira tentativa pode ser definir Z como N \emph{×} N e
interpretar (\emph{a,b}) \emph{∈} N \emph{×} N para representar o
inteiro \emph{b − a}. Infelizmente, isso não funciona muito bem, pois,
por exemplo, o número inteiro 3 seria representado como (0\emph{,}3) e
como (7\emph{,}10)---e, mais

geralmente, como (\emph{n,n}+3) para todos \emph{n ∈} N.

Então, em vez disso, declararemos dois pares (\emph{a,b}), (\emph{c,d})
\emph{∈} N \emph{×} N para representar
\end{quote}

32

\begin{quote}
\emph{Section A.2. Construções dos conjuntos de números} 33

o mesmo número inteiro sempre que \emph{b − a} = \emph{d − c}. Para
tornar isso formal, algumas coisas precisam acontecer:

(1) Precisamos esclarecer o que queremos dizer com `declarar dois pares
para repres-

entar o mesmo número inteiro' --- isso será feito definindo uma relação
de equival-

ência em N\emph{×}N e depois passando para o quociente (veja
\textbf{??}).

(2) Não podemos usar subtração em nossa definição da relação de
equivalência, ou em

nossa prova de que \emph{é} uma relação de equivalência, pois ainda não
definimos uma

noção de subtração para N.

Fortunately for us, (2) is easy to resolve: we can state the equation
\emph{b − a} = \emph{d − c} for \emph{m,n ∈} N equivalently as
\emph{a}+\emph{d} = \emph{b}+\emph{c}; this is the equation that we will
use to define the equivalence relation on N\emph{×}N in (1).
\end{quote}

✣ \textbf{Lemma A.2.10}

\begin{quote}
A relação \emph{∼} em N \emph{×} N, definida para
(\emph{a,b})\emph{,}(\emph{c,d}) \emph{∈} N \emph{×} N por (\emph{a,b})
\emph{∼} (\emph{c,d}) se e somente se \emph{a}+\emph{d} =
\emph{b}+\emph{c}, é uma relação de equivalência.

\emph{\textbf{Proof}}
\end{quote}

• (\textbf{Reflexividade}) Seja (\emph{a,b}) \emph{∈} N\emph{×}N. Então
\emph{a}+\emph{b} = \emph{b}+\emph{a}, para que (\emph{a,b}) \emph{∼}
(\emph{a,b}).

\begin{quote}
• (\textbf{Simetria}) Seja (\emph{a,b})\emph{,}(\emph{c,d}) \emph{∈}
N\emph{×}Ne assumindo que (\emph{a,b}) \emph{∼} (\emph{c,d}). Então
\emph{a}+\emph{d} = \emph{b}+\emph{c}, e assim \emph{c}+\emph{b} =
\emph{d} +\emph{a}, (\emph{c,d}) \emph{∼} (\emph{a,b}).

• (\textbf{Transitividade}) Seja
(\emph{a,b})\emph{,}(\emph{c,d})\emph{,}(\emph{e, f}) \emph{∈} N
\emph{×} N e assuma que (\emph{a,b}) \emph{∼} (\emph{c,d}) e
(\emph{c,d}) \emph{∼} (\emph{e, f}). Então \emph{a}+\emph{d} =
\emph{b}+\emph{c} e \emph{c}+ \emph{f} = \emph{d} +\emph{e}. Mas então

(\emph{a}+ \emph{f})+(\emph{c}+\emph{d}) =
(\emph{a}+\emph{d})+(\emph{c}+ \emph{f}) = (\emph{b}+\emph{c})+(\emph{d}
+\emph{e}) = (\emph{b}+\emph{e})+(\emph{c}+\emph{d})

então cancelando \emph{c} + \emph{d} de ambos os lados dá \emph{a} +
\emph{f} = \emph{b} + \emph{e}, então (\emph{a,b}) \emph{∼} (\emph{e,
f}), como requerido

Devemos ter o cuidado de observar que não assumimos a existência de uma
operação

de subtração quando cancelamos \emph{c} + \emph{d} de ambos os lados da
equação: o fato de

\emph{u}+\emph{w} = \emph{v}+\emph{w} implica \emph{u} = \emph{v} pois
todo \emph{u,v,w ∈} N pode ser provado por indução em \emph{w} usando
apenas os axiomas de Peano. □

Agora que temos uma relação de equivalência em N\emph{×}N, podemos
considerar Z como o conjunto resultante de classes de equivalência (ou
seja, o quociente).
\end{quote}

33

\begin{longtable}[]{@{}
  >{\raggedright\arraybackslash}p{(\columnwidth - 4\tabcolsep) * \real{0.3333}}
  >{\raggedright\arraybackslash}p{(\columnwidth - 4\tabcolsep) * \real{0.3333}}
  >{\raggedright\arraybackslash}p{(\columnwidth - 4\tabcolsep) * \real{0.3333}}@{}}
\toprule()
\begin{minipage}[b]{\linewidth}\raggedright
\begin{quote}
34
\end{quote}
\end{minipage} &
\multicolumn{2}{>{\raggedright\arraybackslash}p{(\columnwidth - 4\tabcolsep) * \real{0.6667} + 2\tabcolsep}@{}}{%
\begin{minipage}[b]{\linewidth}\raggedright
\begin{quote}
\emph{Appendix A. Miscelânea matemática}
\end{quote}
\end{minipage}} \\
\midrule()
\endhead
\multicolumn{3}{@{}>{\raggedright\arraybackslash}p{(\columnwidth - 4\tabcolsep) * \real{1.0000} + 4\tabcolsep}@{}}{%
\begin{minipage}[t]{\linewidth}\raggedright
\begin{longtable}[]{@{}
  >{\raggedright\arraybackslash}p{(\columnwidth - 0\tabcolsep) * \real{1.0000}}@{}}
\toprule()
\begin{minipage}[b]{\linewidth}\raggedright
✦ \textbf{Construction A.2.11}\\
O \textbf{Conjunto dos inteiros} é o conjunto Z definido por

\begin{quote}
Z = (N\emph{×}N)\emph{/∼}\\
onde \emph{∼} é a relação de equivalência em N\emph{×}N definida por\\
(\emph{a,b}) \emph{∼} (\emph{c,d}) se e somente se \emph{a}+\emph{d} =
\emph{b}+\emph{c}
\end{quote}\strut
\end{minipage} \\
\midrule()
\endhead
\bottomrule()
\end{longtable}

\begin{quote}
for all (\emph{a,b})\emph{,}(\emph{c,d}) \emph{∈} N\emph{×}N.

Interpretamos o elemento {[}(\emph{a,b}){]}\emph{∼ ∈} Z como o número
inteiro \emph{b−a}. Então por exemplo temos
\end{quote}\strut
\end{minipage}} \\
3 = {[}(0\emph{,}3){]}\emph{∼} = {[}(7\emph{,}10){]}\emph{∼} & e &
\begin{minipage}[t]{\linewidth}\raggedright
\begin{quote}
\emph{−}4 = {[}(4\emph{,}0){]}\emph{∼} = {[}(10\emph{,}6){]}\emph{∼}
\end{quote}
\end{minipage} \\
\bottomrule()
\end{longtable}

\begin{quote}
Algumas observações são agora necessárias.

A primeira observação é que podemos definir operações de adição e
multiplicação em Z usando aquelas em N. Formalmente, definimos:

{[}(\emph{a,b}){]}\emph{∼} +{[}(\emph{c,d}){]}\emph{∼} =
{[}(\emph{a}+\emph{c,b}+\emph{d}){]}\emph{∼}---o a intuição aqui é que
para todo \emph{a,b,c,d ∈} N temos (\emph{b−a})+(\emph{d −c}) =
(\emph{b}+\emph{d})\emph{−}(\emph{a}+\emph{c}).

{[}(\emph{a,b}){]}\emph{∼·}{[}(\emph{c,d}){]}\emph{∼} =
{[}(\emph{ad}+\emph{bc,ac}+\emph{bd}){]}\emph{∼}--- a intuição aqui é
que para todo \emph{a,b,c,d ∈} N temos (\emph{b−a})(\emph{d −c}) =
(\emph{ac}+\emph{bd})\emph{−}(\emph{ad} +\emph{bc}). Como as operações +
e \emph{·} são definidas em termos de representantes de classes de
equivalência, devemos (mas não iremos) verificar se essas operações
estão bem definidas --- por exemplo, precisamos verificar se
{[}(\emph{a,b}){]}\emph{∼} = {[}(\emph{a′,b′}){]}\emph{∼} e
{[}(\emph{c,d}){]}\emph{∼} = {[}(\emph{c′,d′}){]}\emph{∼}, então
{[}(\emph{a}+\emph{c,b}+\emph{d}){]}\emph{∼} =
{[}(\emph{a′}+\emph{c′,b′}+ \emph{d′}){]}\emph{∼}.

O que torna os números inteiros especiais é que também podemos negá-los,
o que por sua vez nos permite subtraí-los. Com isso em mente, podemos
definir operações de negação e subtração em Z da seguinte forma:

\emph{−}{[}(\emph{a,b}){]}\emph{∼} = {[}(\emph{b,a}){]}\emph{∼}---a
intuição aqui é que para todo \emph{a,b ∈} N temos \emph{−}(\emph{b−a})
= \emph{a−b}.

{[}(\emph{a,b}){]}\emph{∼ −}{[}(\emph{c,d}){]}\emph{∼} =
{[}(\emph{a}+\emph{d,b}+\emph{c}){]}\emph{∼}---o a intuição aqui é que
para todo \emph{a,b,c,d ∈} N temos (\emph{b − a}) \emph{−} (\emph{d −
c}) = (\emph{b} + \emph{c}) \emph{−} (\emph{a} + \emph{d}). podem ser
definidas uma em relação à outra: elas estão relacionadas pelas
identidades As operações de negação e subtração

\emph{−}{[}(\emph{a,b}){]}\emph{∼} = {[}(0\emph{,}0){]}\emph{∼
−}{[}(\emph{a,b}){]}\emph{∼} e {[}(\emph{a,b}){]}\emph{∼
−}{[}(\emph{c,d}){]}\emph{∼} = {[}(\emph{a,b}){]}\emph{∼}
+(\emph{−}{[}(\emph{c,d}){]}\emph{∼}) Novamente, devemos, mas não
iremos, provar que as operações de negação e subtração estão bem
definidas.
\end{quote}

34

\begin{quote}
\emph{Section A.2. Construções dos conjuntos de números} 35

A segunda e talvez mais alarmante observação é que esta construção de Z
significa

que não temos N \emph{⊆} Z: os elementos de Z são classes de
equivalência de pares de números naturais, o que significa que o número
natural `3' não é igual ao inteiro `3'

(que é realmente igual a {[}(0\emph{,}3){]}\emph{∼}). No entanto,
consideraremos N como sendo um subconjunto de Z identificando cada
número natural \emph{n} com o elemento {[}(0\emph{,n}){]}\emph{∼ ∈} Z. O
que justifica esta identificação é o seguinte exercício.
\end{quote}

✎ \textbf{Exercise A.2.12}

\begin{quote}
Prove que cada elemento de Z, conforme definido em Construction A.2.11,
é igual a

exatamente um dos seguintes: {[}(0\emph{,}0){]}\emph{∼}, ou
{[}(0\emph{,n}){]}\emph{∼} para alguns \emph{n \textgreater{}} 0, ou
{[}(\emph{n,}0){]}\emph{∼} para alguns \emph{n \textgreater{}} 0. ◁

Exercise A.2.12 implica que a função de `inclusão' \emph{i} : N \emph{→}
Z definida por \emph{i}(\emph{n}) = {[}(0\emph{,n}){]}\emph{∼} para todo
\emph{n ∈} N é injetivo. Quando \emph{n ∈} N, normalmente escreveremos
apenas`\emph{n}' para denotar o inteiro \emph{i}(\emph{n}) =
{[}(0\emph{,n}){]}\emph{∼}.

Observe que \emph{i}(\emph{m}+\emph{n}) =
\emph{i}(\emph{m})+\emph{i}(\emph{n}) e \emph{i}(\emph{mn}) =
\emph{i}(\emph{m})\emph{i}(\emph{n}), então quando escrevemos algo

como `\emph{m} + \emph{n}' para \emph{m,n ∈} N, não importa se estamos
primeiro adicionando \emph{m} e \emph{n} como números naturais e depois
interpretando o resultado como um número inteiro,

ou primeiro interpretando \emph{m} e \emph{n} como inteiros e depois
adicionando o resultado usando

a operação de adição para inteiros.

Observe também que a identificação de \emph{n ∈} N com
\emph{i}(\emph{n}) = {[}(0\emph{,n}){]}\emph{∼ ∈} Z dá \emph{−n} =
{[}(\emph{n,}0){]}\emph{∼}para todos \emph{n ∈} N, e então Exercise
A.2.12 implica que todo número inteiro é igual a exatamente um de 0,
\emph{n} ou \emph{−n} para algum número natural positivo \emph{n}.

O próximo resultado prova que Z é um \emph{anel}---o que isso significa
em essência é que as

operações de adição, multiplicação e negação satisfazem as propriedades
básicas que

tomamos como certas ao fazer aritmética com inteiros.
\end{quote}

\begin{longtable}[]{@{}
  >{\raggedright\arraybackslash}p{(\columnwidth - 0\tabcolsep) * \real{1.0000}}@{}}
\toprule()
\begin{minipage}[b]{\linewidth}\raggedright
\begin{longtable}[]{@{}
  >{\raggedright\arraybackslash}p{(\columnwidth - 0\tabcolsep) * \real{1.0000}}@{}}
\toprule()
\begin{minipage}[b]{\linewidth}\raggedright
✣ \textbf{Theorem A.2.13} (Z é um anel)

\begin{quote}
(a) (\textbf{Associatividade de adição}) (\emph{a}+\emph{b})+\emph{c} =
\emph{a}+(\emph{b}+\emph{c}) para todo \emph{a,b,c ∈} Z. (b)
(\textbf{Comutatividade de adição}) \emph{a}+\emph{b} =
\emph{b}+\emph{a} para todo \emph{a,b ∈} Z.

(c) (\textbf{Zero}) \emph{a}+0 = \emph{a} para todo \emph{a ∈} Z.

(d) (\textbf{Negação}) \emph{a}+(\emph{−a}) = 0 para todo \emph{a ∈} Z.

(e) (\textbf{Associatividade de multiplicação}) (\emph{a·b})\emph{·c} =
\emph{a·}(\emph{b·c}) para todo \emph{a,b,c ∈} Z. (f)
(\textbf{Comutatividade da multiplicação}) \emph{a·b} = \emph{b·a} para
todo \emph{a,b ∈} Z.

(g) (\textbf{Um}) \emph{a·}1 = \emph{a} para todo \emph{a ∈} Z.
\end{quote}
\end{minipage} \\
\midrule()
\endhead
\bottomrule()
\end{longtable}

\begin{quote}
(h) (\textbf{Distributividade}) \emph{a·}(\emph{b}+\emph{c}) =
(\emph{a·b})+(\emph{a·c}) para todo \emph{a,b,c ∈} Z.
\end{quote}
\end{minipage} \\
\midrule()
\endhead
\bottomrule()
\end{longtable}

35

36 \emph{Appendix A. Miscelânea matemática}

\emph{\textbf{Proof}}\\
Muitas páginas seriam desperdiçadas escrevendo a prova deste teorema em
todos os detalhes. Em vez disso, provaremos apenas a parte (h); qualquer
leitor particularmente masoquista é convidado a provar as partes
(a)--(g) por conta própria.

Para ver se a lei da distributividade é válida, seja \emph{a,b,c ∈} Z, e
seja \emph{m,n, p,q,r,s ∈} N tal que \emph{a} =
{[}(\emph{m,n}){]}\emph{∼}, \emph{b} = {[}(\emph{p,q}){]}\emph{∼} e
\emph{c} = {[}(\emph{r,s}){]}\emph{∼}. (Omitiremos o subscrito \emph{∼}
a seguir.)

Então:

\begin{quote}
\emph{a·}(\emph{b}+\emph{c})\\
= {[}(\emph{m,n}){]}\emph{·}({[}(\emph{p,q}){]}+{[}(\emph{r,s}){]})\\
= {[}(\emph{m,n}){]}\emph{·}{[}(\emph{p}+\emph{r,q}+\emph{s}){]}\\
=
{[}(\emph{m}(\emph{q}+\emph{s})+\emph{n}(\emph{p}+\emph{r})\emph{,m}(\emph{p}+\emph{r})+\emph{n}(\emph{q}+\emph{s})){]}

= {[}(\emph{mq}+\emph{ms}+\emph{np}+\emph{nr,mp}+\emph{mr}
+\emph{nq}+\emph{ns}){]}
\end{quote}

= {[}(\emph{mq}+\emph{np,mp}+\emph{nq}){]}+{[}(\emph{ms}+\emph{nr,mr}
+\emph{ns}){]}

\begin{longtable}[]{@{}
  >{\raggedright\arraybackslash}p{(\columnwidth - 4\tabcolsep) * \real{0.3333}}
  >{\raggedright\arraybackslash}p{(\columnwidth - 4\tabcolsep) * \real{0.3333}}
  >{\raggedright\arraybackslash}p{(\columnwidth - 4\tabcolsep) * \real{0.3333}}@{}}
\toprule()
\begin{minipage}[b]{\linewidth}\raggedright
como requerido.
\end{minipage} & \begin{minipage}[b]{\linewidth}\raggedright
\begin{quote}
=
{[}(\emph{m,n}){]}\emph{·}{[}(\emph{p,q}){]}+{[}(\emph{m,n}){]}\emph{·}{[}(\emph{r,s}){]}
= (\emph{a·b})+(\emph{a·c})
\end{quote}
\end{minipage} & \begin{minipage}[b]{\linewidth}\raggedright
□
\end{minipage} \\
\midrule()
\endhead
\bottomrule()
\end{longtable}

\textbf{Números racionais}

Assim como construímos os números inteiros a partir dos números
naturais, formal-izando o que entendemos por ``diferença'' de dois
números naturais, construiremos os números racionais a partir dos
inteiros, formalizando o que entendemos por ``pro-porção'' de dois
inteiros.

Como os números racionais devem assumir a forma\emph{a b}com \emph{a,b
∈} Z e \emph{b ̸}= 0, uma primeira tentativa de construir Q pode ser
pegar Q = Z \emph{×} (Z \emph{\textbackslash{} \{}0\emph{\}}), e deixar
o par (\emph{a,b}) representar a fração \emph{d fracab}. Mas assim como
um número inteiro não pode ser expresso exclusivamente como a diferença
de dois números naturais, um número racional não pode ser expresso
exclusivamente como a razão de dois inteiros --- por exemplo, temos 1 2=
2 4.

Isso significa que devemos identificar (\emph{a,b}) \emph{∈}
Z\emph{×}(Z\emph{\textbackslash\{}0\emph{\}}) com (\emph{c,d}) \emph{∈}
Z\emph{×}(Z\emph{\textbackslash\{}0\emph{\}}) sempre que\emph{a b}=
\emph{c d}. Mas espere! Ainda não definimos uma operação de divisão,
port-anto não podemos usá-la em nossa construção de Q---então, em vez
disso, identifi-caremos (\emph{a,b}) com (\emph{c,d}) sempre que
\emph{ad} = \emph{bc}, observando que isso será equivalente a \emph{b}=
\emph{c d}depois de definirmos uma operação de divisão.

36

\begin{quote}
\emph{Section A.2. Construções dos conjuntos de números} 37

✣ \textbf{Lemma A.2.14}\\
A relação \emph{∼} em Z \emph{×} (Z \emph{\textbackslash{}
\{}0\emph{\}}), definida para (\emph{a,b})\emph{,}(\emph{c,d}) \emph{∈}
Z \emph{×} (Z \emph{\textbackslash{} \{}0\emph{\}}) deixando
(\emph{a,b}) \emph{∼} (\emph{c,d}) se e somente se \emph{ad} =
\emph{bc}, é uma relação de equivalência.

\emph{\textbf{Sketch of proof}}
\end{quote}

\begin{longtable}[]{@{}
  >{\raggedright\arraybackslash}p{(\columnwidth - 8\tabcolsep) * \real{0.2000}}
  >{\raggedright\arraybackslash}p{(\columnwidth - 8\tabcolsep) * \real{0.2000}}
  >{\raggedright\arraybackslash}p{(\columnwidth - 8\tabcolsep) * \real{0.2000}}
  >{\raggedright\arraybackslash}p{(\columnwidth - 8\tabcolsep) * \real{0.2000}}
  >{\raggedright\arraybackslash}p{(\columnwidth - 8\tabcolsep) * \real{0.2000}}@{}}
\toprule()
\multicolumn{5}{@{}>{\raggedright\arraybackslash}p{(\columnwidth - 8\tabcolsep) * \real{1.0000} + 8\tabcolsep}@{}}{%
\begin{minipage}[b]{\linewidth}\raggedright
\begin{quote}
A prova é essencialmente a mesma de Lemma A.2.10, mas com adição
substituída por

multiplicação; a prova da transitividade usa o fato de que para
\emph{u,v,w ∈} Z com \emph{u ̸}= 0, temos \emph{uv} = \emph{uw ⇒ v} =
\emph{w}, o que pode ser provado por indução em
\emph{\textbar voc\textbar{}} ⩾ 1 sem usar uma operação de divisão. □
\end{quote}

\begin{longtable}[]{@{}
  >{\raggedright\arraybackslash}p{(\columnwidth - 0\tabcolsep) * \real{1.0000}}@{}}
\toprule()
\begin{minipage}[b]{\linewidth}\raggedright
✦ \textbf{Construction A.2.15}\\
O \textbf{conjunto de números racionais} é o conjunto Q definido por

Q = (Z\emph{×}(Z\emph{\textbackslash\{}0\emph{\}}))\emph{/∼}

\begin{quote}
onde \emph{∼} é a relação de equivalência em
Z\emph{×}(Z\emph{\textbackslash\{}0\emph{\}}) definida por
\end{quote}

(\emph{a,b}) \emph{∼} (\emph{c,d}) se e somente se \emph{ad} =
\emph{bc}\strut
\end{minipage} \\
\midrule()
\endhead
\bottomrule()
\end{longtable}

\begin{quote}
para todos (\emph{a,b})\emph{,}(\emph{c,d}) \emph{∈}
Z\emph{×}(Z\emph{\textbackslash\{}0\emph{\}}).

Escreveremos `\emph{a b}' para denotar o elemento
{[}(\emph{a,b}){]}\emph{∼ ∈} Q, observando que \emph{a b}= \emph{c d}se

e somente se \emph{ad} = \emph{bc}. Agora podemos definir operações de
adição, multiplicação,

negação e subtração em Q em termos daquelas em Z---ou seja, dado
\emph{a,b,c,d ∈} Z com \emph{b,d ̸}= 0, definimos
\end{quote}\strut
\end{minipage}} \\
\midrule()
\endhead
\emph{b}+ \emph{c d}= \emph{ad} +\emph{bc} &
\begin{minipage}[t]{\linewidth}\raggedright
\begin{quote}
\emph{,}
\end{quote}
\end{minipage} & \emph{b· c d}= \emph{ac bd,} & \emph{−a b}=
\emph{\sout{−}a \sout{b},} & \begin{minipage}[t]{\linewidth}\raggedright
\begin{quote}
\emph{b− c d}= \emph{ad −bc}
\end{quote}
\end{minipage} \\
\bottomrule()
\end{longtable}

\begin{quote}
Como sempre, devemos (mas não iremos) verificar se essas operações estão
bem defin-idas.

Novamente, não temos Z \emph{⊆} Q, mas podemos incorporar Z em Q de uma
forma que respeite as operações aritméticas de adição , multiplicação,
negação e subtração.
\end{quote}

✎ \textbf{Exercise A.2.16}\\
Prove que a função \emph{i} : Z \emph{→} Q definida por
\emph{i}(\emph{a}) =\emph{a} \sout{1} é uma injeção, e para todo
\emph{a,b ∈} Z temos \emph{i}(\emph{a}+\emph{b}) =
\emph{i}(\emph{a})+\emph{i}(\emph{b}), \emph{i}(\emph{ab}) =
\emph{i}(\emph{a})\emph{i}(\emph{b}), \emph{i}(\emph{−a}) =
\emph{−i}(\emph{a}) e \emph{i}(\emph{ab}) =
\emph{i}(\emph{a})\emph{−i}(\emph{b}). ◁

\begin{quote}
À luz disso, dado \emph{a ∈} Z, normalmente escreveremos apenas
`\emph{a}' para o número racional \emph{i}(\emph{a}) =\emph{a} o caso.
1\emph{∈} Q, então podemos fingir que Z \emph{⊆} Q mesmo que este não
seja estritamente
\end{quote}

37

\begin{quote}
38 \emph{Appendix A. Miscelânea matemática}

A construção de Q também permite definir operações recíprocas e de
divisão. Observe

que\emph{c d}= 0 se e somente se \emph{c} = \emph{c·}1 = 0\emph{·d} = 0;
então se \emph{a,b,c,d ∈} Q com \emph{b,c,d ̸}= 0,

então definimos
\end{quote}

� \emph{c}�\emph{−}1 =\emph{d}

\begin{quote}
\emph{c} \textbf{e} \emph{b÷ fraccd} = \emph{ad}

O comentário usual sobre bem-definição se aplica. Observe que para todo
\emph{a,b ∈} Z com

\emph{b ̸}= 0 temos
\end{quote}

\emph{a÷b} =\emph{a} 1\emph{÷ b} 1= \emph{a·}1 1\emph{·b}= \emph{a}

\begin{quote}
Portanto, dispensaremos o símbolo `\emph{÷}' e simplesmente usaremos a
notação de fração.

Observe também que as operações recíprocas e de divisão podem ser
definidas uma em

função da outra: elas são relacionadas pelas identidades

{]} y\emph{−}1=1

\emph{y}\\
\textbf{e}\\
\emph{x}

\emph{y}= \emph{x·y−}1 para todo \emph{x,y ∈} Q com \emph{y ̸}= 0.

Q é um campo

\textbf{Números Reais}
\end{quote}

\begin{longtable}[]{@{}
  >{\raggedright\arraybackslash}p{(\columnwidth - 0\tabcolsep) * \real{1.0000}}@{}}
\toprule()
\begin{minipage}[b]{\linewidth}\raggedright
\begin{longtable}[]{@{}
  >{\raggedright\arraybackslash}p{(\columnwidth - 0\tabcolsep) * \real{1.0000}}@{}}
\toprule()
\begin{minipage}[b]{\linewidth}\raggedright
✦ \textbf{Definition A.2.17} (A construção dos números reais por
Dedekind) O \textbf{conjunto de} (\textbf{Dedekind}) \textbf{números
reais} é o conjunto R definido por
\end{minipage} \\
\midrule()
\endhead
\bottomrule()
\end{longtable}

\begin{quote}
R = \emph{\{D ⊆} Q \emph{\textbar{} D} é limitado acima e fechado para
baixo\emph{\}}
\end{quote}
\end{minipage} \\
\midrule()
\endhead
\bottomrule()
\end{longtable}

\begin{quote}
\textbf{To do:} Operações aritméticas, ordem

\textbf{To do:} Motivar Cauchy reais
\end{quote}

\begin{longtable}[]{@{}
  >{\raggedright\arraybackslash}p{(\columnwidth - 0\tabcolsep) * \real{1.0000}}@{}}
\toprule()
\begin{minipage}[b]{\linewidth}\raggedright
✦ \textbf{Definition A.2.18} (A construção dos números reais por Cauchy)
O \textbf{conjunto de} (\textbf{Cauchy}) \textbf{números reais} é o
conjunto R definido por

\begin{quote}
R = \emph{\{}(\emph{xn}) \emph{∈} QN\emph{\textbar{}} (\emph{xn}) é
Cauchy\emph{\}/∼}onde \emph{∼} é a relação de equivalência definida
por\\
(\emph{xn}) \emph{∼} (\emph{yn}) se e somente se (\emph{xn −yn})
\emph{→} 0 para todas as sequências de Cauchy
(\emph{xn})\emph{,}(\emph{yn}) de números racionais.
\end{quote}\strut
\end{minipage} \\
\midrule()
\endhead
\bottomrule()
\end{longtable}

38

\begin{quote}
\emph{Section A.2. Construções dos conjuntos de números} 39

\textbf{To do:} Operações aritméticas, ordem

\textbf{To do:} Motivar a definição de números complexos
\end{quote}

\begin{longtable}[]{@{}
  >{\raggedright\arraybackslash}p{(\columnwidth - 0\tabcolsep) * \real{1.0000}}@{}}
\toprule()
\begin{minipage}[b]{\linewidth}\raggedright
\begin{longtable}[]{@{}
  >{\raggedright\arraybackslash}p{(\columnwidth - 0\tabcolsep) * \real{1.0000}}@{}}
\toprule()
\begin{minipage}[b]{\linewidth}\raggedright
✦ \textbf{Definition A.2.19}
\end{minipage} \\
\midrule()
\endhead
\bottomrule()
\end{longtable}

\begin{quote}
O \textbf{conjunto dos números complexos} é o conjunto C = R\emph{×}R.
\end{quote}
\end{minipage} \\
\midrule()
\endhead
\bottomrule()
\end{longtable}

\begin{quote}
\textbf{To do:} Operações aritimeticas

\textbf{Estruturas algébricas}\\
\textbf{To do:} Monóides, grupos, anéis

\textbf{Axiomatizando os números reais}\\
\textbf{To do:}
\end{quote}

39

\begin{longtable}[]{@{}
  >{\raggedright\arraybackslash}p{(\columnwidth - 2\tabcolsep) * \real{0.5000}}
  >{\raggedright\arraybackslash}p{(\columnwidth - 2\tabcolsep) * \real{0.5000}}@{}}
\toprule()
\begin{minipage}[b]{\linewidth}\raggedright
\begin{quote}
40
\end{quote}
\end{minipage} & \begin{minipage}[b]{\linewidth}\raggedright
\emph{Appendix A. Miscelânea matemática}
\end{minipage} \\
\midrule()
\endhead
\multicolumn{2}{@{}>{\raggedright\arraybackslash}p{(\columnwidth - 2\tabcolsep) * \real{1.0000} + 2\tabcolsep}@{}}{%
\begin{minipage}[t]{\linewidth}\raggedright
\begin{longtable}[]{@{}
  >{\raggedright\arraybackslash}p{(\columnwidth - 0\tabcolsep) * \real{1.0000}}@{}}
\toprule()
\begin{minipage}[b]{\linewidth}\raggedright
✣ \textbf{Axioms A.2.20} (Axiomas de campo)

Seja \emph{X} um conjunto equipado com elementos 0 (`zero') e 1
(`unidade'), e operações

binárias + (`adição') e \emph{·} (`multiplicação'). A estrutura
(\emph{X,}0\emph{,}1\emph{,}+\emph{,·}) é um \textbf{campo} se
satisfizer os seguintes axiomas:

\begin{quote}
• \textbf{Zero e unidade}

(F1) 0 \emph{̸}= 1.

• \textbf{Axiomas para adição}
\end{quote}

(F2) (Associatividade) \emph{x}+(\emph{y}+\emph{z}) =
(\emph{x}+\emph{y})+\emph{z} para todo \emph{x,y,z ∈ X}. (F3)
(Identidade) \emph{x}+0 = \emph{x} para todo \emph{x ∈ X}.

(F4) (Inverso) Para todo \emph{x ∈ X}, onde existe \emph{y ∈ X} tanto
que \emph{x}+\emph{y} = 0. (F5) (Comutatividade) \emph{x}+\emph{y} =
\emph{y}+\emph{x} para todo \emph{x,y ∈ X}.

\begin{quote}
• \textbf{Axiomas por multiplicação}

(F6) (Associatividade) \emph{x·}(\emph{y·z}) = (\emph{x·y})\emph{·z}
para todo \emph{x,y,z ∈ X}.

(F7) (Identidade) \emph{x·}1 = \emph{x} para todo \emph{x ∈ X}.

(F8) (Inverso) Para todo \emph{x ∈ X} com \emph{x ̸}= 0, onde existe
\emph{y ∈ X} tanto que \emph{x·y} = 1. (F9) (Comutatividade) \emph{x·y}
= \emph{y·x} para todo \emph{x,y ∈ X}.

• \textbf{Distributividade}
\end{quote}
\end{minipage} \\
\midrule()
\endhead
\bottomrule()
\end{longtable}

\begin{quote}
(F10) \emph{x·}(\emph{y}+\emph{z}) = (\emph{x·y})+(\emph{x·z}) para todo
\emph{x,y,z ∈ X}.
\end{quote}
\end{minipage}} \\
\bottomrule()
\end{longtable}

✐ \textbf{Example A.2.21}\\
Os racionais Q e os reais R formam campos com suas noções usuais de
zero, unidade, adição e multiplicação. No entanto, os inteiros Z não, já
que por exemplo 2 não tem inverso multiplicativo. ◁

✐ \textbf{Example A.2.22}\\
Seja \emph{p \textgreater{}} 0 primo. O conjunto Z\emph{/p}Z (ver
\textbf{??}) é um campo, com elemento zero {[}0{]}\emph{p} e elemento
unitário {[}1{]}\emph{p}, e com adição e multiplicação definida por

\begin{quote}
{[}\emph{a}{]}\emph{p} +{[}\emph{b}{]}\emph{p} =
{[}\emph{a}+\emph{b}{]}\emph{p} e {[}\emph{a}{]}\emph{p
·}{[}\emph{b}{]}\emph{p} = {[}\emph{ab}{]}\emph{p}\\
para todo \emph{a,b ∈} Z. A boa definição dessas operações é imediata
desde \textbf{??} e o teorema da aritmética modular (\textbf{??}).

O único axioma que não é fácil de verificar é o axioma do inverso
multiplicativo (F8). Na verdade, se {[}\emph{a}{]}\emph{p ∈} Z\emph{/p}Z
então {[}\emph{a}{]}\emph{p ̸}= {[}0{]}\emph{p} se e somente se \emph{p}
∤ \emph{a}. Mas se \emph{p} ∤ \emph{a} então \emph{a ⊥ p}, então
\emph{a} tem um inverso multiplicativo \emph{u} módulo \emph{p}. Isso
implica que {[}\emph{a}{]}\emph{p ·}{[}\emph{u}{]}\emph{p} =
{[}\emph{au}{]}\emph{p} = {[}1{]}\emph{p}. Então (F8) é válido. ◁
\end{quote}

40

\begin{quote}
\emph{Section A.2. Construções dos conjuntos de números} 41
\end{quote}

✎ \textbf{Exercise A.2.23}\\
Seja \emph{n \textgreater{}} 0 composto. Prove que Z\emph{/n}Z não é um
corpo, onde zero, unidade, adição e multiplicação são definidos como em
Example A.2.22. ◁

\begin{quote}
Axioms A.2.20 diga-nos que todo elemento de um corpo tem um inverso
aditivo, e todo elemento \emph{diferente de zero} de um corpo tem um
inverso multiplicativo. Seria conveniente se os inversos fossem
\emph{{[}úicos} sempre que existissem. Proposition A.2.24 prova que este
é o caso.
\end{quote}

✣ \textbf{Proposition A.2.24} (Singularidade dos inversos) Seja
(\emph{X,}0\emph{,}1\emph{,}+\emph{,·}) e um campo e seja \emph{x ∈ X}.
Então

\begin{quote}
(a) Suponha que \emph{y,z ∈ X} sejam tais que \emph{x}+\emph{y} = 0 e
\emph{x}+\emph{z} = 0. Então \emph{y} = \emph{z}. (b) Suponha que
\emph{x ̸}= 0 e \emph{y,z ∈ X} sejam tais que \emph{x·y} = 1 e
\emph{x·z} = 1. Então \emph{y} = \emph{z}.

\emph{\textbf{Proof} de (a)}\\
Por cálculo, temos

\emph{y} = \emph{y}+0 por (F3)

= \emph{y}+(\emph{x}+\emph{z}) por definição de \emph{z}

= (\emph{y}+\emph{x})+\emph{z} por associatividade (F2)

= (\emph{x}+\emph{y})+\emph{z} by commutativity (F5)

= 0+\emph{z} por definição de \emph{y}

= \emph{z}+0 por comutatividade (F5)

= \emph{z} by (F3)

então de fato \emph{y} = \emph{z}.

A prova de (b) é essencialmente a mesma e fica como exercício. □

Como os inversos são únicos, faz sentido ter uma notação para se referir
a eles.

✦ \textbf{Notation A.2.25}\\
Seja (\emph{X,}0\emph{,}1\emph{,}+\emph{,·}) um campo e seja \emph{x ∈
X}. Escreva \emph{−x} para o inverso aditivo (único) de \emph{x} e, se
\emph{x ̸}= 0 escreva \emph{x−}1para o inverso multiplicativo (único) de
\emph{x}.
\end{quote}

✐ \textbf{Example A.2.26}\\
Nos campos Q e R, o inverso aditivo \emph{−x} de um elemento \emph{x} é
simplesmente seu negativo, e o inverso multiplicativo \emph{x−}1de algum
\emph{x ̸}= 0 é simplesmente seu recíproco\uline{1 \emph{x}}. ◁

✐ \textbf{Example A.2.27}\\
Seja \emph{p \textgreater{}} 0 primo e seja {[}\emph{a}{]}\emph{p ∈}
Z\emph{/p}Z. Então \emph{−}{[}\emph{a}{]}\emph{p} =
{[}\emph{−a}{]}\emph{p} e, se \emph{p} ∤ \emph{a}, então
{[}\emph{a}{]}\emph{−}1 onde \emph{u} é qualquer número inteiro
satisfatório \emph{au ≡} 1 mod \emph{p}. \emph{p}=
{[}\emph{u}{]}\emph{p}, ◁

✎ \textbf{Exercise A.2.28}

41

\begin{quote}
42 \emph{Appendix A. Miscelânea matemática}

Seja (\emph{X,}0\emph{,}1\emph{,}+\emph{,·}) um campo. Prove que
\emph{−}(\emph{−x}) = \emph{x} para todo \emph{x ∈ X}, e que
(\emph{x−}1)\emph{−}1= \emph{x} para todo \emph{x ∈ X} diferente de
zero. ◁
\end{quote}

✐ \textbf{Example A.2.29}

\begin{quote}
Seja (\emph{X,}0\emph{,}1\emph{,}+\emph{,·}) um campo. Provamos que se
\emph{x ∈ X} então \emph{x ·} 0 = 0. Bem, 0 = 0 + 0 por (F3). Portanto,
\emph{x·}0 = \emph{x·}(0+0). Pela distributividade (F10), temos
\emph{x·}(0+0) = (\emph{x·}0)+(\emph{x·}0). Por isso\\
\emph{x·}0 = (\emph{x·}0)+(\emph{x·}0)

Seja \emph{y} = \emph{−}(\emph{x·}0). Então

0 = \emph{x·}0+\emph{y}\\
= ((\emph{x·}0)+(\emph{x·}0))+\emph{y}\\
= (\emph{x·}0)+((\emph{x·}0)+\emph{y})\\
= (\emph{x·}0)+0\\
= \emph{x·}0

by (F4)\\
como acima\\
by associativity (F2)\\
by (F4)\\
by (F3)

então de fato nós temos \emph{x·}0 = 0. ◁
\end{quote}

✎ \textbf{Exercise A.2.30}\\
Seja (\emph{X,}0\emph{,}1\emph{,}+\emph{,·}) um campo. Prove que
(\emph{−}1)\emph{·x} = \emph{−x} para todo \emph{x ∈ X}, e que
(\emph{−x})\emph{−}1= \emph{−}(\emph{x−}1) para todo \emph{di
ferentedezeroxX}. ◁

\begin{quote}
O que torna os números reais úteis não é simplesmente a nossa capacidade
de adicioná-los, subtraí-los, multiplicá-los e dividi-los; também
podemos comparar seu tamanho ---na verdade, é isso que dá origem à noção
informal de \emph{reta numérica}. Axioms A.2.31 esclarece exatamente o
que significa os elementos de um campo serem montados em uma `reta
numérica'.
\end{quote}

42

\begin{longtable}[]{@{}
  >{\raggedright\arraybackslash}p{(\columnwidth - 2\tabcolsep) * \real{0.5000}}
  >{\raggedright\arraybackslash}p{(\columnwidth - 2\tabcolsep) * \real{0.5000}}@{}}
\toprule()
\begin{minipage}[b]{\linewidth}\raggedright
\begin{quote}
\emph{Section A.2. Construções dos conjuntos de números}
\end{quote}
\end{minipage} & \begin{minipage}[b]{\linewidth}\raggedright
43
\end{minipage} \\
\midrule()
\endhead
\multicolumn{2}{@{}>{\raggedright\arraybackslash}p{(\columnwidth - 2\tabcolsep) * \real{1.0000} + 2\tabcolsep}@{}}{%
\begin{minipage}[t]{\linewidth}\raggedright
\begin{longtable}[]{@{}
  >{\raggedright\arraybackslash}p{(\columnwidth - 0\tabcolsep) * \real{1.0000}}@{}}
\toprule()
\begin{minipage}[b]{\linewidth}\raggedright
✣ \textbf{Axioms A.2.31} (Axiomas de campo ordenados)

Seja \emph{X} um conjunto, 0\emph{,}1 \emph{∈ X} sejam elementos,
+\emph{,·} sejam operações binárias e ⩽ seja uma relação em \emph{X}. A
estrutura (\emph{X,}0\emph{,}1\emph{,}+\emph{,·,}⩽) é um \textbf{campo
ordenado} se satisfaz os axiomas de campo (F1)--(F10) (veja \textbf{??})
e, adicionalmente, satisfaz os seguintes

\begin{quote}
axiomas:

• \textbf{Axiomas de ordem linear}

(PO1) (Reflexitividade) \emph{x} ⩽ \emph{x} para todo \emph{x ∈ X}.

(PO2) (Antissimétria) Para todo \emph{x,y ∈ X}, se \emph{x} ⩽ \emph{y} e
\emph{y} ⩽ \emph{x}, então \emph{x} = \emph{y}. (PO3) (Transitividade)
Para todo \emph{x,y,z ∈ X}, se \emph{x} ⩽ \emph{y} e \emph{y} ⩽
\emph{z}, então \emph{x} ⩽ \emph{z}. (PO4) (Linearidade) Para todo
\emph{x,y ∈ X}, qualquer \emph{x} ⩽ \emph{y} ou \emph{y} ⩽ \emph{x}.

• \textbf{Interação de ordem com aritmética}

(OF1) Para todo \emph{x,y,z ∈ X}, se \emph{x} ⩽ \emph{y}, então
\emph{x}+\emph{z} ⩽ \emph{y}+\emph{z}.
\end{quote}
\end{minipage} \\
\midrule()
\endhead
\bottomrule()
\end{longtable}

\begin{quote}
(OF2) Para todo \emph{x,y ∈ X}, se 0 ⩽ \emph{x} e 0 ⩽ \emph{y}, então 0
⩽ \emph{xy}.
\end{quote}
\end{minipage}} \\
\bottomrule()
\end{longtable}

✐ \textbf{Example A.2.32}

\begin{quote}
O corpo Q dos números racionais e o campo R dos números reais, com suas
noções usuais de ordenação, podem ser facilmente vistos como campos
ordenados. ◁
\end{quote}

✐ \textbf{Example A.2.33}

\begin{quote}
Provamos que, em qualquer corpo ordenado, temos 0 ⩽ 1. Observe primeiro
que 0 ⩽ 1 ou 1 ⩽ 0 por linearidade (PO4). Se 0 ⩽ 1 então terminamos,
então suponha 1 ⩽ 0.

Então 0 ⩽ \emph{−}1; de fato:

0 = 1+(\emph{−}1) por (F4)

⩽ 0+(\emph{−}1) por (OF1), desde 1 ⩽ 0

= (\emph{−}1)+0 por comutatividade (F5)

= \emph{−}1 by (F3)

Por (OF2), segue-se que 0 ⩽ (\emph{−}1)(\emph{−}1). Mas
(\emph{−}1)(\emph{−}1) = 1 por Exercise A.2.30 e, portanto, 0 ⩽ 1. Como
1 ⩽ 0 e 0 ⩽ 1, temos 0 = 1 por antisimetria (PO2). Mas isso contradiz o
axioma (F1). Portanto, 0 ⩽ 1. Na verdade, 0 \emph{\textless{}} 1 já que
0 \emph{̸}= 1. ◁

Vimos que Q e R são campos ordenados (Examples A.2.26 and A.2.32), e que
Z\emph{/p}Z é um campo para \emph{p \textgreater{}} 0 prime (Example
A.2.22). A seguinte proposição é um resultado

interessante que prova que não existe noção de `ordenação' sob a qual o
campo Z\emph{/p}Z possa ser transformado em um campo ordenado!

✣ \textbf{Proposition A.2.34}
\end{quote}

43

\begin{quote}
44 \emph{Appendix A. Miscelânea matemática}

Seja \emph{p \textgreater{}} 0 primo. Não há relação ⩽ em Z\emph{/p}Z
que satisfaça os axiomas de campo ordenados.

\emph{\textbf{Proof}}

Acabamos de mostrar que {[}0{]} ⩽ {[}1{]}. Segue-se que, para todo
\emph{a ∈} Z, temos {[}\emph{a}{]} ⩽ {[}\emph{a}{]}+{[}1{]}; de fato:
{[}a{]} = {[}a{]}+{[}0{]} por (F3)\\
⩽ {[}\emph{a}{]}+{[}1{]}by (OF1), já que {[}0{]} ⩽ {[}1{]}\\
= {[}\emph{a} + 1{]}por definição de + em Z\emph{/p}Z É uma indução
direta provar que {[}\emph{a}{]} ⩽ {[}\emph{a} + \emph{n}{]} para todo
\emph{n ∈} N. Mas então temos
\end{quote}

{[}1{]} ⩽ {[}1+(\emph{p−}1){]} = {[}\emph{p}{]} = {[}0{]}

\begin{quote}
então {[}0{]} ⩽ {[}1{]} e {[}1{]} ⩽ {[}0{]}. Isso implica {[}0{]} =
{[}1{]} por antissimetria (PO2), contradizendo o axioma (F1). □
\end{quote}

✎ \textbf{Exercise A.2.35}

\begin{quote}
Seja (\emph{X,}0\emph{,}1\emph{,}+\emph{,·}) um campo. Prove que se
\emph{X} é finito, então não há relação ⩽ em \emph{X} tal que
(\emph{X,}0\emph{,}1\emph{,}+\emph{,·,}⩽) seja um corpo ordenado. ◁

Theorem A.2.36 abaixo resume algumas propriedades de campos ordenados
que são

usados em nossas provas. Observe, entretanto, que esta certamente
\emph{não} é uma lista ex-

austiva de propriedades elementares de campos ordenados que usamos ---
para declarar

explicitamente e provar que tudo isso não seria uma leitura brilhante.
\end{quote}

\begin{longtable}[]{@{}
  >{\raggedright\arraybackslash}p{(\columnwidth - 0\tabcolsep) * \real{1.0000}}@{}}
\toprule()
\begin{minipage}[b]{\linewidth}\raggedright
\begin{longtable}[]{@{}
  >{\raggedright\arraybackslash}p{(\columnwidth - 0\tabcolsep) * \real{1.0000}}@{}}
\toprule()
\begin{minipage}[b]{\linewidth}\raggedright
✣ \textbf{Theorem A.2.36}\\
Seja (\emph{X,}0\emph{,}1\emph{,}+\emph{,·,}⩽) seja um campo ordenado.
Então

\begin{quote}
(a) Para todo \emph{x,y ∈ X}, \emph{x} ⩽ \emph{y} se e apenas se 0 ⩽
\emph{y−x};\\
(b) Para todo \emph{x ∈ X}, \emph{−x} ⩽ 0 ⩽ \emph{x} ou \emph{x} ⩽ 0 ⩽
\emph{−x};\\
(c) Para todo \emph{x,x′,y,y′∈ X}, se \emph{x} ⩽ \emph{x′}e \emph{y} ⩽
\emph{y′}, então \emph{x}+\emph{y} ⩽ \emph{x′}+\emph{y′}; (d) Para todo
\emph{x,y,z ∈ X}, se 0 ⩽ \emph{x} e \emph{y} ⩽ \emph{z}, então \emph{xy}
⩽ \emph{xz};\\
(e) Para todo nonzero \emph{x ∈ X}, if 0 ⩽ \emph{x}, então 0 ⩽
\emph{x−}1.
\end{quote}\strut
\end{minipage} \\
\midrule()
\endhead
\bottomrule()
\end{longtable}

\begin{quote}
(f) Para todo nonzero \emph{x,y ∈ X}, if \emph{x} ⩽ \emph{y}, então
\emph{y−}1⩽ \emph{x−}1.
\end{quote}\strut
\end{minipage} \\
\midrule()
\endhead
\bottomrule()
\end{longtable}

\begin{quote}
\emph{\textbf{Proof} de (a), (b) e (e)}

(a) (\emph{⇒}) suponha \emph{x} ⩽ \emph{y}. Então, por aditividade
(OF1), \emph{x} + (\emph{−x}) ⩽ \emph{y} + (\emph{−x}), isso é 0 ⩽
\emph{y−x}. (\emph{⇐}) Suponha 0 ⩽ \emph{y−x}. Por aditividade (OF1),
0+\emph{x} ⩽ (\emph{y−x})+\emph{x}; isto é, \emph{x} ⩽ \emph{y}.
\end{quote}

44

\begin{quote}
\emph{Section A.2. Construções dos conjuntos de números} 45

(b) Sabemos por linearidade (PO4) que 0 ⩽ \emph{x} ou \emph{x} ⩽ 0. Se 0
⩽ \emph{x}, então por (OF1) temos 0 + (\emph{−x}) ⩽ \emph{x} +
(\emph{−x}), ou seja \emph{−x} ⩽ 0. Da mesma forma, se \emph{x} ⩽ 0
então 0 ⩽ \emph{−x}.

(e) Suponha 0 ⩽ \emph{x}. Por linearidade (PO4), 0 ⩽ \emph{x−}1ou
\emph{x−}1⩽ 0. Se \emph{x−}1⩽ 0, então por (d) temos \emph{x−}1\emph{·
x} ⩽ 0 \emph{· x}, que é 1 ⩽ 0. Isso contradiz Example A.2.33, então
devemos ter 0 ⩽ \emph{x−}1.

As demonstrações das demais propriedades ficam como exercício. □

Queríamos caracterizar os reais completamente, mas até agora não
conseguimos fazê-lo --- na verdade, Example A.2.32 mostrou que ambos Q e
R são campos ordenados, então os axiomas de campo ordenados não são
suficientes para distinguir Q de R. A peça final do quebra-cabeça é
\emph{completude}. Este único axioma adicional distingue Q de R, e de
fato caracteriza completamente R (veja Theorem A.2.38).
\end{quote}

\begin{longtable}[]{@{}
  >{\raggedright\arraybackslash}p{(\columnwidth - 0\tabcolsep) * \real{1.0000}}@{}}
\toprule()
\begin{minipage}[b]{\linewidth}\raggedright
\begin{longtable}[]{@{}
  >{\raggedright\arraybackslash}p{(\columnwidth - 0\tabcolsep) * \real{1.0000}}@{}}
\toprule()
\begin{minipage}[b]{\linewidth}\raggedright
✣ \textbf{Axioms A.2.37} (Axiomas de campo ordenados completos)

Seja \emph{X} um conjunto, 0\emph{,}1 \emph{∈ X} sejam elementos,
+\emph{,·} sejam operações binárias e ⩽ seja uma relação em \emph{X}. A
estrutura (\emph{X,}0\emph{,}1\emph{,}+\emph{,·,}⩽) é um \textbf{campo
ordenado completo} se for um campo ordenado --- isto é, satisfaz os
axiomas (F1) --(F10), (PO1)--(PO4) e (OF1)--

(OF2) (veja Axioms A.2.20 and A.2.31)---e, além disso, satisfaz o
seguinte \textbf{axioma de}

\begin{quote}
\textbf{completude} :

(C1) Seja \emph{A ⊆ X}. Se \emph{A} tiver um limite superior, então terá
um limite superior mínimo. Especificamente, se existe \emph{u ∈ X} tal
que \emph{a} ⩽ \emph{u} para todo \emph{a ∈ A}, então existe \emph{s ∈
X} tal que

\emph{⋄ a} ⩽ \emph{s} para todo \emph{a ∈ A}; e

\emph{⋄} Se \emph{s′∈ X} é tanto que \emph{a} ⩽ \emph{s′}para todo
\emph{a ∈ A}, então \emph{s} ⩽ \emph{s′}.
\end{quote}
\end{minipage} \\
\midrule()
\endhead
\bottomrule()
\end{longtable}

\begin{quote}
Chamamos esse valor \emph{s ∈ X} a \textbf{supremo} for \emph{A}.
\end{quote}
\end{minipage} \\
\midrule()
\endhead
\bottomrule()
\end{longtable}

\begin{longtable}[]{@{}
  >{\raggedright\arraybackslash}p{(\columnwidth - 0\tabcolsep) * \real{1.0000}}@{}}
\toprule()
\begin{minipage}[b]{\linewidth}\raggedright
✣ \textbf{Theorem A.2.38}\\
Os números reais (R\emph{,}0\emph{,}1\emph{,}+\emph{,·,}⩽) formam um
campo ordenado completo. Além disso, quaisquer dois campos ordenados
completos são essencialmente iguais.

\begin{longtable}[]{@{}
  >{\raggedright\arraybackslash}p{(\columnwidth - 0\tabcolsep) * \real{1.0000}}@{}}
\toprule()
\begin{minipage}[b]{\linewidth}\raggedright
\end{minipage} \\
\midrule()
\endhead
\bottomrule()
\end{longtable}\strut
\end{minipage} \\
\midrule()
\endhead
\bottomrule()
\end{longtable}

\begin{quote}
A noção de `semelhança' aludida em Theorem A.2.38 é mais apropriadamente
chamada de \emph{isomorfismo}. Uma prova deste teorema é complexa e está
muito além do escopo deste livro, por isso foi omitida. O que isso nos
diz é que não importa exatamente como definimos os reais, pois qualquer
campo ordenado completo serve. Podemos, portanto, prosseguir com a
confiança de que, independentemente da noção de «números reais»
\end{quote}

45

46 \emph{Appendix A. Miscelânea matemática}

que escolhermos, tudo o que provarmos será verdadeiro relativamente a
essa noção. Isso é melhor, já que na verdade ainda não definimos o
conjunto R de números reais!

As duas abordagens mais comuns para construir um conjunto de números
reais são: \textbf{Reais de Dedekind.} Nesta abordagem, os números reais
são identificados com subcon-juntos particulares de Q---falando
informalmente, \emph{r ∈} R é identificado com o conjunto de números
racionais menores que \emph{r}.

\textbf{Reais de Cauchy.} Nesta abordagem, os números reais são
identificados com classes é identificado com o conjunto de sequências de
números racionais números que con-de equivalência de sequências de
números racionais --- falando informalmente, \emph{r ∈} R vergem para
\emph{r} (no sentido de \textbf{??}).

46

\begin{quote}
\emph{Section A.3. Limites de funções} 47

Section A.3

\textbf{Limites de funções}

No final de \textbf{??} mencionamos o uso de \emph{limites} de funções
sem definir corretamente o que

queríamos dizer. Esta seção reconhecidamente brusca é dedicada a tornar
preciso o que

queremos dizer com matemática.

\textbf{Limites}
\end{quote}

\begin{longtable}[]{@{}
  >{\raggedright\arraybackslash}p{(\columnwidth - 0\tabcolsep) * \real{1.0000}}@{}}
\toprule()
\begin{minipage}[b]{\linewidth}\raggedright
✦ \textbf{Definition A.3.1}\\
Seja \emph{D ⊆} R. Um \textbf{ponto limite} de \emph{D} é um número real
\emph{a} tal que, para todo \emph{δ \textgreater{}} 0, existe algum
\emph{x ∈ D} tal que 0 \emph{\textless{} \textbar x−a\textbar{}
\textless{} δ}.\strut
\end{minipage} \\
\midrule()
\endhead
\bottomrule()
\end{longtable}

✣ \textbf{Lemma A.3.2}

\begin{quote}
Seja \emph{D ⊆} R. Um número real \emph{a} é um ponto limite de \emph{D}
se e somente se existe uma sequência (\emph{xn}) de elementos de
\emph{D}, que não é eventualmente constante, tal que (\emph{xn}) \emph{→
a} .

\emph{\textbf{Proof}}
\end{quote}

□

\begin{quote}
(\emph{⇒}) Seja \emph{a ∈} R e assuma que \emph{a} é um ponto limite de
\emph{D}. Para cada \emph{n} ⩾ 1, seja \emph{xn} algum elemento de
\emph{D} tal que 0 \emph{\textless{} \textbar xn −a\textbar{}
\textless{}}1 \emph{n}.

Evidentemente (\emph{xn}) \emph{→ a}: de fato, dado \emph{ε
\textgreater{}} 0, deixando \emph{N} ⩾ max\emph{\{}1\emph{,}\uline{1}
\emph{ε} para todos \emph{n} ⩾ \emph{N}. \emph{ε\}} dá \emph{\textbar xn
−uma\textbar{} \textless{}}

Além disso, a sequência (\emph{xn}) não é eventualmente constante: se
fosse, existiriam \emph{N} ⩾ 1

e \emph{b ∈} R tais que \emph{xn} = \emph{b} para todos \emph{n geN}.
Mas então, pelo teorema da compressão (\textbf{??}), teríamos

1\\
0 ⩽ lim \emph{n→}∞\emph{\textbar xn −a\textbar{}} =
\emph{\textbar ba\textbar{}} ⩽ lim \emph{n}= 0

e então \emph{b} = \emph{a}. Mas isso contradiz o fato de que
\emph{\textbar xn −a\textbar{} \textgreater{}} 0 para todo \emph{n} ⩾ 1.

(\emph{⇐}) Seja \emph{a ∈} R e assuma que existe uma sequência
(\emph{xn}) de elementos de \emph{D}, que não é eventualmente constante,
tal que (\emph{xn}) \emph{→ um}. Então para todo \emph{δ \textgreater{}}
0 existe algum \emph{N ∈} N tal que \emph{\textbar xn −a\textbar{}
\textless{} ε} para todos \emph{n} ⩾ \emph{N}. Como (\emph{xn}) não é
eventualmente constante, existe algum \emph{n} ⩾ \emph{N} tal que
\emph{\textbar xn − a\textbar{} \textgreater{}} 0---caso contrário
(\emph{xn}) seria eventualmente constante com valor \emph{a}! Mas então
\emph{xn ∈ D} e 0 \emph{\textless{} \textbar xn −a\textbar{} \textless{}
δ}, então \emph{a} é um ponto limite de \emph{D}.
\end{quote}

47

\begin{longtable}[]{@{}
  >{\raggedright\arraybackslash}p{(\columnwidth - 2\tabcolsep) * \real{0.5000}}
  >{\raggedright\arraybackslash}p{(\columnwidth - 2\tabcolsep) * \real{0.5000}}@{}}
\toprule()
\begin{minipage}[b]{\linewidth}\raggedright
\begin{quote}
48
\end{quote}
\end{minipage} & \begin{minipage}[b]{\linewidth}\raggedright
\emph{Appendix A. Miscelânea matemática}
\end{minipage} \\
\midrule()
\endhead
\bottomrule()
\end{longtable}

\begin{longtable}[]{@{}
  >{\raggedright\arraybackslash}p{(\columnwidth - 0\tabcolsep) * \real{1.0000}}@{}}
\toprule()
\begin{minipage}[b]{\linewidth}\raggedright
✦ \textbf{Definition A.3.3}\\
Seja \emph{D ⊆} R. O \textbf{fechamento} de \emph{D} é o conjunto
\emph{D} (LATEX code: \textbackslash overline\{D\}) definido por\\
\emph{D} = \emph{D∪\{a ∈} R \emph{\textbar{} a} é um ponto limite de
\emph{D\}}\\
Ou seja, \emph{D} é dado por \emph{D} junto com seus pontos
limites.\strut
\end{minipage} \\
\midrule()
\endhead
\bottomrule()
\end{longtable}

✐ \textbf{Example A.3.4}

\begin{longtable}[]{@{}
  >{\raggedright\arraybackslash}p{(\columnwidth - 2\tabcolsep) * \real{0.5000}}
  >{\raggedright\arraybackslash}p{(\columnwidth - 2\tabcolsep) * \real{0.5000}}@{}}
\toprule()
\multicolumn{2}{@{}>{\raggedright\arraybackslash}p{(\columnwidth - 2\tabcolsep) * \real{1.0000} + 2\tabcolsep}@{}}{%
\begin{minipage}[b]{\linewidth}\raggedright
\begin{quote}
Temos (0\emph{,}1) = {[}0\emph{,}1{]}. Na verdade, (0\emph{,}1) \emph{⊆}
(0\emph{,}1) já que \emph{D ⊆ D} para todo \emph{D ⊆} R. Além disso, as
sequências (\uline{1 \emph{n}}) e (1 \emph{−} \sout{1 \emph{n}}) não são
constantes, assumem valores em (0\emph{,}1) e

convergem para 0 e 1 respectivamente, de modo que 0 \emph{∈}
(0\emph{,}1) e 1 \emph{∈} (0\emph{,}1). Portanto {[}0\emph{,}1{]}
\emph{⊆} (0\emph{,}1).

Dado \emph{a ∈} R, se \emph{a \textgreater{}} 1, então deixar \emph{δ} =
1 \emph{− a \textgreater{}} 0 revela que \emph{\textbar x − a\textbar{}}
⩾ \emph{δ} para todo \emph{x ∈ D}; e da mesma forma, se \emph{a
\textless{}} 0, então deixar \emph{δ} = \emph{−a \textgreater{}} 0
revela que \emph{\textbar x − a\textbar{}} ⩾ \emph{δ} para todo \emph{x
∈ D}. Portanto, nenhum elemento de
R\emph{\textbackslash{}}{[}0\emph{,}1{]} é um elemento de \emph{D}, de
modo que (0\emph{,}1) = {[}0\emph{,}1{]}. ◁
\end{quote}

✎ \textbf{Exercise A.3.5}
\end{minipage}} \\
\midrule()
\endhead
\begin{minipage}[t]{\linewidth}\raggedright
\begin{quote}
Seja \emph{a,b ∈} R com \emph{a \textless{} b}. Prove que (\emph{a,b}) =
(\emph{a,b}{]} = {[}\emph{a,b}) = {[}\emph{a,b}{]}.❖ \textbf{Convention
A.3.6}
\end{quote}
\end{minipage} & ◁ \\
\multicolumn{2}{@{}>{\raggedright\arraybackslash}p{(\columnwidth - 2\tabcolsep) * \real{1.0000} + 2\tabcolsep}@{}}{%
\begin{minipage}[t]{\linewidth}\raggedright
\begin{quote}
Para o restante desta seção, sempre que declararmos \emph{f} : \emph{D
→} R como uma função, será assumido que o domínio \emph{D} é um
subconjunto de R, e que todo ponto de \emph{D} é um

ponto limite de \emph{D}. Em outras palavras, \emph{D} não possui
\emph{pontos isolados}, que são pontos

separados de todos os outros elementos de \emph{D} por uma distância
positiva. Por exemplo,
\end{quote}
\end{minipage}} \\
\begin{minipage}[t]{\linewidth}\raggedright
\begin{quote}
no conjunto (0\emph{,}1{]}\emph{∪\{}2\emph{\}}, o elemento 2 \emph{∈} R
é um ponto isolado.
\end{quote}
\end{minipage} & ◁ \\
\bottomrule()
\end{longtable}

\begin{longtable}[]{@{}
  >{\raggedright\arraybackslash}p{(\columnwidth - 0\tabcolsep) * \real{1.0000}}@{}}
\toprule()
\begin{minipage}[b]{\linewidth}\raggedright
✦ \textbf{Definition A.3.7}

Seja \emph{f} : \emph{D →} R uma função, seja \emph{a ∈ D}, e seja
\emph{ℓ ∈} R. Dizemos que \emph{ℓ} é um \textbf{limit} de
\emph{f}(\emph{x}) quando \emph{x} \textbf{se aproxima de} \emph{a} se

\emph{∀ε \textgreater{}} 0\emph{, ∃δ \textgreater{}} 0\emph{, ∀x ∈ D,} 0
\emph{\textless{} \textbar x−a\textbar{} \textless{} δ ⇒ \textbar{}
f}(\emph{x})\emph{−ℓ\textbar{} \textless{} ε}

Em outras palavras, para valores de \emph{x ∈ D} próximos de \emph{a}
(mas não iguais a \emph{a}), os valores de \emph{f}(\emph{x}) tornam-se
arbitrariamente próximos de \emph{ℓ}.

Escrevemos `\emph{f}(\emph{x}) \emph{→ ℓ} como \emph{x → a}' (LATEX
code: \textbackslash to) para denotar a afirmação de que \emph{ℓ} é um
limite de \emph{f}(\emph{x}) conforme \emph{x} se aproxima \emph{a}.
\end{minipage} \\
\midrule()
\endhead
\bottomrule()
\end{longtable}

✐ \textbf{Example A.3.8}\\
Defina \emph{f} : R \emph{→} R por \emph{f}(\emph{x}) = \emph{x} para
todo \emph{x ∈} R. Então \emph{f}(\emph{x}) \emph{→} 0 como \emph{x →}
0. Para ver isso, seja \emph{ε \textgreater{}} 0 e defina \emph{δ} =
\emph{ε \textgreater{}} 0. Então, para todo \emph{x ∈} R, se 0
\emph{\textless{} \textbar x − a\textbar{} \textless{} δ} = \emph{ε},
então\\
\emph{\textbar f}(\emph{x})\emph{− f}(\emph{uma})\emph{\textbar{}} =
\emph{\textbar x−a\textbar{} \textless{} ε}\\
como requerido. ◁ 48

\begin{quote}
\emph{Section A.3. Limites de funções} 49
\end{quote}

✎ \textbf{Exercise A.3.9}

\begin{quote}
Seja \emph{f} : \emph{D →} R uma função, seja \emph{a ∈ D} e seja
\emph{ℓ ∈} R. Corrija alguma sequência (\emph{xn}) de elementos de
\emph{D}, não eventualmente constante, tal que (\emph{xn}) \emph{→ a}.
Prove que se \emph{f}(\emph{x}) \emph{→ ℓ} como \emph{x → a}, então a
sequência ( \emph{f}(\emph{xn})) converge para \emph{ℓ}. ◁

O próximo exercício diz-nos que os limites das funções são únicos, desde
que existam.

Sua prova se parece muito com o resultado análogo que provamos para
sequências em

\textbf{??}.
\end{quote}

✎ \textbf{Exercise A.3.10}

\begin{quote}
Seja \emph{f} : \emph{D →} R uma função, seja \emph{a ∈ D}, e
\emph{ℓ}1\emph{,ℓ}2 \emph{∈} R. Prove que se \emph{f}(\emph{x}) \emph{→
ℓ}1 como \emph{x → a}, e \emph{f}(\emph{x}) \emph{→ ℓ}2 como \emph{x →
a}, então \emph{ℓ}1 = \emph{ℓ}2. ◁

Quando o domínio \emph{D} de uma função \emph{f} : \emph{D →} R é
ilimitado, também podemos estar interessados em descobrir como os
valores de \emph{f}(\emph{x}) se comportam como \emph{x ∈ D} fica
(positiva ou negativamente) cada vez maior.
\end{quote}

\begin{longtable}[]{@{}
  >{\raggedright\arraybackslash}p{(\columnwidth - 0\tabcolsep) * \real{1.0000}}@{}}
\toprule()
\begin{minipage}[b]{\linewidth}\raggedright
✦ \textbf{Definition A.3.11}\\
Seja \emph{f} : \emph{D →} R uma função e seja \emph{ℓ ∈} R. Se \emph{D}
é ilimitado acima --- isto é, para todo

\textbf{aumenta sem limites} se \emph{p ∈} R, existe \emph{x ∈ D} com
\emph{x \textgreater{} p} --- então dizemos \emph{ℓ} é um
\textbf{limite} de \emph{f}(\emph{x}) à medida que \emph{x}

\begin{quote}
\emph{∀ε \textgreater{}} 0\emph{, ∃p ∈} R\emph{, ∀x ∈ D, x
\textgreater{} p ⇒ \textbar{} f}(\emph{x})\emph{−ℓ\textbar{} \textless{}
ε}\\
Escrevemos `\emph{f}(\emph{x}) \emph{→ ℓ} as \emph{x →} ∞' (LATEX code:
\textbackslash infty) para denotar a afirmação de que \emph{ℓ} é um
limite de \emph{f}(\emph{x}) como \emph{x} aumenta sem limite.
\end{quote}

Da mesma forma, se \emph{D} é ilimitado abaixo --- isto é, para todo
\emph{p ∈} R, existe \emph{x ∈ D} com \emph{x \textless{} p}--- então
dizemos \emph{ell} é um \textbf{limite} de \emph{f}(\emph{x}) conforme
\emph{x} \textbf{diminui sem limite} se

\begin{quote}
\emph{∀ε \textgreater{}} 0\emph{, ∃p ∈} R\emph{, ∀x ∈ D, x \textless{} p
⇒ \textbar{} f}(\emph{x})\emph{−ℓ\textbar{} \textless{} ε}\\
Escrevemos `\emph{f}(\emph{x}) \emph{→ ℓ} como \emph{x → −}∞' para
denotar a afirmação de que \emph{ℓ} é um limite de \emph{f}(\emph{x}) à
medida que \emph{x} diminui sem limite.
\end{quote}\strut
\end{minipage} \\
\midrule()
\endhead
\bottomrule()
\end{longtable}

✐ \textbf{Example A.3.12}\\
Seja \emph{f} : R \emph{→} R a função definida por \emph{f}(\emph{x}) =

\begin{quote}
\emph{x}\\
1+\emph{\textbar x\textbar{}}para todo \emph{x ∈} R. Então:
\end{quote}

\begin{longtable}[]{@{}
  >{\raggedright\arraybackslash}p{(\columnwidth - 6\tabcolsep) * \real{0.2500}}
  >{\raggedright\arraybackslash}p{(\columnwidth - 6\tabcolsep) * \real{0.2500}}
  >{\raggedright\arraybackslash}p{(\columnwidth - 6\tabcolsep) * \real{0.2500}}
  >{\raggedright\arraybackslash}p{(\columnwidth - 6\tabcolsep) * \real{0.2500}}@{}}
\toprule()
\multicolumn{4}{@{}>{\raggedright\arraybackslash}p{(\columnwidth - 6\tabcolsep) * \real{1.0000} + 6\tabcolsep}@{}}{%
\begin{minipage}[b]{\linewidth}\raggedright
\begin{quote}
\emph{f}(\emph{x}) \emph{→} 1 como \emph{x →} ∞. Para ver isso, seja
\emph{ε \textgreater{}} 0 e defina \emph{p} =
max\emph{\{}1\emph{,}\uline{1} \emph{x} \emph{ε\}}. Então, para

todo \emph{x \textgreater{} p}, temos \emph{x \textgreater{}} 0, de modo
que \emph{f}(\emph{x}) = 1+\emph{x}, e \emph{x \textgreater{}} \sout{1
\emph{ε}}\emph{−}1. Por isso:
\end{quote}
\end{minipage}} \\
\midrule()
\endhead
\emph{\textbar{}} & 1+\emph{x−}1\emph{\textbar{}} = \emph{\textbar{} −}1
1+\emph{x\textbar{}} = & \begin{minipage}[t]{\linewidth}\raggedright
\begin{quote}
1\\
1+\emph{x\textless{}}
\end{quote}\strut
\end{minipage} & \begin{minipage}[t]{\linewidth}\raggedright
\begin{quote}
1\\
1+(\uline{1 \emph{ε}}\emph{−}1) = \emph{ε}
\end{quote}\strut
\end{minipage} \\
\bottomrule()
\end{longtable}

49

\begin{quote}
50 \emph{Appendix A. Miscelânea matemática}

como requerido.
\end{quote}

\begin{longtable}[]{@{}
  >{\raggedright\arraybackslash}p{(\columnwidth - 10\tabcolsep) * \real{0.1667}}
  >{\raggedright\arraybackslash}p{(\columnwidth - 10\tabcolsep) * \real{0.1667}}
  >{\raggedright\arraybackslash}p{(\columnwidth - 10\tabcolsep) * \real{0.1667}}
  >{\raggedright\arraybackslash}p{(\columnwidth - 10\tabcolsep) * \real{0.1667}}
  >{\raggedright\arraybackslash}p{(\columnwidth - 10\tabcolsep) * \real{0.1667}}
  >{\raggedright\arraybackslash}p{(\columnwidth - 10\tabcolsep) * \real{0.1667}}@{}}
\toprule()
\multicolumn{6}{@{}>{\raggedright\arraybackslash}p{(\columnwidth - 10\tabcolsep) * \real{1.0000} + 10\tabcolsep}@{}}{%
\begin{minipage}[b]{\linewidth}\raggedright
\begin{quote}
\emph{f}(\emph{x}) \emph{→ −}1 como \emph{x →} ∞. Para ver isso, seja
\emph{ε \textgreater{}} 0 e defina \emph{p} =
min\emph{\{−}1\emph{,\sout{−}}\uline{1 \emph{ε}}\emph{\}}. Então,

para todo \emph{x \textless{} p}, temos \emph{x \textless{}} 0, de modo
que \emph{f}(\emph{x}) = 1\emph{−x}, e \emph{x \textless{}
\sout{−}}\uline{1 \emph{ε}}+1. Por isso:
\end{quote}
\end{minipage}} \\
\midrule()
\endhead
\emph{\textbar{}} & \begin{minipage}[t]{\linewidth}\raggedright
\begin{quote}
\emph{x}\\
1\emph{−x−}(\emph{−}1)\emph{\textbar{}} = \emph{\textbar{}}
\end{quote}\strut
\end{minipage} & \begin{minipage}[t]{\linewidth}\raggedright
\begin{quote}
1\\
1\emph{−x\textbar{}} =
\end{quote}\strut
\end{minipage} & \begin{minipage}[t]{\linewidth}\raggedright
\begin{quote}
1\\
1\emph{−x\textless{}}
\end{quote}\strut
\end{minipage} &
\multicolumn{2}{>{\raggedright\arraybackslash}p{(\columnwidth - 10\tabcolsep) * \real{0.3333} + 2\tabcolsep}@{}}{%
\begin{minipage}[t]{\linewidth}\raggedright
\begin{quote}
1\\
1\emph{−}(\emph{−}\uline{1 \emph{ε}}+1) = \emph{ε}
\end{quote}\strut
\end{minipage}} \\
\multicolumn{5}{@{}>{\raggedright\arraybackslash}p{(\columnwidth - 10\tabcolsep) * \real{0.8333} + 8\tabcolsep}}{%
como requerido. Então \emph{f}(\emph{x}) \emph{→} 1 como \emph{x →} ∞ e
\emph{f}(\emph{x}) \emph{→ −}1 como \emph{x → −}∞.✎ \textbf{Exercise
A.3.13}} & ◁ \\
\multicolumn{6}{@{}>{\raggedright\arraybackslash}p{(\columnwidth - 10\tabcolsep) * \real{1.0000} + 10\tabcolsep}@{}}{%
\begin{minipage}[t]{\linewidth}\raggedright
\begin{quote}
Seja \emph{f} : \emph{D →} R uma função e \emph{ℓ}1\emph{,ℓ}2 \emph{∈}
R. Prove que se \emph{D} é ilimitado acima, e se análogo para limites
como \emph{x → −}∞ no caso em que \emph{D} é ilimitado abaixo.
\emph{f}(\emph{x}) \emph{→ ℓ}1 como \emph{x →} ∞ e \emph{f}(\emph{x})
\emph{→ ℓ}2 como \emph{x →} ∞, então \emph{ℓ}1 = \emph{ℓ}2. Prove o
resultado◁
\end{quote}
\end{minipage}} \\
\bottomrule()
\end{longtable}

\begin{quote}
Os resultados de Exercises A.3.10 and A.3.13 justificam a seguinte
definição.
\end{quote}

\begin{longtable}[]{@{}
  >{\raggedright\arraybackslash}p{(\columnwidth - 0\tabcolsep) * \real{1.0000}}@{}}
\toprule()
\begin{minipage}[b]{\linewidth}\raggedright
\begin{longtable}[]{@{}
  >{\raggedright\arraybackslash}p{(\columnwidth - 0\tabcolsep) * \real{1.0000}}@{}}
\toprule()
\begin{minipage}[b]{\linewidth}\raggedright
✦ \textbf{Definition A.3.14}\\
Seja \emph{f} : \emph{D →} R e seja \emph{a ∈} {[}\emph{−}∞\emph{,}∞{]}.
Supondo que os limites em questão sejam bem definidos e existam,
escrevemos lim \emph{x→af}(\emph{x}) para denotar o único número real
\emph{ℓ ∈} R tal\strut
\end{minipage} \\
\midrule()
\endhead
\bottomrule()
\end{longtable}

\begin{quote}
que \emph{f}(\emph{x}) \emph{→ ℓ} como \emph{x → a}.
\end{quote}\strut
\end{minipage} \\
\midrule()
\endhead
\bottomrule()
\end{longtable}

50

Appendix B\\
\textbf{Dicas para exercícios selecion-ados}

\textbf{Chapter 0}\\
\textbf{?? exercises}\\
\textbf{Section A.1}\\
\textbf{Section A.2}\\
\textbf{Hint for Exercise A.2.9}\\
Defina duas relações ⩽ e ⩽\emph{′}em N, uma usando a definição recursiva
e a outra usando a definição como fórmula lógica; então use a indução
para provar que \emph{m} ⩽ \emph{n ⇔ m} ⩽\emph{′n} para todo \emph{m,n
∈} N.

\textbf{Hint for Exercise A.2.28}\\
Prove que \emph{x} é um inverso aditivo para \emph{−x} (no sentido de
Axioms A.2.20(F4)) e use a unicidade dos inversos aditivos. Da mesma
forma para \emph{x−}1.

\textbf{Section A.3}

51

52 \emph{Appendix B. Hints for selected exercises}

52

\textbf{Índices}

53

\textbf{Index of topics}

axioma da completude, 45

base-\emph{b} expansion, 6

campo, 40\\
campo ordenado completo, 45

discriminante, 18\\
divisor, 8\\
divisão, 8

fator, 8

\begin{quote}
número, 29\\
number base, 6\\
numéro irracional, 14\\
número natural, 29\\
número natural\\
von Neumann, 29\\
número natural de von Neumann , 29

polinômios, 17\\
proposição, 1\\
prova, 1
\end{quote}

\begin{longtable}[]{@{}
  >{\raggedright\arraybackslash}p{(\columnwidth - 2\tabcolsep) * \real{0.5000}}
  >{\raggedright\arraybackslash}p{(\columnwidth - 2\tabcolsep) * \real{0.5000}}@{}}
\toprule()
\begin{minipage}[b]{\linewidth}\raggedright
Inteiro\\
par, 9\\
´impar, 9\strut
\end{minipage} & \begin{minipage}[b]{\linewidth}\raggedright
\begin{quote}
quociente, 10\\
raiz, 18\\
restante, 10
\end{quote}\strut
\end{minipage} \\
\midrule()
\endhead
\bottomrule()
\end{longtable}

limites de umaa função, 48 múltiplo, 8

\begin{quote}
sistema de numeração\\
Hindu--árabe, 5\\
sistema numérico, 5
\end{quote}

natural teorema da divisão, 10

55

56 \emph{Index of topics}

56

\textbf{Index of notation}

\emph{n}vN --- \emph{numéro natural de von}

\begin{quote}
\emph{neuman}, 29
\end{quote}

57

58 \emph{Index of notation}

58

\textbf{Index of LATEX commands} Math mode commands

\begin{quote}
\textbackslash ge, ⩾ , 13

\textbackslash in, \emph{∈}, 3 \textbackslash le, ⩽, 13

\textbackslash mathbb, A\emph{,}B\emph{,...}, 4, 8, 12, 15
\end{quote}

59

60 \emph{Index of LATEX commands}

60

\textbf{Licence}

Este livro, tanto em formato físico quanto eletrônico, está licenciado
sob uma Li-cença Pública Internacional Creative Commons
Attribution-ShareAlike 4.0. A licença é replicada abaixo no site
Creative Commons: Por favor, leia o conteúdo desta licença com r ou
adaptar o material deste livro.

Se você tiver dúvidas ou quiser solicitar permissões não concedidas por
esta licença, entre em contato com o autor (Clive Newstead,
clive@infinitedescent.xyz).

\textbf{Licença Pública Internacional Creative Commons
Attribution-ShareAlike 4.0}

Ao exercer os Direitos Licenciados (definidos abaixo), Você aceita e
concorda em ficar vinculado aos termos e condições desta Licença Pública
Internacional Creative Com-mons Attribution-ShareAlike 4.0 (``Licença
Pública''). Na medida em que esta Licença Pública possa ser interpretada
como um contrato, são concedidos a Você os Direitos Li-cenciados em
consideração à Sua aceitação destes termos e condições, e o Licenciador
concede a Você tais direitos em consideração aos benefícios que o
Licenciador recebe ao disponibilizar o Material Licenciado. sob estes
termos e condições.

\textbf{Section 1 --- Definitions.}

\begin{quote}
a. \textbf{Material Adaptado} significa material sujeito a Direitos
Autorais e Direitos Simil-ares que é derivado ou baseado no Material
Licenciado e no qual o Material Licen-ciado é traduzido, alterado,
organizado, transformado ou de outra forma modificado de uma maneira que
requer permissão sob os Direitos Autorais e Similares. Direitos detidos
pelo Licenciante. Para os fins desta Licença Pública, quando o Material
\end{quote}

61

62 \emph{Index of LATEX commands}

\begin{quote}
Licenciado for uma obra musical, performance ou gravação sonora, o
Material Ad-aptado será sempre produzido quando o Material Licenciado
for sincronizado em relação temporal com uma imagem em movimento.

b. \textbf{Licença do Adaptador} significa a licença que Você aplica aos
Seus Direitos Auto-rais e Direitos Similares em Suas contribuições ao
Material Adaptado de acordo com os termos e condições desta Licença
Pública.

c. \textbf{Licença compatível com BY-SA} significa uma licença listada
em\\
, aprovada pela Creative Commons como a Pública Licença.

d. \textbf{Direitos Autorais e Direitos Similares} significa direitos
autorais e/ou direitos semelhantes intimamente relacionados aos direitos
autorais, incluindo, sem limit-ação, desempenho, transmissão, gravação
de som e Direitos de Banco de Dados Sui Generis, independentemente de
como os direitos são rotulados ou categorizados. Para os fins desta
Licença Pública, os direitos especificados na Seção 2(b)(1)-(2) não são
Direitos Autorais e Direitos Similares.

e. \textbf{Medidas Tecnológicas Eficazes} significa aquelas medidas que,
na ausência de autoridade adequada, não podem ser contornadas sob leis
que cumprem obrigações sob o Artigo 11 do Tratado de Direitos Autorais
da OMPI adotado em 20 de dezem-bro de 1996, e/ou acordos internacionais
semelhantes .

f. \textbf{Exceções e Limitações} significa uso justo, negociação justa
e/ou qualquer outra exceção ou limitação aos Direitos Autorais e
Direitos Similares que se aplica ao Seu uso do Material Licenciado.

g. \textbf{Elementos de Licença} significa os atributos de licença
listados no nome de uma Li-cença Pública Creative Commons. Os Elementos
de Licença desta Licença Pública são Atribuição e Compartilhamento pela
mesma Licença.

h. \textbf{Material Licenciado} significa a obra artística ou literária,
banco de dados ou outro material ao qual o Licenciador aplicou esta
Licença Pública.

i. \textbf{Direitos Licenciados} significa os direitos concedidos a Você
sujeitos aos termos e condições desta Licença Pública, que são limitados
a todos os Direitos Autorais e Direitos Similares que se aplicam ao Seu
uso do Material Licenciado e que o Licenciador tem autoridade para
licença.

j. \textbf{Licenciador} significa o(s) indivíduo(s) ou entidade(s) que
concedem direitos sob esta Licença Pública.

k. \textbf{Compartilhar} significa fornecer material ao público por
qualquer meio ou pro-cesso que exija permissão sob os Direitos
Licenciados, como reprodução, exib-ição pública, apresentação pública,
distribuição, disseminação, comunicação ou im-portação, e tornar
material disponível ao público, inclusive de maneiras pelas quais
\end{quote}

62

\emph{Index of LATEX commands} 63

\begin{quote}
os membros do público possam acessar o material em um local e em um
horário escolhido individualmente por eles.

l. \textbf{Direitos Sui Generis sobre Bases de Dados} significa outros
direitos que não os direitos de autor resultantes da Directiva 96/9/CE
do Parlamento Europeu e do Con-selho, de 11 de Março de 1996, relativa à
protecção jurídica das bases de dados, conforme alterada e/ou sucedida,
bem como outros essencialmente direitos equi-valentes em qualquer parte
do mundo.
\end{quote}

m. \textbf{Você} significa o indivíduo ou entidade que exerce os
Direitos Licenciados sob esta Licença Pública. \textbf{Seu} tem um
significado correspondente.

\textbf{Seção 2 - Escopo..}

\begin{quote}
a. \textbf{Concessão de licença.}

1. Sujeito aos termos e condições desta Licença Pública, o Licenciador
concede a Você uma licença mundial, isenta de royalties, não
sublicenciável, não exclusiva e irrevogável para exercer os Direitos
Licenciados no Material Licenciado para:

A. reproduzir e compartilhar o Material Licenciado, no todo ou em parte;
e

B. produzir, reproduzir e compartilhar material adaptado.

2. Exceções e Limitações. Para evitar dúvidas, onde Exceções e
Limitações se ap-licam ao Seu uso, esta Licença Pública não se aplica e
Você não precisa cumprir seus termos e condições.

3. \uline{Termo}. O prazo desta Licença Pública está especificado na
Seção 6(a).

4. Mídias e formatos; modificações técnicas permitidas. O Licenciante
autoriza Você a exercer os Direitos Licenciados em todas as mídias e
formatos, sejam agora conhecidos ou criados futuramente, e a fazer as
modificações técnicas ne- cessárias para fazê-lo. O Licenciador renuncia
e/ou concorda em não reivindicar qualquer direito ou autoridade para
proibi-lo de fazer modificações técnicas ne- cessárias para exercer os
Direitos Licenciados, incluindo modificações técnicas necessárias para
contornar Medidas Tecnológicas Eficazes. Para os fins desta Licença
Pública, simplesmente fazer modificações autorizadas por esta Seção
2(a)(4) nunca produz Material Adaptado.

5. Destinatários a jusante.
\end{quote}

\begin{longtable}[]{@{}
  >{\raggedright\arraybackslash}p{(\columnwidth - 2\tabcolsep) * \real{0.5000}}
  >{\raggedright\arraybackslash}p{(\columnwidth - 2\tabcolsep) * \real{0.5000}}@{}}
\toprule()
\begin{minipage}[b]{\linewidth}\raggedright
A. \emph{Oferta do Licenciador -- Material Licenciado.}
\end{minipage} & \begin{minipage}[b]{\linewidth}\raggedright
\begin{quote}
\emph{Cada destinatário do Ma-}
\end{quote}
\end{minipage} \\
\midrule()
\endhead
\multicolumn{2}{@{}>{\raggedright\arraybackslash}p{(\columnwidth - 2\tabcolsep) * \real{1.0000} + 2\tabcolsep}@{}}{%
\begin{minipage}[t]{\linewidth}\raggedright
\begin{quote}
\emph{terial Licenciado recebe automaticamente uma oferta do Licenciante
para}
\end{quote}
\end{minipage}} \\
\bottomrule()
\end{longtable}

\begin{quote}
\emph{exercer os Direitos Licenciados sob os termos e condições desta
Licença Pública.}
\end{quote}

63

\begin{longtable}[]{@{}
  >{\raggedright\arraybackslash}p{(\columnwidth - 4\tabcolsep) * \real{0.3333}}
  >{\raggedright\arraybackslash}p{(\columnwidth - 4\tabcolsep) * \real{0.3333}}
  >{\raggedright\arraybackslash}p{(\columnwidth - 4\tabcolsep) * \real{0.3333}}@{}}
\toprule()
\multirow{3}{*}{\begin{minipage}[b]{\linewidth}\raggedright
64
\end{minipage}} &
\multicolumn{2}{>{\raggedright\arraybackslash}p{(\columnwidth - 4\tabcolsep) * \real{0.6667} + 2\tabcolsep}@{}}{%
\begin{minipage}[b]{\linewidth}\raggedright
\emph{Index of LATEX commands}
\end{minipage}} \\
& \begin{minipage}[b]{\linewidth}\raggedright
\begin{quote}
B. \emph{Oferta adicional do Licenciante -- Material Adaptado.}
\end{quote}
\end{minipage} & \begin{minipage}[b]{\linewidth}\raggedright
\begin{quote}
\emph{Cada destinatário}
\end{quote}
\end{minipage} \\
&
\multicolumn{2}{>{\raggedright\arraybackslash}p{(\columnwidth - 4\tabcolsep) * \real{0.6667} + 2\tabcolsep}@{}}{%
\begin{minipage}[b]{\linewidth}\raggedright
\begin{quote}
\emph{do Material Adaptado seu recebe automaticamente uma oferta do
Licen-ciante para exercer os Direitos Licenciados no Material Adaptado
sob as}
\end{quote}
\end{minipage}} \\
\midrule()
\endhead
\bottomrule()
\end{longtable}

\begin{quote}
\emph{condições da Licença do Adaptador que Você aplica.}

C. \emph{Sem restrições downstream. Você não poderá oferecer ou impor
quaisquer termos ou condições adicionais ou diferentes, ou aplicar
quaisquer Medidas Tecnológicas Eficazes ao Material Licenciado, se isso
restringir o exercício dos Direitos Licenciados por qualquer
destinatário do Material Licenciado.}

6. \uline{Sem endosso}. Nada nesta Licença Pública constitui ou pode ser
interpretado como permissão para afirmar ou sugerir que Você está, ou
que Seu uso do Ma-terial Licenciado está, conectado ou patrocinado,
endossado ou concedido status oficial pelo Licenciador ou outros
designados para receber atribuição conforme previsto na Seção
3(a)(1)(A)(i).

b. \textbf{Outros direitos.}

1. Os direitos morais, como o direito à integridade, não são licenciados
sob esta Li-cença Pública, nem a publicidade, a privacidade e/ou outros
direitos de person-alidade semelhantes; no entanto, na medida do
possível, o Licenciante renuncia e/ou concorda em não fazer valer
quaisquer direitos detidos pelo Licenciante na medida limitada
necessária para permitir que Você exerça os Direitos Licencia-dos, mas
não de outra forma.

2. Os direitos de patentes e marcas registradas não são licenciados sob
esta Licença Pública.

3. Na medida do possível, o Licenciante renuncia a qualquer direito de
cobrar royalties de Você pelo exercício dos Direitos Licenciados, seja
diretamente ou por meio de uma sociedade de gestão coletiva sob qualquer
esquema de licen-ciamento obrigatório ou legal voluntário ou
dispensável. Em todos os outros casos, o Licenciante reserva-se
expressamente qualquer direito de cobrar tais royalties.
\end{quote}

\textbf{Seção 3 -- Condições de Licença.}

O exercício dos Direitos Licenciados está expressamente sujeito às
seguintes condições.

\begin{quote}
a. \textbf{Atribuição.}

1. Se você compartilhar o Material Licenciado (inclusive na forma
modificada), deverá:

A. reterá o seguinte se for fornecido pelo Licenciador com o Material
Licen-ciado: identificação do(s) criador(es) do Material Licenciado e
quaisquer
\end{quote}

64

\emph{Index of LATEX commands} 65

\begin{quote}
outros designados para receber atribuição, de qualquer forma razoável
soli-citada pelo Licenciante (inclusive por pseudônimo, se designado);
um aviso de direitos autorais;\\
i. um aviso referente a esta Licença Pública;

ii. um aviso que se refere à isenção de garantias;

iii. um URI ou hiperlink para o Material Licenciado na medida razoavel-
mente praticável;

B. indicar se Você modificou o Material Licenciado e reter uma indicação
de quaisquer modificações anteriores; e

C. indica que o Material Licenciado está licenciado sob esta Licença
Pública e inclui o texto, ou o URI ou hiperlink para esta Licença
Pública.

2. Você pode satisfazer as condições da Seção 3(a)(1) de qualquer
maneira razoável com base no meio, meio e contexto em que Você
Compartilha o Material Licen-ciado. Por exemplo, pode ser razoável
satisfazer as condições fornecendo um URI ou hiperlink para um recurso
que inclua as informações necessárias.

3. Se solicitado pelo Licenciador, Você deverá remover qualquer
informação exi- gida pela Seção 3(a)(1)(A) na medida razoavelmente
praticável.

b. \textbf{ShareAlike.}

Além das condições da Seção 3(a), se você compartilhar material adaptado
que você produz, as condições a seguir também se aplicam.

1. A licença do adaptador que você aplica deve ser uma licença Creative
Commons com os mesmos elementos de licença, desta versão ou posterior,
ou uma licença compatível com BY-SA.

2. Você deve incluir o texto ou o URI ou hiperlink para a Licença do
Adaptador que Você aplica. Você pode satisfazer esta condição de
qualquer maneira razoável com base no meio, meio e contexto em que você
compartilha o Material Ad-aptado.

3. Você não pode oferecer ou impor quaisquer termos ou condições
adicionais ou diferentes, nem aplicar quaisquer Medidas Tecnológicas
Eficazes ao Material Adaptado que restrinja o exercício dos direitos
concedidos sob a Licença do Adaptador que você aplica.
\end{quote}

\textbf{Seção 4 -- Direitos Sui Generis sobre Banco de Dados.}

Onde os Direitos Licenciados incluem Direitos de Banco de Dados Sui
Generis que se aplicam ao Seu uso do Material Licenciado:

\begin{quote}
a. para evitar dúvidas, a Seção 2(a)(1) concede a Você o direito de
extrair, reutilizar, reproduzir e compartilhar todo ou uma parte
substancial do conteúdo do banco de dados;
\end{quote}

65

66 \emph{Index of LATEX commands}

\begin{quote}
b. se Você incluir todo ou uma parte substancial do conteúdo do banco de
dados em um banco de dados no qual Você tem Direitos de Banco de Dados
Sui Generis, então o banco de dados no qual Você tem Direitos de Banco
de Dados Sui Generis (mas não seu conteúdo individual) é Material
Adaptado, inclusive para fins da Seção 3(b); e

c. Você deverá cumprir as condições da Seção 3(a) se compartilhar todo
ou uma parte substancial do conteúdo do banco de dados.
\end{quote}

Para evitar dúvidas, esta Seção 4 complementa e não substitui Suas
obrigações sob esta Licença Pública quando os Direitos Licenciados
incluem outros Direitos Autorais e Direitos Similares.

\textbf{Seção 5 -- Isenção de Garantias e Limitação de
Responsabilidade.}

\begin{quote}
\textbf{a. A menos que de outra forma realizado separadamente pelo
Licenciador, na medida do possível, o Licenciador oferece o Material
Licenciado tal como está e conforme disponível, e não faz representações
ou garantias de qualquer tipo em relação ao Material Licenciado, sejam
expressas, implícitas, estatutárias ou outras . Isto inclui, sem
limitação, garantias de título, comercialização, ad-equação a uma
finalidade específica, não violação, ausência de defeitos latentes ou
outros, precisão ou presença ou ausência de erros, conhecidos ou
detectá-veis ou não. Quando isenções de garantia não forem permitidas,
no todo ou em parte, esta isenção de responsabilidade poderá não se
aplicar a Você.}

\textbf{b. Na medida do possível, em nenhum caso o Licenciador será
responsável per-ante Você em qualquer teoria legal (incluindo, sem
limitação, negligência) ou de outra forma por quaisquer perdas diretas,
especiais, indiretas, incidentais, consequenciais, punitivas, exemplares
ou outras , custos, despesas ou danos de-correntes desta Licença Pública
ou do uso do Material Licenciado, mesmo que o Licenciante tenha sido
avisado da possibilidade de tais perdas, custos, despe-sas ou danos.
Quando uma limitação de responsabilidade não for permitida, total ou
parcialmente, esta limitação poderá não se aplicar a Você.}

c. A isenção de garantias e limitação de responsabilidade fornecidas
acima devem ser interpretadas de uma maneira que, na medida do possível,
se aproxime mais de uma isenção de responsabilidade absoluta e renúncia
de toda responsabilidade.
\end{quote}

\textbf{Seção 6 -- Vigência e Rescisão.}

\begin{quote}
a. Esta Licença Pública se aplica pela vigência dos Direitos Autorais e
Similares aqui licenciados. No entanto, se Você não cumprir esta Licença
Pública, Seus direitos sob esta Licença Pública terminarão
automaticamente.
\end{quote}

66

\emph{Index of LATEX commands} 67

\begin{quote}
b. Quando o Seu direito de usar o Material Licenciado for rescindido de
acordo com a Seção 6(a), ele restabelecerá:

{[}1.{]}

c. automaticamente a partir da data em que a violação for sanada, desde
que seja sanada dentro de 30 dias após a descoberta da violação; ou

d. mediante reintegração expressa pela Licenciante.

Para evitar dúvidas, esta Seção 6(b) não afeta nenhum direito que o
Licenciador possa ter de buscar soluções para Suas violações desta
Licença Pública.

e. Para evitar dúvidas, o Licenciante também poderá oferecer o Material
Licenciado sob termos ou condições separados ou interromper a
distribuição do Material Li-cenciado a qualquer momento; no entanto,
isso não encerrará esta Licença Pública.

f. As Seções 1, 5, 6, 7 e 8 sobreviverão ao término desta Licença
Pública.
\end{quote}

\textbf{Seção 7 -- Outros Termos e Condições.}

\begin{quote}
a. O Licenciante não estará vinculado a quaisquer termos ou condições
adicionais ou diferentes comunicados por Você, a menos que expressamente
acordado.

b. Quaisquer acordos, entendimentos ou acordos relativos ao Material
Licenciado não declarados aqui são separados e independentes dos termos
e condições desta Li-cença Pública.
\end{quote}

\textbf{Seção 8 -- Interpretação.}

\begin{quote}
a. Para evitar dúvidas, esta Licença Pública não reduz e não deve ser
interpretada como reduzindo, limitando, restringindo ou impondo
condições a qualquer uso do Material Licenciado que possa ser feito
legalmente sem permissão sob esta Licença Pública.

b. Na medida do possível, se qualquer disposição desta Licença Pública
for consid-erada inexequível, ela deverá ser automaticamente reformada
na medida mínima necessária para torná-la exequível. Se a disposição não
puder ser reformada, ela será separada desta Licença Pública sem afetar
a aplicabilidade dos demais termos e condições.

c. Nenhum termo ou condição desta Licença Pública será renunciado e
nenhum des-cumprimento será consentido, a menos que expressamente
acordado pelo Licen-ciador.
\end{quote}

67

68 \emph{Index of LATEX commands}

d. Nada nesta Licença Pública constitui ou pode ser interpretado como
uma limitação

\begin{quote}
ou renúncia a quaisquer privilégios e imunidades que se apliquem ao
Licenciador
\end{quote}

ou a Você, inclusive dos processos legais de qualquer jurisdição ou
autoridade.

68

\end{document}
